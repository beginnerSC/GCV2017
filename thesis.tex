\documentclass[12pt]{article}


%\title{Improving Pricing Accuracy of Various Numerical Methods with the Control Variate Technique}
\title{Extension of the Control Variate Technique to Option Pricing Methods Beyond Monte Carlo Simulation}
\date{}%\today}

%replace Levy by L\'evy


\usepackage[latin1]{inputenc}
\usepackage[margin=1in]{geometry}
\usepackage{graphicx}
% \usepackage{latexsym}
% \usepackage{algorithm}
% \usepackage{algorithmic}
\usepackage{epsfig}
\usepackage{color}
% \usepackage{rotating}
\usepackage{natbib}
\usepackage{algorithm}
\usepackage{algorithmic}
\usepackage{amsmath}
\usepackage{amsfonts}
\usepackage{amssymb}
\usepackage{hyperref}
\usepackage{multicol}
\usepackage{setspace}
\usepackage{booktabs}
\usepackage{framed}


\newtheorem{theorem}{Theorem}[section]
\newtheorem{lemma}[theorem]{Lemma}
\newtheorem{proposition}[theorem]{Proposition}
\newtheorem{corollary}[theorem]{Corollary}

\newenvironment{proof}[1][Proof]{\begin{trivlist}
\item[\hskip \labelsep {\bfseries #1}]}{\end{trivlist}}
\newenvironment{definition}[1][Definition]{\begin{trivlist}
\item[\hskip \labelsep {\bfseries #1}]}{\end{trivlist}}
\newenvironment{example}[1][Example]{\begin{trivlist}
\item[\hskip \labelsep {\bfseries #1}]}{\end{trivlist}}
\newenvironment{remark}[1][Remark]{\begin{trivlist}
\item[\hskip \labelsep {\bfseries #1}]}{\end{trivlist}}

\newcommand{\qed}{\nobreak \ifvmode \relax \else
      \ifdim\lastskip<1.5em \hskip-\lastskip
      \hskip1.5em plus0em minus0.5em \fi \nobreak
      \vrule height0.75em width0.5em depth0.25em\fi}




\author{Student: Chun-Yuan Chiu\qquad Advisor: Dr. Tian-Shyr Dai}



\begin{document}
\maketitle

{\large
\begin{center}
Institute of Information Management\\
%College of Management\\
National Chiao Tung University\\
\end{center}
}

$$$$

\begin{abstract}
Although mostly used in connection with the Monte Carlo simulation,
the control-variate technique has applications in a surprisingly large class of numerical algorithms in option pricing.
This paper studies the condition under which a numerical method can benefit from the control variate technique
and the kind of approximation that can serve as a control variate.
We demonstrate the ideas to Fourier transform-based pricing algorithms, convolution-based pricing algorithms, and binomial trees.
Numerical results are provided to show the efficiency and accuracy of the control variate-enhanced algorithms.
\end{abstract}

\newpage
\tableofcontents
\newpage

\doublespacing


\section{Introduction}


The problem of option pricing is an important subject in the field of financial engineering,
not only because some options have very large daily trading volume,
but also because many important quantities (e.g., the probability of default of a company or the value of many popular financial derivatives) reduce to option pricing problems.
Under the most popular and analytically tractable Black-Scholes model,
for some European options such as binary, call and put options, and some path-dependent options like barrier options, there are closed-form pricing formulas.
For others like American options and Asian options, however, there are not.
There are also option pricing models under more realistic assumptions, such as the stochastic-volatility model developed by \citet{heston1993closed},
and models based on L\'evy-type jump process by \citet{merton1976option}, \citet{kou2002jump}, \citet{carr2002fine}.
Many of these models do not have excellent analytical tractability like the Black-Scholes model;
Even the simplest call option does not have a closed-form pricing formula.
When deriving a closed-form formula is not possible, financial mathematicians usually work toward two directions
-- deriving analytical approximation formulas, or applying numerical pricing algorithms.



Analytical approximations and numerical methods both have their own advantages and disadvantages.
Applying an analytical approximation, all the computations is done efficiently, but typically the result is not very accurate.
Numerical pricing algorithms can be time-consuming but one can always get more accuracy given more computational time or resources.



The control-variate (CV) technique is standard in reducing the variance of the results of Monte Carlo simulation.
By combining simulation with an analytical formula,
CV is often able to reduce the variance and hence raise the efficiency of the original algorithm.
A good example by \citet{kemna1990pricing} is arithmetic Asian options pricing with Monte Carlo simulation with the geometric Asian option as the control variate.
By combining simulation for the arithmetic Asian option
(which does not have a closed-form solution) and the closed-form formula for the geometric Asian option,
the CV technique lowers the variance dramatically.




Although mostly used in connection with Monte Carlo simulation, the CV technique can in fact be used to enhance almost any numerical method.
This paper presents a general framework of the CV technique.
It lays out the conditions a CV and the numerical method should satisfy for the CV technique to work.
As illustrations, we demonstrate how to apply this general version of the CV technique to Fourier transform (FT) and convolution-based pricing algorithms, and binomial trees.



FT-based pricing algorithms and convolution-based algorithms are widely used in the literature (e.g., \citet{benhamou2002fast}, \citet{carr1999option}, \citet{lord2008fast}, \citet{schmelzle2010option}).
An FT-based pricing algorithm can deal with European options pricing under any model with a closed-form characteristic function,
which includes some stochastic-volatility models and all exponential L\'evy models (see \citet{tankov2003financial}).
Convolution-based pricing algorithms can be used to evaluate many path dependent options.
Our methodology is able to enhance all those pricing algorithms in terms of accuracy and efficiency.




The CV technique is applied to binomial trees by \citet{hull1988use}.
\citet{tankov2003financial} also derives an FT-based pricing algorithm that implicitly uses a control variate,
which is mentioned in a survey by \citet{schmelzle2010option}.
Apart from the works, the CV technique has always been combined with Monte Carlo.
This paper will substantially broaden the applicability of the CV technique far beyond Monte Carlo.
We will apply it to FT and convolution-based pricing algorithms.
We also discuss choices of CV.
In the literature, the CV technique is applied to the algorithm as a whole.
This is true even for binomial trees in \citet{hull1988use}.
In the case of convolution-based algorithms and binomial trees, the CV technique will be applied to steps within the algorithm.
In fact, different CVs are picked in different steps.
Even if an algorithm as a whole does not satisfy the conditions under which it can benefit from the use of CVs, some of its individual parts may.




This paper is organized as follows.
Section 2 explains the idea of the CV technique and the conditions under which it can be used.
Section 3 discusses how to apply the CV technique to FT-based pricing algorithms, as well as a methodology to find a CV.
Section 4 uses the convolution-based pricing algorithms to illustrate how to apply the CV technique within them.
Section 5 demonstrates how to apply the CV technique to binomial trees.
Numerical results are given in the end of each section to show the improvements over original algorithms.
Section 6 concludes.





\section{Preliminary}
The purpose of this thesis is to propose efficient and accurate numerical algorithms for financial derivatives pricing.
A financial derivative is a financial contract whose value is derived by one or several underlying securities.
Its fair value, or the no-arbitrage price, can be derived as an expectation under the risk neutral probability measure according to the martingale pricing theory.
Since the entire thesis is based on this theory,
in this section we review it and other basic elements in derivatives pricing.



\subsection{Martingale Pricing Theory}
\label{sec:martingale}

The goal here is to sketch the derivation of the fair value of a financial derivative under certain assumptions.
We start from a simplified version of the martingale pricing theory.
A more general and detailed discussion can be found in Chapter 5 in \citet{shreve2004stochastic}.



Consider a derivative with maturity $T$ and payoff $X$.
Denote by $S_t$ the underlying security price at time $t$.
An example is a call option written on a stock that gives its holder the right, but not the obligation, to buy the stock at a predetermined strike price at time $T$.
Assuming that the strike price is $K$, and that the underlying stock price at time $t$ is $S_t$, then we have the payoff $X = \max(S_T - K, 0)$.
The operation $\max(\cdot, 0)$ is so common in the field of financial engineering that we give it a somewhat more simplified notation $(\cdot)^+$ and write the payoff as $X = (S_T - K)^+$.
As illustrated in this example, at time zero the payoff $X$ can be a random variable.
In fact, in most cases it is a random variable.



To obtain the fair value of the derivative, first we construct a portfolio with money market account and the underlying asset.
We change the portfolio constantly according to certain strategy so that the value of this portfolio at time $T$ is equal to the payoff $X$ of the derivative in all scenarios.
In addition, once this portfolio is set up, no money can be put into or pull out of the portfolio.
When an agent changes the portfolio, he or she can only sell some asset from the portfolio to buy another.
A strategy with all these properties is called a self-financing strategy that replicates $X$.



To find such a strategy, it is convenient to have more assumptions, including the dynamics of the underlying asset $S_t$.
The following assumptions are the starting point of the celebrated Black-Scholes model.
Assume a constant risk free interest rate $r$.
For constant $\mu$ and $\sigma>0$, assume that the underlying asset price process follows the stochastic differential equation (SDE)
\begin{eqnarray*}
dS_t = \mu S_t dt + \sigma S_t dW_t,
\end{eqnarray*}
where $W_t$ is a Brownian motion in the real world (filtered) probability space $(\Omega, \mathcal F, (\mathcal F_t), P)$.
The solution of this SDE is known as the geometric Brownian motion because it can be shown that $S_t = S_0 \exp((\mu-\sigma^2/2)t + \sigma W_t)$,
which is a Brownian with drift taking exponential.
The constant $\mu$ and $\sigma$ are used to model the mean rate of return and the volatility of the underlying asset, respectively.
Denote $B_t =  e^{rt}$ the value of one unit of money market account, and denote by $Z_t$ the discounted underlying asset $B_t^{-1} S_t$.
By the Ito lemma, we can evaluate the differential of $Z_t$ as
\begin{eqnarray*}
d(B_t^{-1} S_t) &=& -r B_t^{-1} S_t dt + B_t^{-1} dS_t - r B_t^{-1} dt d S_t \\
&=& -r B_t^{-1}S_t dt + B_t^{-1}(\mu S_t dt + \sigma S_t dW_t)\\
%&=& B_t^{-1}S_t((\mu - r) dt + \sigma dW_t)\\
&=& \sigma B_t^{-1}S_t\left(\left(\frac{\mu - r}{\sigma}\right) dt + dW_t\right)\\
&=& \sigma Z_t\left(\xi dt + dW_t\right),
\end{eqnarray*}
where we denote the ratio $(\mu - r)/\sigma$ by $\xi$.
It is called the market price of risk because it is the excess return per unit of volatility.
Let $Y_t = \exp({-\frac{\xi^2}{2} t + \xi W_t})$.
It can be shown that $E[Y_T] = 1$ and hence we can define a new probability measure
$$
\tilde P(E) = \int_E Y_T\,dP,
$$
under which, by the Girsanov theorem, the stochastic process $\tilde W_t = \xi t + W_t$ is a Brownian motion.
This new measure $\tilde P$ is called the risk neutral probability measure.



We can write $dZ_t = \sigma Z_t d\tilde W_t$ and it becomes clear that $Z_t$ is a martingale under $\tilde P$ since it is an Ito integral.
We further define $E_t = \tilde E[B_T^{-1}X|\mathcal F_t]$, where $\tilde E$ stands for the expectation under $\tilde P$.
$E_t$ is also a martingale because for $0<s<t$ we have
$\tilde E[E_t | \mathcal F_s] = \tilde E[\tilde E[B_T^{-1}X|\mathcal F_t] | \mathcal F_s] = \tilde E[B_T^{-1}X|\mathcal F_s] = E_s$.
This is a direct consequence of the tower property.



Now we have two martingales $Z_t$ and $E_t$.
By the martingale representation theorem, there exists an adapted process $\phi_t$ such that $dE_t = \phi_tdZ_t$.
Define $\psi_t = E_t - \phi_t Z_t$.
Then $V_t = \phi_t S_t + \psi_t B_t$ is the self-financing strategy that replicates $X$.
Before we explain why it is self-financing and why it replicates $X$, let us take a look at the strategy.
The portfolio consists of $\phi_t$ shares of the underlying asset and $\psi_t$ units of money market account.
Since $S_t$ and $B_t$ are the value of the underlying asset and the money market account, respectively,
the time $t$ value of the portfolio is $V_t = \phi_t S_t + \psi_t B_t$.
The value of $\phi_t$ moves as $S_t$, and $\psi_t$ is determined by $\phi_t$,
so we change the portfolio according to the change of the value of the underlying asset.



To see that $V_t$ is a self-financing strategy that replicates $X$, we first need the identity $V_t = B_t E_t$, which is derived as follows
\begin{eqnarray*}
V_t &=& \phi_t S_t + \psi_t B_t\\
&=& \phi_t S_t + (E_t - \phi_t Z_t)B_t\\
&=& \phi_t S_t + B_t E_t - \phi_t B_t Z_t = B_t E_t.
\end{eqnarray*}
The last equality holds because $B_t Z_t = S_t$.
To say that $V_t$ replicates $X$, we need to prove $V_T = X$, which is obvious given the above identity.
Indeed,
\begin{eqnarray*}
V_T &=& B_T E_T \\
&=& B_T E_{\mathbb Q}[B_T^{-1}X|\mathcal F_T]\\
&=& B_T B_T^{-1}X = X.
\end{eqnarray*}
To verify that this strategy is self-financing, we first introduce the mathematical definition:
$V_t$ is self-financing if and only if
$$
dV_t = \phi_t dS_t + \psi_t dB_t.
$$
Intuitively, this identity says that all the change of the value of $V_t$ is from the change of $S_t$ and $B_t$, but not $\phi_t$ and $\psi_t$.
In other words, although we can change our portfolio anytime, it has to be selling one asset to buy another.
No change of $\phi_t$ or $\psi_t$ can effect the value of $V_t$.
Initially we set up the portfolio: $\phi_0$ shares of $S$ and $\psi_0$ shares of $B$.
Once it is done, no money can be put in or pulled out.
Now we are ready to verify that $V_t$ is indeed self-financing.
Note that
\begin{eqnarray*}
dV_t &=& B_t dE_t + E_t dB_t\\
&=& B_t \phi_t dZ_t + (\psi_t + \phi_t Z_t)dB_t\\
&=& \psi_t dB_t + \phi_t (  B_t dZ_t + Z_t dB_t )
\end{eqnarray*}
Since $B_t dZ_t + Z_t dB_t = d(B_tZ_t)$, the right hand side is equal to $\psi_t dB_t + \phi_t dS_t$ and the definition of self-financing is thus satisfied.



Having $V_t$ constructed means that
we have a way to obtain the same payoff $X$ as the derivative at maturity $T$ just by buying and selling the underlying asset and the money market account.
No need to actually buy the derivative itself.
Since $V_t$ is self-financing, by the no-arbitrage principle, at all time it must have the same value as the derivative.
Thus we conclude that the fair value, or the no-arbitrage price, of this derivative is
\begin{eqnarray*}
V_t &=& B_t E_t\\
&=& B_t \tilde E[B_T^{-1}X|\mathcal F_t]\\
&=& e^{-r(T-t)}\tilde E[X|\mathcal F_t].
\end{eqnarray*}
In particular, at $t=0$ we have
\begin{eqnarray}
V_0 = e^{-rT} \tilde E[X],\label{eq:universal_pricing_formula}
\end{eqnarray}
meaning the fair initial value of a derivative is the discounted expected value under the risk neutral probability.



The pricing formula $(\ref{eq:universal_pricing_formula})$ is quite general.
Since we do not specify what $X$ is, it can be the payoff of any derivatives, even path-dependent ones.
Path-dependent derivatives are derivatives whose payoff is determined not only by the value of the underlying asset at maturity,
but also the trajectory of the underlying asset value throughout the life of the derivative.
Examples include Asian options and barrier options, which will be discussed in later sections.




The formula $(\ref{eq:universal_pricing_formula})$ is still valid even if the Black-Scholes assumptions are changed.
One only needs to modify the above argument accordingly.
A well-known example is the stochastic volatility model proposed by \citet{heston1993closed},
in which to find the risk neutral probability measure we need the two dimensional Girsanov theorem.
There are also exponential Levy models such as the ones proposed in \citet{merton1976option} and \citet{kou2002jump}
where both the Girsanov theorem and the martingale representation theorem needs to be extended to handle the jump processes,
but in the end, the pricing formula (\ref{eq:universal_pricing_formula}) can also be obtained.
An excellent introduction of derivative pricing models built with jump processes can be found in \citet{tankov2003financial}.



\section{General Control Variate Method}





The control variate technique is an effective variance reduction technique in the Monte Carlo method
which can actually be extended to many other numerical pricing methods.
To better explain the general version, we first review the old version in the Monte Carlo method.
Denote by $\mathsf E[Y]$ the expectation of a random variable $Y$,
and $\mathbb E[Y]$ the Monte Carlo estimator $\sum_{i=1}^nY(\omega_i)/n$, where $\omega_i$ are random samples.
The plain Monte Carlo method uses $\mathbb E[Y]$ as an estimate of $\mathsf E[Y]$.
The error of this estimate depends on the variance of $\mathbb E[Y]$.
The smaller the variance is, the more likely the estimate is accurate.



The control variate technique is based on the decomposition $\mathsf E[Y] = \mathsf E[Y-C] + \mathsf E[C]$.
The idea is that if one can find a random variable $C$ such that $\mathsf E[C]$ is known analytically
and the variance of $Y-C$ is smaller than the the variance of $Y$,
then one can use the estimate
\begin{eqnarray*}
\mathsf E[Y] &\approx& \mathbb E[Y-C] + \mathsf E[C] \\
            &=& \left(\sum_{i=1}^n(Y(\omega_i) - C(\omega_i))/n\right) + \mathsf E[C],
\end{eqnarray*}
which has a smaller variance compared to $\mathbb E[Y]$ and hence is a better estimate.
The random variable $C$ is called the control variate, or control for short, and $Y-C$ is called the difference.
In other words, in the control variate technique $Y$ is written as the sum of the difference and the control and,
since $\mathsf E[C]$ is known, the Monte Carlo estimator is only applied to the difference.



To extend the above idea to other numerical methods, we move to a general setting.
Suppose we are to evaluate $\mathsf T(f)$, a linear operator $\mathsf T$ acting on the function $f$, and we have a numerical approximation $\mathbb T(f)$.
An example is that $\mathsf T$ is the integral operator $\int_a^b$, so $\mathsf T(f) = \int_a^b f\,dx$, while $\mathbb T$ is the trapezoidal rule approximation of $T$.
An analogous idea to the control variate technique is to find a function $g$, the control, such that $\mathsf T(g)$ is known analytically,
and the error of $\mathbb T(f-g)$ is smaller than the error of $\mathbb T(f)$.
Then
\begin{eqnarray}
\mathsf T(f) \approx \mathbb T(f-g) + \mathsf T(g)\label{eq:GCVapprox}
\end{eqnarray}
would be a better approximation than $\mathbb T(f)$.
This statement is quite general.
The only assumption is the linearity of $\mathsf T$.
See Figure \ref{fig:CV_GCV_comp} for a comparison of these two versions of control variate technique.



\begin{figure}[t!]
\rule{\textwidth}{0.4pt}\\
{\scriptsize
\parbox[t]{0.43\textwidth}{
Control Variate Method in Monte Carlo
\begin{itemize}
    \item Want to evaluate $\mathsf E[Y]$
        \begin{itemize}
            \item[]
        \end{itemize}
    \item Monte Carlo estimator
        \begin{eqnarray*}
        \textstyle \mathsf E[Y] \approx \mathbb E[Y] = \sum_{i=1}^n Y(\omega_i)/n
        \end{eqnarray*}
    \item Control variate method
        \begin{eqnarray*}
        \mathsf E[Y] = \mathsf E[Y-C] + \mathsf E[C]
        \end{eqnarray*}
    \begin{itemize}
        \item $C$ is the control variate
        \item $Y-C$ is the difference
    \end{itemize}
    \item Find a $C$ such that
    \begin{itemize}
        \item $\text{Var}(Y-C) < \text{Var}(Y)$
        \item $\mathsf E[C]$ is known analytically
    \end{itemize}
    \item Apply the estimator on $Y-C$ instead of $Y$
\end{itemize}
}
\parbox[t]{0.57\textwidth}{
General Control Variate Method
\begin{itemize}
    \item Want to evaluate $\mathsf T(f)$
    \begin{itemize}
        \item linear operator $\mathsf T$ on any function $f$
    \end{itemize}
    \item Approximating operator
        \begin{eqnarray*}
        \mathsf T(f) \approx \mathbb T(f)
        \end{eqnarray*}
    \item General control variate method
        \begin{eqnarray*}
        \mathsf T(f) = \mathsf T(f-g) + \mathsf T(g)
        \end{eqnarray*}
    \begin{itemize}
        \item $g$ is the control variate
        \item $f-g$ is the difference
    \end{itemize}
    \item Find a $g$ such that
    \begin{itemize}
        \item error of $\mathbb T(f)$ $<$ error of $\mathbb T(f-g)$
        \item $\mathsf T(g)$ is known analytically
    \end{itemize}
    \item Apply the approximating operator $\mathbb T$ on $f-g$ instead of $f$
\end{itemize}
}}
\rule{\textwidth}{0.4pt}
\caption{Control Variate Method in Monte Carlo and the General Version}\label{fig:CV_GCV_comp}
\end{figure}








\subsection{An Example}
\label{sec:GCV_ncdf}
Before going to option pricing, let us use a simple example to demonstrate how the control variate method works with numerical integration.
Suppose we want to apply the trapezoidal rule to evaluate the cumulative normal distribution function $N(x) = \int_{-\infty}^x n(t)\,dt = 1/2 + \int_{0}^x n(t)\,dt, $
where $n(t) = \exp(-t^2/2)/\sqrt{2\pi}$ is the standard normal density function.
A common approximation for $N(x)$ is
$$
N(x)\approx 1-n(x)\left(a_1 p + a_2 p^2 + a_3 p^3 + a_4 p^4 + a_5 p^5\right) \qquad \forall x\geq 0,
$$
where $p = 1/(1+a_0 x)$ and all $a_i$'s are given constants.\footnote{
$a_0=0.2316419$, $a_1=0.319381530$, $a_2=-0.356563782$, $a_3=1.781477937$, $a_4=-1.821255978$, $a_5=1.330274429.$
This approximation is from \citet{abramowitz1972handbook}.
}
This function, denoted by $F(x)$ hereafter, can lead us to a good control variate.


Since $F(x)$ behaves like $N(x)$, $dF(x)/dx$ should be similar to $n(x)$.
We define $n_{\text{CV}}(x) = dF(x)/dx$ and use it as the control variate. Denote by $n_{\text{Diff}}$ the difference $n-n_{\text{CV}}$ and rewrite
%{\color{red} eq:CVdecomp\_n}
\begin{eqnarray}
\int_{0}^x n\,dt &=& \int_{0}^x n-n_{\text{CV}}\,dt + \int_{0}^x n_{\text{CV}}\,dt\nonumber\\
&=& \int_{0}^x n_{\text{Diff}}\,dt + \int_{0}^x n_{\text{CV}}\,dt.\label{eq:CVdecomp_n}
\end{eqnarray}
Now the second term of the right hand side is known analytically as $F(x)-F(0)$, so we apply the trapezoidal rule only on the first term.
This way, since $n_{\text{Diff}}$ only takes small values, we will get a better approximation than na\"ively applying numerical integration to evaluate the left hand side.
The plots of $n(t)$ and $n_{\text{Diff}}(t)$ shown in Figure \ref{fig:n and n_diff} give an explanation why the numerical error is reduced.


\begin{figure}[!t]
\begin{center}
\parbox[t]{0.45\textwidth}{
$n(t)$\\\\
  \centerline{\epsfig{figure=eps/n.eps, height=2in, width=0.5\textwidth}}
}\hfill\quad
\parbox[t]{0.45\textwidth}{
$n_{\text{Diff}}(t)$\\\\
  \centerline{\epsfig{figure=eps/n_diff.eps, height=2in, width=0.5\textwidth}}
}
\end{center}
\caption{
Plots of $n(t)$ and $n_{\text{Diff}}(t)$.
Through Eq.\ (\ref{eq:CVdecomp_n}), we can apply numerical integration on $n_{\text{Diff}}(x)$ instead of $n(t)$.
Since $n_{\text{Diff}}(t)$ takes small values, the error induced by numerical integration is small.
}
\label{fig:n and n_diff}
\end{figure}



\section{Fourier Transform Based Pricing Algorithms}
% Option pricing algorithms based on the Fourier transform are widely used in financial engineering.
% An important reason is that many option pricing models do not have a closed form cumulative distribution function,
% but their characteristic function is known analytically.
% An FT-based pricing algorithm first derives the Fourier transform of the option price or tail probabilities in terms of the characteristic function,
% and then evaluates the inverse Fourier transform numerically.
% The resulting algorithms are just numerical integrations applied to some function that is known analytically, so the control variate method can be easily applied.



The Black-Scholes model has many unrealistic assumptions such as iid log-return, constant volatility, and continuous paths.
Over decades more realistic derivatives pricing models have been developed, including the stochastic volatility model proposed by \citet{heston1993closed},
and models based on discontinuous L\'evy processes such as \citet{merton1976option}, \citet{kou2002jump} and \citet{carr2002fine}.
In many of these models, a closed form pricing formula is not possible to obtain.
Only the characteristic function of the log-return is known in closed form.
Fourier transform (FT) based pricing algorithms are therefore developed to fill the need of efficient derivatives pricing.



In FT based pricing algorithms, the Fourier transform of the derivative price is first derived.
Mostly it is in terms of the characteristic function of the log-return of the underlying asset and is known in closed form.
By the Fourier inversion theorem, taking the inverse Fourier transform would give us the derivative price.
Since there is no closed form formula for the derivative prices, the inverse Fourier transform is done numerically, which is just a numerical integration.
As previously discussed, the accuracy of a numerical integration can be improved by the control variate technique.



The control variate technique applied to FT-based pricing algorithms has been studied by, for example, \citet{tankov2003financial} and \citet{schmelzle2010option},
but has otherwise not received much attention in the literature.
Here we give a more general argument, including a comprehensive discussion on how to choose the control variate.
In this section we derive two FT based pricing algorithms --
the Black-Scholes style formula used in \citet{heston1993closed}, and the algorithm by \citet{carr1999option}.
We then show how to modify these algorithms with the control variate technique.
Extension to other types of FT based pricing algorithm is straightforward.




\subsection{The Black-Scholes Style Formula}


The Black-Scholes style pricing formula for vanilla call options is first derived by \citet{heston1993closed}.
The only assumption needed to derive this formula is that the risk-free interest rate is constant.
The pricing formula is convenient for any model for the underlying asset where the characteristic function of the log return is known analytically,
a constraint satisfied by many common option pricing models, including the Heston model, the Black-Scholes model, and all exponential L\'evy models.



Let $r$ be the constant risk-free interest rate, and $S_t$ the dynamics of the price of the underlying asset of the option to be priced.
Let $\phi_T(u) = E[e^{iu\log(S_T/S_0)}]$ be the characteristic function of the log-return $\log\left(S_T/S_0\right)$.
The Black-Scholes style pricing formula for vanilla call option with strike $K$ and maturity $T$ is
\begin{eqnarray*}
c = S_0 \Pi_1 - K e^{-rT} \Pi_2,\label{eq:BSstyle}
\end{eqnarray*}
where
\begin{eqnarray*}
\Pi_1 &=& \frac{1}{2} + \frac{1}{\pi}\int_0^{\infty} \text{Re}\left[\frac{e^{-iu \log\left(\frac{K}{S_0}\right)}\phi_T(u-i)}{iu\phi_T(-i)}\right]\,du,\\
\Pi_2 &=& \frac{1}{2} + \frac{1}{\pi}\int_0^{\infty} \text{Re}\left[\frac{e^{-iu \log\left(\frac{K}{S_0}\right)}\phi_T(u)}{iu}\right]\,du,
\end{eqnarray*}
Given this formula, evaluating the option price with numerical integration rules is straightforward.




To derive the formula, we start from the universal pricing formula (\ref{eq:universal_pricing_formula}).
Since the payoff of a vanilla call option is $(S_T-K)^+$, we can write $c = e^{-rT}\tilde E[(S_T - K)^+]$.
Note that
\begin{eqnarray*}
c &=& e^{-rT}\tilde E[(S_T - K)^+]\\
&=& e^{-rT}\tilde E\left[(S_T - K)1_{\{S_T>K\}}\right]\\
&=& e^{-rT}\tilde E\left[S_T 1_{\{S_T>K\}}\right] - K e^{-rT} E\left[1_{\{S_T>K\}}\right]\\
&=& S_0\tilde E\left[e^{-rT}\left(\frac{S_T}{S_0}\right) 1_{\{S_T>K\}}\right] - K e^{-rT} P( S_T>K ).
\end{eqnarray*}
In terms of the log-return, we rewrite
\begin{eqnarray}
c = S_0\underbrace{\tilde E\left[e^{-rT}\left(\frac{S_T}{S_0}\right) 1_{\left\{\log\frac{S_T}{S_0}>\log\frac{K}{S_0}\right\}}\right]}_{\Pi_1} - K e^{-rT} \underbrace{\tilde P\left( \log\frac{S_T}{S_0}>\log\frac{K}{S_0} \right)}_{\Pi_2}.\label{eq:S0E-KDP}
\end{eqnarray}
The majority of the derivation of the pricing formula is the following two lemmas:
\begin{lemma}
Let $\phi(u) = \tilde E[e^{iuX}]$ be the characteristic function of a random variable $X$.
Then
$$
\tilde P(X>H) = \frac{1}{2} + \frac{1}{\pi}\int_0^{\infty} \text{Re}\left[ \frac{e^{-iuH}\phi(u)}{iu} \right]\,du.
$$
\label{lemma:prob}
\end{lemma}
\begin{lemma}
With the same notation as Lemma \ref{lemma:prob}, we have
$$
\frac{\tilde E\left[e^X 1_{\{X>H\}}\right]}{\tilde E[e^X]} = \frac{1}{2} + \frac{1}{\pi}\int_0^{\infty} \text{Re}\left[ \frac{e^{-iuH}\phi(u-i)}{iu\phi(-i)} \right]\,du.
$$
\label{lemma:Delta}
\end{lemma}
Once these lemmas are proved, the only step it takes to derive an explicit formula for (\ref{eq:BSstyle})
is to apply Lemmas \ref{lemma:Delta} and \ref{lemma:prob} to rewrite the first and the second term, respectively.




\begin{proof} of Lemma \ref{lemma:prob}: Plugging in the definition of the characteristic function, the right hand side is
$$
\frac{1}{2} + \frac{1}{\pi}\int_0^{\infty} \text{Re}\left[ \frac{e^{-iuH}E[e^{iuX}]}{iu} \right]\,du,
$$
where the integrand can be rewritten as
\begin{eqnarray*}
\text{Re}\left[ \tilde E\left[\frac{e^{iu(X-H)}}{iu}\right] \right] &=& \tilde E\left[ \text{Re}\left[\frac{e^{iu(X-H)}}{iu}\right] \right]\\
&=& \tilde E\left[ \text{Re}\left[\frac{\cos (u(X-H)) + i\sin (u(X-H))}{iu}\right] \right]\\
&=& \tilde E \left[\frac{\sin (u(X-H))}{u} \right].
\end{eqnarray*}
Note that the in the expectation we have a even function of $u$.
Now the right hand side of Lemma \ref{lemma:prob} can be written as
\begin{eqnarray}
\frac{1}{2} + \frac{1}{\pi}\int_0^{\infty} \tilde E \left[\frac{\sin (u(X-H))}{u} \right]\,du = \tilde E \left[\frac{1}{2} + \frac{1}{\pi}\int_0^{\infty} \frac{\sin (u(X-H))}{u} \,du\right], \label{eq:rhsFinalStep}
\end{eqnarray}
where Fubini's theorem is used to pass the integral inside the expectation.



Then we need the identity
$$
\int_0^{\infty} \frac{\sin t}{t}\,dt = \frac{\pi}{2},
$$
which is derived below.
Note that
\begin{eqnarray}
\int_{-\infty}^0 e^{xt}\,dx = \frac{1}{t}.\label{eq:int_sint/t}
\end{eqnarray}
Thus
\begin{eqnarray}
\int_0^{\infty} \frac{\sin t}{t}\,dt &=& \int_0^{\infty}\int^0_{-\infty} e^{xt} \sin t\,dxdt\nonumber\\
&=& \int^0_{-\infty}\int_0^{\infty} e^{xt} \sin t\,dtdx.\label{eq:doubleInt}
\end{eqnarray}
The domain of integration is set up in a way that $xt$ is negative, in order to make sure of the convergence of the integral.
Integration by part yields
$$
\int e^{xt} \sin t\,dt = \frac{1}{x} e^{xt} \sin t - \frac{1}{x^2} e^{xt} \cos t - \int \frac{1}{x^2}e^{xt} \sin t\,dt.
$$
Due to the fact that $xt<0$, we have
\begin{eqnarray*}
\int_0^{\infty} e^{xt} \sin t\,dt &=& \left[\frac{ \frac{1}{x} e^{xt} \sin t - \frac{1}{x^2} e^{xt} \cos t }{1+\frac{1}{x^2}}\right]_{t=0}^{t=\infty} \\
&=& \left[e^{xt}~\frac{ x \sin t - \cos t }{1 + x^2}\right]_{t=0}^{t=\infty}\\
&=& \frac{1}{1+x^2}.
\end{eqnarray*}
Plug this into (\ref{eq:doubleInt}) to get
\begin{eqnarray*}
\int_0^{\infty} \frac{\sin t}{t}\,dt = \int^0_{-\infty} \frac{1}{1+x^2}\,dx = \frac{\pi}{2}.
\end{eqnarray*}





With the identity (\ref{eq:int_sint/t}) established, we can further simplify (\ref{eq:rhsFinalStep}).
Substituting $t = u(X-H)$ in (\ref{eq:int_sint/t}), we get
\begin{eqnarray*}
\int_0^{\infty} \frac{\sin (u(X-H))}{u} \,du =
\begin{cases}
\int_0^{\infty} \frac{\sin t}{t}\,dt = \frac{\pi}{2}&\mbox{if }X>H\\
0&\mbox{if }X=H\\
\int_0^{-\infty} \frac{\sin t}{t}\,dt = -\frac{\pi}{2} &\mbox{if }X<H
\end{cases}.
\end{eqnarray*}
Plug this into (\ref{eq:rhsFinalStep}) to get $E\left[1_{\left\{X>H\right\}}\right] = P(X>H)$.\hfill\qed
\end{proof}
\begin{proof} of Lemma \ref{lemma:Delta}: Plugging in the definition of the characteristic function, the right hand side is
\begin{eqnarray*}
&&\frac{1}{2} + \frac{1}{\pi}\int_0^{\infty} \text{Re}\left[ \frac{e^{-iuH}E[e^{i(u-i)X}]}{iuE[e^{X}]} \right]\,du \\
&=& \frac{1}{2} + \frac{1}{\pi}\int_0^{\infty} \frac{1}{E[e^X]}\text{Re}\left[ \frac{e^{-iuH}E[e^{iuX+X}]}{iu} \right]\,du \\
&=& \frac{1}{2} + \frac{1}{\pi}\int_0^{\infty} \frac{1}{E[e^X]}\text{Re}\left[ E\left[\frac{e^{iu(X-H)+X}}{iu}\right] \right]\,du \\
&=& \frac{1}{2} + \frac{1}{\pi}\int_0^{\infty} \frac{1}{E[e^X]}E\left[\text{Re}\left[ \frac{e^{iu(X-H)+X}}{iu}\right] \right]\,du.
\end{eqnarray*}
Now we apply Euler's formula $e^{ix} = \cos x + i\sin x$ to get rid of the $\text{Re}(\cdot)$ function, like we did in the proof of Lemma 4.1.
Thus we can write
\begin{eqnarray*}
&&\frac{1}{2} + \frac{1}{\pi}\int_0^{\infty} \frac{1}{E[e^X]}E\left[\text{Re}\left[ \frac{e^{iu(X-H)+X}}{iu}\right] \right]\,du\\
&=& \frac{1}{2} + \frac{1}{\pi}\int_0^{\infty} \frac{1}{E[e^X]}E\left[e^X\text{Re}\left[ \frac{\cos(u(X-H)) + i\sin(u(X-H))}{iu}\right] \right]\,du \\
&=& \frac{1}{2} + \frac{1}{\pi}\int_0^{\infty} \frac{1}{E[e^X]}E\left[e^X\frac{\sin(u(X-H))}{u} \right]\,du \\
&=& \frac{1}{E[e^X]}E\left[ e^X \left(\frac{1}{2} + \frac{1}{\pi}\int_0^{\infty}\frac{\sin(u(X-H))}{u}\,du \right)\right] = \frac{E\left[e^X 1_{\{X>H\}}\right]}{E[e^X]},
\end{eqnarray*}
which completes the proof. \hfill\qed
\end{proof}



Vanilla options pricing under the Black-Scholes model does not really require the Black-Scholes style formula,
because we have the closed form Black-Scholes pricing formula
\begin{eqnarray*}
c &=& S_0 N(d_1) - K e^{-rT} N(d_2), \\
d_1 &=& \frac{\log \frac{S_0}{K} + \left(r + \frac{\sigma^2}{2}\right)T }{\sigma\sqrt T}, \\
d_2 &=& \frac{\log \frac{S_0}{K} + \left(r - \frac{\sigma^2}{2}\right)T }{\sigma\sqrt T} = d_1 - \sigma\sqrt T,
\end{eqnarray*}
where $N(\cdot)$ is the cumulative distribution function of the standard normal distribution.
In fact, the similarity between these two formulas is the reason why the Black-Scholes style formula got its name in the first place.
Below we derive the Black-Scholes formula from the Black-Scholes style formula.




In the Black-Scholes model, the log return of the underlying asset under the risk-neutral probability measure is assumed to follow a Brownian motion with drift:
$$
\log\frac{S_T}{S_0} = \left(r-\frac{\sigma^2}{2}\right)T + \sigma W_T \sim N\left(\left(r-\frac{\sigma^2}{2}\right)T, \sigma^2 T\right).
$$
Thus its characteristic function is known in closed form as
$$
\phi_T(u) = e^{i \left(r-\frac{\sigma^2}{2}\right)T u - \frac{\sigma^2 T}{2}u^2}.
$$
With the characteristic function, the $\Pi_2$ in the Black-Scholes style formula can be written as
\begin{eqnarray*}
&& \frac{1}{2} + \frac{1}{\pi}\int_0^{\infty} \text{Re}\left[\frac{e^{-iu \log\left(\frac{K}{S_0}\right)}\phi_T(u)}{iu}\right]\,du\\
&=& \frac{1}{2} + \frac{1}{\pi}\int_0^{\infty} \text{Re}\left[\frac{e^{iu \left(\log\frac{S_0}{K} + \left(r-\frac{\sigma^2}{2}\right)T \right) - \frac{\sigma^2 T}{2}u^2}}{iu}\right]\,du.
\end{eqnarray*}
Again we apply Euler's formula $e^{ix} = \cos x + i\sin x$ to get rid of the $\text{Re}(\cdot)$ function to get
\begin{eqnarray*}
&&\frac{1}{2} + \frac{1}{\pi}\int_0^{\infty} \text{Re}\left[\frac{e^{iu \left(\log\frac{S_0}{K} + \left(r-\frac{\sigma^2}{2}\right)T \right) - \frac{\sigma^2 T}{2}u^2}}{iu}\right]\,du\\
&=& \frac{1}{2} + \frac{1}{\pi}\int_0^{\infty} \text{Re}\left[\frac{\cos\left(u \left(\log\frac{S_0}{K} + \left(r-\frac{\sigma^2}{2}\right)T \right) \right) + i\sin\left(u \left(\log\frac{S_0}{K} + \left(r-\frac{\sigma^2}{2}\right)T \right) \right)}{iu}\right]e^{- \frac{\sigma^2 T}{2}u^2}\,du\\
&=& \frac{1}{2} + \frac{1}{\pi}\int_0^{\infty} \frac{\sin\left(u \left(\log\frac{S_0}{K} + \left(r-\frac{\sigma^2}{2}\right)T \right) \right)}{u} e^{- \frac{\sigma^2 T}{2}u^2} \,du.
\end{eqnarray*}
Set
$$
d_2 = \frac{\log \frac{S_0}{K} + \left(r-\frac{\sigma^2}{2}\right)T}{\sigma\sqrt T}
$$
for simplicity and apply the substitution $t = \sigma\sqrt T u$ to simplify the last integral above as
\begin{eqnarray}
&&\frac{1}{2} + \frac{1}{\pi}\int_0^{\infty} \frac{\sin\left(u \left(\log\frac{S_0}{K} + \left(r-\frac{\sigma^2}{2}\right)T \right) \right)}{u} e^{- \frac{\sigma^2 T}{2}u^2} \,du\nonumber\\
&=& \frac{1}{2} + \frac{1}{\pi}\int_0^{\infty} \frac{\sin\left(d_2 t \right)}{t} e^{- \frac{t^2}{2}} \,dt.\label{eq:Pi_2}
\end{eqnarray}
Note that since $d_2$ is just a constant parameter, the integrand is a well-behaved smooth function of $t$.
This is critical when the Black-Scholes style formula is used with numerical integration.
To further simplify the last expression in (\ref{eq:Pi_2}) we expand the integrand and integrate term by term.
This technique can be justified by the dominated convergence theorem.
In the end, we obtain a series that matches the series of $N(x)$.
Thus $\Pi_2$ can be simplified as $N(d_2)$.



To find $\Pi_1$, we could use Lemma \ref{lemma:Delta}, but instead let us apply the Girsanov Theorem and rewrite
\begin{eqnarray}
\Pi_1 &=& \tilde E\left[e^{-rT}\left(\frac{S_T}{S_0}\right) 1_{\left\{\log\frac{S_T}{S_0}>\log\frac{K}{S_0}\right\}}\right]\nonumber\\
&=& \bar E\left[ 1_{\left\{\log\frac{S_T}{S_0}>\log\frac{K}{S_0}\right\}}\right]\nonumber\\
&=& \bar P\left(\log\frac{S_T}{S_0}>\log\frac{K}{S_0}\right),\label{eq:Pi1NewMeasure}
\end{eqnarray}
where the expectation $\bar E$ is taken under another probability measure $\bar P$,
and the Radon-Nikodym derivative of $\bar P$ with respect to the risk-neutral probability measure is
\begin{eqnarray}
e^{-rT}\left(\frac{S_T}{S_0}\right) = e^{-\frac{\sigma^2}{2}T + \sigma \tilde W_T}.\label{eq:RNderivative}
\end{eqnarray}
This change of measure is equivalent to a change of numeraire in derivatives pricing.
Numeraire is a unit of measure used to evaluate security prices.
While any security can be the numeraire, the most commonly used numeraire is the money market account.
When the numeraire is changed, the risk-neutral probability measure should also be changed accordingly.
By switching the probability measure from $\tilde P$ to $\bar P$,
we are now using the underlying asset as the numeraire, and hence $\bar P$ is known as the stock measure.
So $\Pi_1$ in the Black-Scholes style formula is the probability that the call option will end up in the money, under the stock measure.
Here the purpose of changing the numeraire is simply to simplify the computation.
For more details about changing the numeraire in derivatives pricing, see for example \citet{shreve2004stochastic}.
With the Radon-Nikodym derivative (\ref{eq:RNderivative}), the Girsanov Theorem tells us that
$$
\bar W_t = \tilde W_t - \sigma t
$$
is a Brownian motion under the stock measure $\bar P$.
Thus we can rewrite the log return as
\begin{eqnarray*}
\log\frac{S_T}{S_0} &=& \left(r-\frac{\sigma^2}{2}\right)T + \sigma\left(\bar W_T +\sigma T\right)\\
 &=& \left(r + \frac{\sigma^2}{2}\right)T + \sigma \bar W_T,
\end{eqnarray*}
which has distribution $N((r + \sigma^2/2)T, \sigma^2 T)$ under $\bar P$.
Now apply Lemma \ref{lemma:prob} again on (\ref{eq:Pi1NewMeasure}) like how we got $\Pi_2$ above.
We obtain
\begin{eqnarray*}
\Pi_1 = \bar P\left(\log\frac{S_T}{S_0}>\log\frac{K}{S_0}\right) = \frac{1}{2} + \frac{1}{\pi}\int_0^{\infty} \frac{\sin\left( d_1 t\right)}{t} e^{- \frac{t^2}{2}} \,dt
\end{eqnarray*}
with
$$
d_1 = \frac{\log\frac{S_0}{K} + \left(r + \frac{\sigma^2}{2}\right)T}{\sigma\sqrt T}.
$$
The last integral is similar to (\ref{eq:Pi_2}) and can be simplified in the same way.
Finally we obtain $\Pi_1 = N(d_1)$.




\subsection{Carr and Madan's Algorithm}



Consider a vanilla call option with strike price $K$, maturity $T$.
Let the risk-free interest rate be a constant $r$.
Let the underlying asset price process be $S_t$ and $\phi_T(u)$ the characteristic function of the log-return.
Let $k = \log K$ and $b = 1.5$, $a(u) = e^{i(u - (b+1)i)\log S_0}$.
Then the time 0 value of the call option is
\begin{eqnarray}
c = \frac{e^{-bk}}{\pi}\int_0^\infty e^{-iku} \frac{~e^{-rT}a(u)\phi_T(u-(b+1)i)~}{b^2+b-u^2+i(2b+1)u}\,du.\label{eq:PCarr}
\end{eqnarray}
This is the very popular FT-based pricing algorithm developed by \citet{carr1999option}.
As our second example of the FT-based pricing algorithm, in this section we derive the algorithm.
Its modification by the control variate technique will be discussed in later sections.



A na\"ive idea to derive an FT-based pricing algorithm is to first derive the Fourier transform of the option price in closed form
and then take the inverse Fourier transform numerically.
But this will not work because the option price is not an integrable function and its Fourier transform is therefore not defined.
\citet{carr1999option} multiply the option price by a damping factor to get an integrable function, so that this idea can work.
Specifically, let $c(k)$ be the call option price in terms of the log-strike $k = \log K$.
As a function of $k$, this is not integrable because as $k\rightarrow -\infty$, we have $K \rightarrow 0^+$,
in which case the call option reduces to the right to get the stock for free at maturity,
and the fair price of this option at time 0 is naturally $S_0$.
Thus the graph of the function $c(k)$ has a nonzero horizontal asymptote $S_0$ on the right.
To resolve this problem, \citet{carr1999option} use a damping factor $e^{bk}$ and prove that the function $e^{bk}c(k)$ is integrable as long as $b>0$.
The derivation of Carr and Madan's algorithm is based on the Fourier transform of this function.
From the above argument we can also see that the constant $b$ in Carr and Madan's algorithm (\ref{eq:PCarr}) need not be 1.5.
Although Carr and Madan choose 1.5 in the original paper, any positive constant would suffice.



To derive Carr and Madan's algorithm (\ref{eq:PCarr}) we need the definition of the Fourier transform and the inverse Fourier transform.
For integrable functions $f(k)$ and $F(k)$, the Fourier transform is defined by
$$
\mathcal F(f(k))(v) = \int_{-\infty}^\infty e^{ivk}f(k)\,dk,
$$
and the inverse Fourier transform is defined by
$$
\mathcal F^{-1}(F(v))(k) = \frac{1}{2\pi} \int_{-\infty}^\infty e^{-ivk}F(v)\,dv.
$$
As long as $f$ is integrable, the Fourier inversion theorem guarantees that $\mathcal F^{-1}[\mathcal F[f]] = \mathcal F[\mathcal F^{-1}[f]] = f$,
that is the Fourier transform and the inverse Fourier transform can undo each other.



Since $e^{bk}c(k)$ is integrable and the Fourier transform is well-defined, we can write the option price as
\begin{eqnarray}
c(k) = e^{-bk} \mathcal F^{-1}[\mathcal F[e^{bk} c(k)]]. \label{eq:CarrInt}
\end{eqnarray}
As we will show later, $\mathcal F[e^{bk} c(k)]$ can be derived in closed form in terms of the characteristic function $\phi_T(u)$,
and the inverse transform $\mathcal F^{-1}$ can be written as an integral by definition.
So simplifying (\ref{eq:CarrInt}) would give us Carr and Madan's algorithm (\ref{eq:PCarr}).
To derive the explicit formula of $\mathcal F[e^{bk}c(k)]$, first rewrite the option price as follows
\begin{eqnarray*}
c(k) &=& e^{-rT}\tilde E[(S_T - K)^+]\\
 &=& e^{-rT}\tilde E[(e^{X_T + \log S_0} - e^k)^+] \\
&=& e^{-rT} \int_{k-\log S_0}^\infty (e^{s+\log S_0} - e^k)q_T(s)\,ds,
\end{eqnarray*}
where $q_T$ is the probability density function of the log return $\log(S_T/S_0)$.
Next plug into the definition of the Fourier transform and rewrite
\begin{eqnarray*}
\mathcal F(e^{bk}c(k))(u) &=& e^{-rT} \int_{-\infty}^\infty e^{iku}\int_{k-\log S_0}^\infty e^{bk}(e^{s+\log S_0} - e^k)q_T(s)\,ds\,dk.
\end{eqnarray*}
Move $e^{iku}$ into the second integral and apply Fubini's theorem to switch the order of the integration to get
\begin{eqnarray*}
\mathcal F(e^{bk}c(k))(u) &=& e^{-rT} \int_{-\infty}^\infty \int^{s+\log S_0}_{-\infty} e^{iku}e^{bk}(e^{s+\log S_0} - e^k)q_T(s)\,dk\,ds.
\end{eqnarray*}
The inner integral can be found and simplified as follows
\begin{eqnarray*}
\mathcal F(e^{bk}c(k))(u) &=& e^{-rT} \int_{-\infty}^\infty q_T(s) \left[\int^{s+\log S_0}_{-\infty} e^{(iu+b)k+s+\log S_0} - e^{(iu+b+1)k}\,dk\right]\,ds\\
&=& e^{-rT} \int_{-\infty}^\infty q_T(s) \left[ \frac{e^{(iu+b)k+s+\log S_0}}{iu+b} - \frac{e^{(iu+b+1)k}}{iu+b+1}\right]^{k=s+\log S_0}_{k=-\infty}\,ds\\
&=& e^{-rT} \int_{-\infty}^\infty q_T(s) e^{(iu+b+1)(s+\log S_0)} \left[ \frac{1}{iu+b} - \frac{1}{iu+b+1}\right]\,ds.
\end{eqnarray*}
Moving all factors without $s$ out of the integral, we obtain
\begin{eqnarray*}
%&=& \frac{e^{-rT}}{b^2+b-u^2+i(2b+1)u} \int_{-\infty}^\infty q_T(s) e^{(iu+b+1)(s+\log S_0)} \,ds\\
\mathcal F(e^{bk}c(k))(u) &=& \frac{e^{-rT}}{b^2+b-u^2+i(2b+1)u} \int_{-\infty}^\infty q_T(s) e^{i(u-(b+1)i)(s+\log S_0)} \,ds\\
&=& \frac{e^{-rT}e^{i(u-(b+1)i)\log S_0}}{b^2+b-u^2+i(2b+1)u} \int_{-\infty}^\infty q_T(s) e^{i(u-(b+1)i)s} \,ds
\end{eqnarray*}
For simplicity set $a(u) = e^{i(u-(b+1)i)\log S_0}$.
Since the characteristic function is the Fourier transform of the probability density function, the last expression gives
\begin{eqnarray*}
\mathcal F(e^{bk}c(k))(u) = \frac{~e^{-rT}a(u)\phi_T(u-(b+1)i)~}{b^2+b-u^2+i(2b+1)u}.
\end{eqnarray*}



Having derived the closed form formula of $\mathcal F[e^{bk} c(k)]$,
one can use the definition of the inverse Fourier transform to rewrite (\ref{eq:CarrInt}) as
\begin{eqnarray*}
c(k) &=& e^{-bk} \mathcal F^{-1}\left[\frac{~e^{-rT}a(u)\phi_T(u-(b+1)i)~}{b^2+b-u^2+i(2b+1)u}\right]\\
     &=& \frac{e^{-bk}}{2\pi}\int_{-\infty}^\infty e^{-iku} \frac{~e^{-rT}a(u)\phi_T(u-(b+1)i)~}{b^2+b-u^2+i(2b+1)u}\,du.
\end{eqnarray*}
As shown in \cite{carr1999option}, the above integrand is an even function of $u$, so we can write
$$
c = \frac{e^{-bk}}{\pi}\int_0^\infty e^{-iku} \frac{~e^{-rT}a(u)\phi_T(u-(b+1)i)~}{b^2+b-u^2+i(2b+1)u}\,du.
$$
This completes the derivation.



\subsection{Modified Algorithms}



Most of the FT-based pricing algorithms are just one dimensional integrals,
like the Black-Scholes style formula (\ref{eq:BSstyle}) and Carr and Madan's algorithm (\ref{eq:PCarr}) derived previously.
To implement the algorithms one can apply numerical integration such as the trapezoidal rule,
which can be enhanced by the control variate method as illustrated in Section \ref{sec:GCV_ncdf},
once a suitable control variate of the integrand is found.




To apply the control variate method on numerical integration, first we need to find a control variate of the integrand,
which is a function such that its integral is known analytically, and it behaves similarly to the integrand.
At first glance, with the integrands in (\ref{eq:PCarr}) and (\ref{eq:BSstyle}) it does not seem obvious how to find such functions,
but here we provide a general methodology to find control variates for all FT-based algorithms under all models.



Given an option pricing model, observe that in (\ref{eq:PCarr}) and (\ref{eq:BSstyle}) the integrals are known analytically if and only if we have a closed form option price.
The same is true for many other FT-based pricing algorithms.
Thus any model with a closed form option price can serve as the control model.
Among others we choose the model suggested by \citet{jarrow1982approximate}, which is based on the Edgeworth series expansion.
This model is easy to calibrate, which also helps to find a good control, as we will explain later.



To derive the control variate of the integrand from the Edgeworth series, first denote by $X$ the random variable $\log(S_T/S_0)$.
Recall that the characteristic function of $X$, denoted by $\phi_T(u)$, is assumed to have a closed form expression.
The Edgeworth series expansion requires the first few moments of $X$, which are fortunately also known in closed form
since the $k$-th raw moment is $(-i)^n \frac{d^n\phi_T}{du^n}(0)$ and we have an analytical formula for $\phi_T(u)$.
Let $m$ and $s^2$ be the mean and variance of $X$.
Denote by $q(x)$ the normal density function with mean $m$ and variance $s^2$.
Given the first few cumulants $c_k$ of $X$, the 5th order Edgeworth series approximation of the density function of $X$ is
\begin{eqnarray*}
f^{\text{CV}}(x) = q(x) -\frac{c_3}{3!}\frac{d^3q}{dx^3}(x) + \frac{c_4}{4!}\frac{d^4q}{dx^4}(x) - \frac{c_5}{5!}\frac{d^5q}{dx^5}(x).
\end{eqnarray*}
This is the control variate of the density function of $X$,
Note that $f^{\text{CV}}(x)$ is not a probability density function since it can take negative values,
but its Fourier transform is known in closed form as
\begin{eqnarray}
\phi_T^{\text{CV}}(u)   &=& \int_{-\infty}^{\infty}e^{iux}\left(q(x) -\frac{c_3}{3!}\frac{d^3q}{dx^3}(x) + \frac{c_4}{4!}\frac{d^4q}{dx^4}(x) - \frac{c_5}{5!}\frac{d^5q}{dx^5}(x)\right)\,dx\nonumber\\
                        &=& e^{imu-\frac{s^2 u^2}{2}}\left(\left(1+\frac{c_4}{4!}u^4\right) - i\left(\frac{c_3}{3!}u^3-\frac{c_5}{5!}u^5\right)\right),\label{eq:ctrl_chf}
\end{eqnarray}
which is the control variate of $\phi_T(u)$.
If $f^{\text{CV}}(x)$ behaves similarly to the density function of $X$,
its Fourier transform $\phi_T^{\text{CV}}(u)$ should behave similarly to $\phi_T(u)$.
Thus the integrands of (\ref{eq:PCarr}) and (\ref{eq:BSstyle}) with ``characteristic functions'' $\phi_T^{\text{CV}}(u)$ and $\phi_T(u)$ should behave similarly.
Define the difference of the characteristic function $\phi_T^{\text{Diff}}(u) = \phi_T(u) - \phi_T^{\text{CV}}(u)$.
Now the integrals can be decomposed.
Taking Carr and Madan's algorithm as an example, (\ref{eq:PCarr}) can be rewritten as
\begin{eqnarray}
c &=& \frac{e^{-dk}}{\pi}\int_0^\infty e^{-iuk}\frac{~e^{-rT}a(u)\phi_T^{\text{Diff}}(u-(d+1)i)~}{d^2+d-u^2+i(2d+1)u}\, du \label{eq:PCarrDiff}\\
    &&+ \frac{e^{-dk}}{\pi}\int_0^\infty e^{-iuk}\frac{~e^{-rT}a(u)\phi_T^{\text{CV}}(u-(d+1)i)~}{d^2+d-u^2+i(2d+1)u}\, du, \label{eq:PCarrCV}
\end{eqnarray}
that is the integral of the difference (\ref{eq:PCarrDiff}) plus the integral of the control variate (\ref{eq:PCarrCV}).
%The first and the second integrals (\ref{eq:PCarrDiff}) and (\ref{eq:PCarrCV}) are the difference and the control variate of the option price, respectively.
While numerical integration is still needed to evaluate (\ref{eq:PCarrDiff}), (\ref{eq:PCarrCV}) is known analytically.



Denote the integral (\ref{eq:PCarrCV}) by $c^{\text{CV}}$.
This can be thought of as the control variate of the option price.
Although it does have a closed form expression, it is hard to work out directly from (\ref{eq:PCarrCV}).
To derive the closed form formula, observe that $c^{\text{CV}}$ is just Carr and Madan's pricing formula for a call option with characteristic function $\phi_T^{\text{CV}}(u)$.
Instead of the Fourier transform $\phi_T^{\text{CV}}(u)$, we take a step back and use $f^{\text{CV}}(x)$ to derive the call option, that is
\begin{eqnarray*}
c^{\text{CV}} = e^{-rT}\int_{-\infty}^\infty (S_0 e^s-K)^+\,f^{\text{CV}}(s)\,ds.%\label{eq:cCVint}
\end{eqnarray*}
Since $f^{\text{CV}}(s)$ is the sum of a normal density function and its derivatives, $c^{\text{CV}}$ clearly has an analytical formula.
The formula is derived in \citet{jarrow1982approximate} as
{\small
\begin{eqnarray}
c^{\text{CV}}   &=& S_0\left(1 + \frac{c_3}{3!} + \frac{c_4}{4!} +  \frac{c_5}{5!}\right)e^{m+\frac{s^2}{2}-rT}N\left(\frac{m+s^2-\log\frac{K}{S_0}}{s}\right) - Ke^{-rT} N\left(\frac{m-\log\frac{K}{S_0}}{s}\right)\nonumber\\
                && + Ke^{-rT}q\left(\log\frac{K}{S_0} \right)\left[-\left(\frac{c_5}{5!}\right)p_3\left(\log\frac{K}{S_0} \right) + \left(\frac{c_5}{5!} + \frac{c_4}{4!}\right)p_2\left(\log\frac{K}{S_0} \right)\right.\nonumber\\
                &&\quad\qquad\qquad\qquad\qquad\quad\left. - \left(\frac{c_5}{5!} + \frac{c_4}{4!}  + \frac{c_3}{3!}\right)p_1\left(\log\frac{K}{S_0} \right) + \left(\frac{c_5}{5!} + \frac{c_4}{4!}  + \frac{c_3}{3!}\right)\right],\label{eq:cCV}
\end{eqnarray}
}
where $N(x)$ is the standard normal distribution function, $q(x)$ is the normal density function with mean $m$ and variance $s^2$, and
\begin{eqnarray*}
%q(x) &=& \frac{1}{\sqrt{2\pi}} e^{-\frac12\left(\frac{x-m}{s}\right)^2},\\
p_k(x) &=& \frac{d^kq(x)}{dx^k}/q(x),\\
p_1(x) &=& (m-x)/s^2, \\
p_2(x) &=& \left(m^2-s^2-2mx+x^2\right)/s^4, \\
p_3(x) &=& (m-x)\left(m^2-3s^2-2mx+x^2\right)/s^6.
\end{eqnarray*}
To sum up, the modified Carr and Madan's algorithm is
\begin{eqnarray*}
c = \frac{e^{-dk}}{\pi}\int_0^\infty e^{-iuk}\frac{~e^{-rT}a(u)\phi_T^{\text{Diff}}(u-(d+1)i)~}{d^2+d-u^2+i(2d+1)u}\, du + c^{\text{CV}},
\end{eqnarray*}
where the integral is evaluated with numerical integration such as the trapezoidal rule,
and $c^{\text{CV}}$ is given by the above closed form formula.





In a similar manner, the control variate technique can also be applied to modify the Black-Scholes Style formula (\ref{eq:BSstyle}).
With the same control variate $\phi_T^{\text{CV}}(u)$ as in (\ref{eq:ctrl_chf}),
we take the decomposition $\phi_T(u) = \phi_T^{\text{CV}}(u) + \phi_T^{\text{Diff}}(u)$ and rewrite $\Pi_1$ as
\begin{eqnarray*}
\Pi_1 &=& \frac{1}{2} + \frac{1}{\pi}\int_0^{\infty} \text{Re}\left[\frac{e^{-iu \log\left(\frac{K}{S_0}\right)}\left(\phi_T^{\text{CV}}(u-i) +  \phi_T^{\text{Diff}}(u - i)\right)}{iu\phi_T(-i)}\right]\,du\\
&=& \frac{1}{2}
+ \frac{1}{\pi}\int_0^{\infty} \text{Re}\left[\frac{e^{-iu \log\left(\frac{K}{S_0}\right)}\phi_T^{\text{CV}}(u-i)}{iu\phi_T(-i)}\right]\,du
+ \frac{1}{\pi}\int_0^{\infty} \text{Re}\left[\frac{e^{-iu \log\left(\frac{K}{S_0}\right)}\phi_T^{\text{Diff}}(u - i)}{iu\phi_T(-i)}\right]\,du.
\end{eqnarray*}
Denote the control variate and the difference of $\Pi_1$ as
\begin{eqnarray*}
\Pi_1^{\text{CV}} &=& \frac{1}{2} + \frac{1}{\pi}\int_0^{\infty} \text{Re}\left[\frac{e^{-iu \log\left(\frac{K}{S_0}\right)}\phi_T^{\text{CV}}(u-i)}{iu\phi_T(-i)}\right]\,du,\\
\Pi_1^{\text{Diff}} &=& \frac{1}{\pi}\int_0^{\infty} \text{Re}\left[\frac{e^{-iu \log\left(\frac{K}{S_0}\right)}\phi_T^{\text{Diff}}(u - i)}{iu\phi_T(-i)}\right]\,du.
\end{eqnarray*}
Similarly, rewrite $\Pi_2$ as
\begin{eqnarray*}
\Pi_2 &=& \frac{1}{2} + \frac{1}{\pi}\int_0^{\infty} \text{Re}\left[\frac{e^{-iu \log\left(\frac{K}{S_0}\right)}\phi_T^{\text{CV}}(u)}{iu}\right]\,du
+ \frac{1}{\pi}\int_0^{\infty} \text{Re}\left[\frac{e^{-iu \log\left(\frac{K}{S_0}\right)}\phi_T^{\text{Diff}}(u)}{iu}\right]\,du,
\end{eqnarray*}
where we denote
\begin{eqnarray*}
\Pi_2^{\text{CV}} &=& \frac{1}{2} + \frac{1}{\pi}\int_0^{\infty} \text{Re}\left[\frac{e^{-iu \log\left(\frac{K}{S_0}\right)}\phi_T^{\text{CV}}(u)}{iu}\right]\,du,\\
\Pi_2^{\text{Diff}} &=& \frac{1}{\pi}\int_0^{\infty} \text{Re}\left[\frac{e^{-iu \log\left(\frac{K}{S_0}\right)}\phi_T^{\text{Diff}}(u)}{iu}\right]\,du.
\end{eqnarray*}
Obviously we can write $\Pi_1 = \Pi_1^{\text{CV}} + \Pi_1^{\text{Diff}}$ and $\Pi_2 = \Pi_2^{\text{CV}} + \Pi_2^{\text{Diff}}$.
Plugging these into the Black-Scholes pricing formula (\ref{eq:BSstyle}), we obtain the modified algorithm
\begin{eqnarray*}
c &=& S_0 \left(\Pi_1^{\text{CV}} + \Pi_1^{\text{Diff}}\right) - K e^{-rT} \left(\Pi_2^{\text{CV}} + \Pi_2^{\text{Diff}}\right) \\
&=&  S_0 \Pi_1^{\text{Diff}} - K e^{-rT} \Pi_2^{\text{Diff}} + c^{\text{CV}},
\end{eqnarray*}
where we denote $c^{\text{CV}} = S_0 \Pi_1^{\text{CV}} - K e^{-rT} \Pi_2^{\text{CV}}$.
It is the same control variate of the price of the call option as the one used in the modified Carr and Madan algorithm.
The closed form formula is given above.




\subsection{Calibrating the Control Model}
\label{sec:find_cumulants}

Using the model based on the Edgeworth expansion as the control, we have five parameters to estimate -- $m, s, c_3, c_4$ and $c_5$.
We want to find a set of parameters so that the $\phi_T^{\text{CV}}(u)$ function in (\ref{eq:ctrl_chf})
behaves similarly to the characteristic function used in pricing.
The assumption of the Edgeworth expansion is that $m$ and $s$ are the mean and the standard deviation of the distribution to be approximated,
and $c_3$, $c_4$ and $c_5$ are its 3rd, 4th and 5th cumulants.
In our case of interest the distribution to be approximated is the distribution of the log-return $\log(S_T/S_0)$.
One way of finding the cumulants is to find the raw moments first, through differentiating the closed form characteristic function $\phi_T(u)$.
Let $m_i$ be the $i$-th raw moment.
Then the the mean and the standard deviation are obviously given by
\begin{eqnarray*}
m &=& m_1,\\
s &=& \sqrt{m_2 - m_1^2}.
\end{eqnarray*}
The cumulants can be obtained by the following formulas
\begin{eqnarray*}
c_3 &=& 2 m_1^3-3 m_1 m_2+m_3, \\
c_4 &=& -6 m_1^4+12 m_1^2 m_2-4 m_1 m_3-3 m_2^2+m_4, \\
c_5 &=& 24 m_1^5-60 m_1^3 m_2+20 m_1^2 m_3+30 m_1 m_2^2-5 m_1 m_4-10 m_2 m_3+m_5.
\end{eqnarray*}
This way of getting the cumulants is, however, sometimes not very convenient.



Many, if not all, popular derivatives pricing models have characteristic functions of the form $\phi_T(u) = e^{\psi_T(u)}$,
where $\psi_T(u)$ is called the characteristic exponent.
Examples include all exponential L\'evy models and the Heston stochastic volatility model.
When differentiating a characteristic function $\phi_T(u)$ of this form to get the raw moments, the expression soon gets complicated.
A better way to find cumulants is to use the definition
-- the $k$-th cumulant of a random variable $X$ is the $k$-th coefficient of the power series expansion of $\log E[e^{u X}]$ multiplied by $k!$.
Now to find the cumulants of $\log(S_T/S_0)$,
if the characteristic function is of the form $\phi_T(u) = E[e^{i u \log(S_T/S_0)}] = e^{\psi_T(u)}$
then we just need the power series of
$$
\log E[e^{u \log(S_T/S_0)}] = \log \phi_T(-iu) = \psi_T(-iu),
$$
the first few terms of which will be much easier to write down explicitly than differentiating $\phi_T(u)$ directly.
As an example, consider the exponential L\'evy model developed by \citet{merton1976option}, whose characteristic exponent is given by
$$
\psi_T(u) = iu\left( r- \lambda\left(e^{\alpha+\frac{\beta^2}{2}}-1\right)-\frac{\sigma^2}{2}\right)T -\frac{\sigma^2 u^2}{2}T + \lambda T \left(e^{i\alpha u -\frac{1}{2}\beta^2 u^2}-1\right),
$$
where $\alpha, \beta>0, \sigma>0$ and $\lambda>0$ are model parameters, and $r$ is the constant risk-free interest rate.
Thus
\begin{eqnarray}
\psi_T(-iu) = u\left( r- \lambda\left(e^{\alpha+\frac{\beta^2}{2}}-1\right)-\frac{\sigma^2}{2}\right)T + \frac{\sigma^2 u^2}{2}T + \lambda T \left(e^{\alpha u +\frac{1}{2}\beta^2 u^2}-1\right).\label{eq:charExpJD}
\end{eqnarray}
We expand the last parentheses using the series $e^x - 1 = x + x^2/2! + x^3/3! + x^4/4! + \cdots$.
We obtain
\begin{eqnarray*}
&&e^{\alpha u +\frac{1}{2}\beta^2 u^2} - 1 \\
&=& \left(\alpha u +\frac{1}{2}\beta^2 u^2\right) + \frac12 \left(\alpha u +\frac{1}{2}\beta^2 u^2\right)^2 + \frac16 \left(\alpha u +\frac{1}{2}\beta^2 u^2\right)^3 + \frac{1}{24} \left(\alpha u +\frac{1}{2}\beta^2 u^2\right)^4 + \cdots\\
&=& \alpha u + \frac12 (\alpha^2 + \beta^2) u^2 + \frac16 (\alpha^3 + 3 \alpha \beta^2) u^3 + \frac{1}{24} (\alpha^4 + 6 \alpha^2 \beta^2 + 3 \beta^4) u^4 + \cdots.
\end{eqnarray*}
Pluggin this into (\ref{eq:charExpJD}) we obtain the power series
\begin{eqnarray*}
\psi_T(-iu) &=& \left( r- \lambda\left(e^{\alpha+\frac{\beta^2}{2}}-1-\alpha\right)-\frac{\sigma^2}{2}\right)Tu + \frac{1}{2}(\sigma^2 + \lambda(\alpha^2 + \beta^2))T u^2\\
&&+ \frac16 \lambda \alpha(\alpha^2 + 3\beta^2)Tu^3 + \frac{1}{24} \lambda(\alpha^4 + 6\alpha^2\beta^2 + 3\beta^4)Tu^4 + \cdots.
\end{eqnarray*}
By definition this is equal to $\sum_{n=1}^\infty c_n\frac{u^n}{n!}$, with $c_n$ the $n$-th cumulant.
Thus we have
\begin{eqnarray*}
c_1 &=& \left( r- \lambda\left(e^{\alpha+\frac{\beta^2}{2}}-1-\alpha\right)-\frac{\sigma^2}{2}\right)T, \\
c_2 &=& (\sigma^2 + \lambda(\alpha^2 + \beta^2))T, \\
c_3 &=& \lambda \alpha(\alpha^2 + 3\beta^2)T, \\
c_4 &=& \lambda(\alpha^4 + 6\alpha^2\beta^2 + 3\beta^4)T,\\
&& \vdots
\end{eqnarray*}
This is the first few cumulants of $\log(S_T/S_0)$ under Merton's model.
The first two cumulants are also the mean and the variance, so we have
$$
m = c_1, \qquad s = \sqrt{c_2}.
$$
In general for any derivatives pricing model with a characteristic function of the form $\phi_T(u) = e^{\psi_T(u)}$,
the first few cumulants can be easily computed by the same methodology.



The derivation of some FT-based pricing algorithms imply that the characteristic function satisfies $\phi_T(-i) = e^{rT}$.
This is a consequence of the assumption that $\log(S_T/S_0) - rT$ is a martingale.
The risk-neutral probability measure is constructed to guarantee this property.
While the characteristic function of a log-return process under the risk-neutral probability should satisfy $\phi_T(-i) = e^{rT}$,
in general this is not always true for the characteristic function of the distribution inferred from a set of moments $m, s, c_3, c_4, c_5$.
When the FT-based algorithm to be modified with the control variate technique has this property,
the coefficients of the control model also have to satisfy $\phi_T^{\text{CV}}(-i) = e^{rT}$ so that
the $c^{\text{CV}}$ formula (\ref{eq:cCV}) and applying the algorithm on $\phi_T^{\text{CV}}$ would give the same result.
Thus we get one less degree of freedom to choose the parameters of the control model.
We set $c_3, c_4$ to be the 3rd and the 4th cumulants of the target model, and then determine $c_5$ by solving $\phi_T(-i) = e^{rT}$.
As a result, we obtain
\begin{eqnarray}
c_5 = 5!\left(e^{rT-m-\frac{s^2}{2}} - \left(1 + \frac{c_3}{3!} + \frac{c_4}{4!}\right)\right).\label{eq:c5}
\end{eqnarray}


In practice when applying the control variate technique to modify an FT-based pricing algorithm,
we do not want to have go through the derivation just to check if it implies $\phi_T(-i) = e^{rT}$.
For convenience, regardless of the type of the FT-based algorithm,
we always use (\ref{eq:c5}) to determine the last coefficient $c_5$ of the control model, instead of using the 5th cumulant.




\subsection{How to Find a Control Variate?}


It is also possible to choose another model as the control variate,
although a control should satisfy two conditions:
\begin{enumerate}
\item $\phi_T^{\text{CV}}(u)$ should behave similarly to $\phi_T(u)$, and
\item the control variate of the option value $c^{\text{CV}}$ is known in closed form.
\end{enumerate}
The second condition means that we need a model under which the option price is known analytically,
and the first one ensures that the difference of the integrand takes small values.



When we choose a control model that satisfies the second condition,
we also need to determine the model parameters so that $\phi_T^{\text{CV}}(u)$ and $\phi_T(u)$ have similar behavior.
This is a calibration problem.
We want to calibrate our control model to the target model, which is the model under which $\phi_T(u)$ is derived.
Since the FT-based pricing algorithms are already computationally cheap,
it would not make much sense if some rather expensive numerical optimization algorithm is involved for calibration.
This explains why the model based on the Edgeworth series is a good control variate.
The parameters of an Edgeworth expansion are just the moments and cumulants of the target model, which are known in closed form.



The Black-Scholes model also has a closed form option pricing formula, and is easy to calibrate.
Thus one can use it as the control model.
This is considered by both \citet{tankov2003financial} and \citet{schmelzle2010option}, and is actually a special case of the Edgeworth series expansion.
Indeed, while the Black-Scholes model only fits the second moment (the volatility),
the Edgeworth series expansion is able to capture the skewness of the target model by fitting higher order moments.
Numerical results are provided later to show that adding the information about the 3rd and the 4th moments of the target model really does yield more accuracy.



\subsection{Numerical Results}

We compare Carr and Madan's algorithm with its control variate modifications.
We use both the Black-Scholes model and the Edgeworth expansion as the control variates
to evaluate a vanilla call option with maturity $T=1$, strike price $K=120$, and initial underlying asset price $S_0=100$ under the Heston model.
The model description and its characteristic function are given in Appendix \ref{appendix:heston}.
The parameters used here are $r = 0.1$, $\kappa = 2$, $\theta = 0.04$, $\sigma = 0.5$, $\rho = -0.7$, $V_0 = 0.04$.
The benchmark is 3.29968857626, obtained by the original Carr and Madan's formula (\ref{eq:PCarr}) evaluated by the trapezoidal rule with domain of integral $[0, 200]$ and number of partition $n=10000$.
A comparison of the efficiency and the accuracy are shown in Figure \ref{tab:CM}.
``Carr'' refers to the original Carr and Madan's algorithm,
``BSCV'' means its modification with the Black-Scholes model as the control variate,
and ``EWCV'' is the result of the modification with the Edgeworth expansion as the control variate.
For all three algorithms the trapezoidal rule is used to integrate over the interval $[0, 200]$.
The first column $n$ is the number of partition.
The ``CPU (ms)'' columns give the running time of the algorithms in milliseconds,
and the ``log error'' columns are the error of the pricing results taking common logarithm.
Roughly speaking, the log error being $-(1+m)$ means the result has $m$-decimal-place accuracy.



From the table in Figure \ref{tab:CM} we can see that given the same number of partition, the control variate modifications always need more time to accomplish the task.
This is because the most time consuming part of an FT-based algorithm is the characteristic function evaluation,
and with control variate we need $\phi_T^{\text{Diff}}(u) = \phi_T(u) - \phi_T^{\text{CV}}(u)$, that is one more characteristic function to evaluate.
But this extra cost is worth it because, as we can see from the table, the accuracy is greatly improved by the control variate.
In around 3 milliseconds, we get one-decimal-place accuracy with the Black-Scholes control,
three-decimal-place accuracy with the Edgeworth expansion,
while the pricing error of Carr and Madan's algorithm is just slightly less than 1.
From this table it is obvious that using a control variate in Carr and Madan's algorithm does improve the accuracy and efficiency,
and that the control variate based on Edgeworth expansion is better than a Black-Scholes control.
Figures \ref{fig:CM} and \ref{fig:CMn} visualize the same numerical results.



\begin{figure}[!t]
\begin{center}
\begin{tabular}{rrrrrrr}
\toprule
{} &      Carr &           &  CarrBSCV &           &  CarrEWCV &           \\
{$n$} &  CPU (ms) & log error &  CPU (ms) & log error &  CPU (ms) & log error \\
\midrule
20  &  0.759470 &  1.549590 &  0.963912 & -1.295336 &  1.577646 & -0.985337 \\
30  &  0.961859 &  1.317691 &  1.294384 & -1.233819 &  2.081359 & -2.013734 \\
40  &  1.243889 &  1.113891 &  1.654414 & -1.434872 &  2.564547 & -3.078093 \\
50  &  1.501288 &  0.918457 &  2.095728 & -1.659546 &  3.076060 & -4.121298 \\
60  &  1.789476 &  0.724625 &  2.454526 & -1.879688 &  3.592500 & -5.167961 \\
70  &  1.993918 &  0.529981 &  2.828925 & -2.094297 &  4.149171 & -6.218208 \\
80  &  2.282106 &  0.333783 &  3.207428 & -2.305139 &  4.639338 & -7.271658 \\
90  &  2.564547 &  0.135955 &  3.564585 & -2.513626 &  5.235830 & -8.327967 \\
100 &  2.837956 & -0.063333 &  3.883152 & -2.720658 &  5.723533 & -9.386850 \\
\bottomrule
\end{tabular}
\end{center}
\caption{
Efficiency and Accuracy Comparison of Carr and Madan's and Its Control Variate Modifications.
``Carr'' is the original Carr and Madan's algorithm,
``CarrBSCV'' its control variate modification with the Black-Scholes control,
and ``CarrEWCV' the modification with control variate based on the Edgeworth expansion.
CPU (ms) is the computational time in milliseconds,
and log error is the pricing error taking common logarithm.
}\label{tab:CM}
\end{figure}


\begin{figure}[!t]
\begin{center}
    \parbox[t]{0.8\textwidth}{
        \centerline{\epsfig{figure=eps/CM.eps, width=0.8\textwidth}}
    }\hfill
\end{center}
\caption{
Efficiency and Accuracy Comparison of Carr and Madan's and Its Control Variate Modifications.
}\label{fig:CM}
\end{figure}


\begin{figure}[!t]
\begin{center}
    \parbox[t]{0.8\textwidth}{
        \centerline{\epsfig{figure=eps/CMn.eps, width=0.8\textwidth}}
    }\hfill
\end{center}
\caption{
Accuracy Comparison of Carr and Madan's and Its Control Variate Modifications.
``num'' is the number of partitions in the numerical integration.
}\label{fig:CMn}
\end{figure}


Figure \ref{tab:BSstyle} gives an efficiency and accuracy comparison of the Black-Scholes style formula and its control variate modifications.
Unlike Carr and Madan's algorithm, however, we get no benefit from the control variate technique.
Given the same number of partition, adding the Black-Scholes type control variate to the algorithm will only improve the pricing error slightly.
If the Edgeworth expansion is used as a control, we also get slightly more accurate pricing results than using the Black-Scholes control.
But if we take the computational time into consideration, the original algorithm is still the best.
As aforementioned, in a control variate modification of an FT-based pricing algorithm,
to find the difference $\phi_T^{\text{Diff}}(u) = \phi_T(u) - \phi_T^{\text{CV}}(u)$,
two characteristic functions need to be evaluated, one for the target model and one of the control model.
This extra cost is not worth it if the pricing accuracy can only improve by a little.



\begin{figure}[!t]
\begin{center}
\begin{tabular}{lrrrrrr}
\toprule
{} &   BSstyle &           & BSCV &           & EWCV &           \\
{$n$} &  CPU (ms) & log error &    CPU (ms) & log error &    CPU (ms) & log error \\
\midrule
20 &  1.689719 &  0.180407 &    2.118307 &  0.061376 &    2.734093 &  0.255447 \\
22 &  1.753350 &  0.069399 &    2.292369 &  0.007426 &    3.049376 &  0.183681 \\
24 &  1.871171 & -0.045524 &    2.470126 & -0.076276 &    3.160628 & -0.017929 \\
26 &  2.040307 & -0.163657 &    2.620789 & -0.178003 &    3.375333 & -0.378673 \\
28 &  2.149506 & -0.284444 &    2.823998 & -0.290679 &    3.571153 & -1.280356 \\
30 &  2.302632 & -0.407447 &    3.146260 & -0.409957 &    3.790784 & -0.884222 \\
32 &  2.439337 & -0.532324 &    3.198807 & -0.533256 &    4.006719 & -0.725494 \\
34 &  2.615862 & -0.658805 &    3.419669 & -0.659124 &    4.366339 & -0.741271 \\
36 &  2.720136 & -0.786677 &    3.585111 & -0.786777 &    4.468560 & -0.820410 \\
38 &  3.015303 & -0.915768 &    3.700468 & -0.915797 &    4.651654 & -0.928571 \\
40 &  3.000524 & -1.045942 &    4.028477 & -1.045950 &    4.902895 & -1.050385 \\
\bottomrule
\end{tabular}

\end{center}
\caption{
Efficiency and Accuracy Comparison of Black-Scholes Style Formula and Its Control Variate Modifications.
}\label{tab:BSstyle}
\end{figure}





\section{Convolution Pricing Algorithms}
\label{sec:CONV}
So far our discussion is limited to one-step pricing algorithms.
The entire algorithm is of the form $\mathbb T(f)$, an operator acting on some function.
But there are pricing algorithms that can be thought of as a sequence of operators acting on a function,
algorithms of the form $\mathbb T_n(\cdots\mathbb T_3(\mathbb T_2(\mathbb T_1(f)))\cdots)$.
Examples include the binomial trees and the convolution pricing algorithms.
Here we demonstrate how the control variate technique can be applied to the latter group.
The binomial tree algorithm with control variates will be discussed in the next section.



\subsection{Background Review}
\label{sec:CONVBackgroundReview}
The convolution pricing algorithms developed by \citet{lord2008fast} are a modified version of the quadrature methods suggested by \citet{andricopoulos2003universal}
that can deal with the pricing problems of many exotic options including Bermudan options, discrete barrier options, look back options and Asian options (see \citet{vcerny2011improved}).
% under a wide class of option pricing models, including all exponential L\`evy models.
Although it only deals with discretely monitored exotic options,
a Richardson extrapolation can be applied to obtain the value of continuously monitored options.
The convolution pricing algorithm is based on the following theorem.
\begin{theorem}
\label{thm:CONV}
Let $W_t$ be the standard Brownian motion, and $\mathcal F_t$ a filtration for the Brownian motion.
Let $f(x)$ be a Borel measurable function, and $X_t = \mu t + \sigma W_t$.
Then, for all $\Delta t>0$, $t>0$,
$$
{\mathsf E}[f(X_{t+\Delta t})|\mathcal F_t] = \int_{-\infty}^\infty f(y)p(x-y)\,dy,
$$
where $x = X_t$, and
$$
p(z) = \frac{1}{\sigma\sqrt{2\pi\Delta t}}~e^{-\frac{(z + \mu\Delta t)^2}{2\sigma^2\Delta t}}
$$
is the transition density function.
In other words, the conditional expectation ${\mathsf E}[f(X_{t+\Delta t})|\mathcal F_t]$ is the convolution of $f(x)$ and a transition density $p(x)$, which we denote by
$$
{\mathsf E}[f(X_{t+\Delta t})|\mathcal F_t] = f(x)\otimes p(x).
$$
\end{theorem}
Due to the Markov property of the Brownian motion, the conditional expectation ${\mathsf E}[f(X_{t+\Delta t})|\mathcal F_t]$ ought to be a function of $X_t$.
The above theorem gives the function explicitly.
For a detailed proof of the theorem see, for example, \citet{shreve2004stochastic}.






The idea behind the convolution pricing algorithm is easy to understand.
Suppose, under the Black-Scholes model, we are to evaluate a discretely monitored exotic option with maturity $T$ and observation points $0 < t_1 < t_2 < \ldots < t_n = T$.
For simplicity assume that $\Delta t = t_{i+1} - t_i$ for all $i=1, 2, \ldots, n-1$.
Assume that under the risk-neutral probability measure the price of the underlying asset follows the log-normal diffusion process $dS_t = rS_tdt + \sigma S_tdW_t.$
Depending on the type of the option to be evaluated, at observation points the option value could jump.
For example, an up-and-out barrier option with barrier $B$ at time $t_i-$ is worthless if $S_t>B$ because it is going to knock out,
but at $t_i+$ it has positive time value since $S_t$ could drop below $B$ before next observation point.
Another example is that the value of a Bermudan put with strike $K$ at time $t_i-$ should be $K - S_{t_i}$ if $S_{t_i}$ is too low,
because the put holder has the right to exercise the contract right away,
but at $t_i+$ the contract can not be exercised until next observation point, so the value drops.
Apart from the observation points, in each time interval $(t_i , t_{i+1})$ the contract is just a vanilla European option, i.e., no early exercise, no knock in nor knock out,
so the price of the contract at $t_i+$ should be the discounted expected value of the price at $t_{i+1}-$ conditional on the filtration at $t_{i}$, which can be evaluated by a convolution due to Theorem \ref{thm:CONV}.
A convolution pricing algorithm begins with the payoff function (in log scale) and evaluates the option price backward in time.
The above argument gives an intuitive explanation of the algorithms in Figures \ref{algo:conv_berm} and \ref{algo:conv_barrier}.





\begin{figure}[!t]
\begin{framed}
\begin{algorithmic}[1]
    \STATE payoff$(x) \leftarrow \max(K - S_0e^x, 0)$
    \STATE put$(t_n, x) \leftarrow$ payoff$(x)$
    \STATE $p(x) \leftarrow e^{-(x+(r-\sigma^2/2)\Delta t)^2/(2\sigma^2\Delta t)}/(\sigma\sqrt{2\pi\Delta t})$
    \FOR{$i=n$ \textbf{downto} $1$ }
        \STATE $\text{put}(t_i, x) \leftarrow \max(\text{ put}(t_i, x), \text{ payoff}(x)~)$
        \STATE $\text{put}(t_{i-1}, x) \leftarrow e^{-r\Delta t}( \text{put}(t_i, x)\otimes p(x) )$
    \ENDFOR
    \STATE \textbf{return} $\text{put}(t_0, 0)$
\end{algorithmic}
\end{framed}
\caption{ Convolution Pricing Algorithm for Bermudan Put Options
} \label{algo:conv_berm}
\end{figure}



\begin{figure}[!th]
\begin{framed}
\begin{algorithmic}[1]
    \STATE payoff$(x) \leftarrow \max(S_0e^x - K, 0)$
    \STATE call$(t_n, x) \leftarrow$ payoff$(x)$
    \STATE $p(x) \leftarrow e^{-(x+(r-\sigma^2/2)\Delta t)^2/(2\sigma^2\Delta t)}/(\sigma\sqrt{2\pi\Delta t})$
    \FOR{$i=n$ \textbf{downto} $1$ }
        \STATE $\text{call}(t_i, x) \leftarrow \text{call}(t_i, x)\times 1_{\{x<\log(B/S_0)\}}$
        \STATE $\text{call}(t_{i-1}, x) \leftarrow e^{-r\Delta t}( \text{call}(t_i, x)\otimes p(x) )$
    \ENDFOR
    \STATE \textbf{return} $\text{call}(t_0, 0)$
\end{algorithmic}
\end{framed}
\caption{ Convolution Pricing Algorithm for Up-And-Out Barrier Call Options
} \label{algo:conv_barrier}
\end{figure}





In both Figures \ref{algo:conv_berm} and \ref{algo:conv_barrier}, $p(x)$ is the transition density function.
The value of the option is set to the payoff in log scale at time $T$ (see line 2).
The variable $x$ represents $\log(S_{t_i}/S_0)$ at time $t_i$.
In line 5 the option value is adjusted according to what should happen at observation point $t_i$.
A Bermudan put holder will decide if he or she wants to exercise the contract and get payoff$(x) = K-S_{t_i}$ in line 5, or keep the contract with the value unchanged.
An up-and-out call is set to 0 at observation point $t_i$ if $x>\log(B/S_0)$, or equivalently $S_{t_i} > B$.
Before line 5, the function put$(\cdot, x)$ or call$(\cdot, x)$ keeps the value of the option at $t_i+$;
after line 5, it keeps the value at $t_i-$.
The discounted conditional expected value is evaluated in line 6 by a convolution,
or more explicitly,
\begin{eqnarray}
\text{put}(t_{i-1}, x) &=& e^{-r\Delta t}\int_{-\infty}^\infty \text{put}(t_{i}, y)p(x-y)\,dy, \label{eq:line6conv}\\
\text{call}(t_{i-1}, x) &=& e^{-r\Delta t}\int_{-\infty}^\infty \text{call}(t_{i}, y)p(x-y)\,dy, \label{eq:line6convBarrier}
\end{eqnarray}
for Figures \ref{algo:conv_berm} and \ref{algo:conv_barrier}, respectively.
Finally when the loop terminates, the option value at $x=0$ is returned.
Many convolution pricing algorithm has a similar form as in Figures \ref{algo:conv_berm} and \ref{algo:conv_barrier}.
In those algorithms, one just needs to change the transition density function $p(x)$ according to the model,
and to change line 5 according to the type of the option to be evaluated.




To implement a convolution pricing algorithm, one first sets up a wide interval $[-a, a]$ and a uniform grid in the interval.
Suppose the grid points are $-a = x_0 < x_1 < x_2 \cdots < x_m = a$, where $x_j = x_0 + jh$, $h = x_1 - x_0$.
% For simplicity, further assume that $m$ is even so that $x_{m/2} = 0$.
% This way when we return $\text{call}(t_0, 0)$ or $\text{put}(t_0, 0)$ in line 8 in Figure \ref{algo:CONV}, since 0 is one of the grid points, no interpolation is needed.
In the implementation, all the functions are kept as an array of function values at $x_0, x_1, \ldots, x_m$.
This is called a grid function.
For example, the transition density $p(x)$ in line 3 in Figures \ref{algo:conv_berm} and \ref{algo:conv_barrier} is stored as an array $\{p(x_0), p(x_1), \ldots, p(x_m)\}$.
To evaluate the convolution in line 6, or (\ref{eq:line6conv}) and (\ref{eq:line6convBarrier}), numerical integration and discrete convolution are involved,
which is typically done by the fast Fourier transform (FFT) algorithm in $O(m\log m)$ computational time.
Note that line 6 is the only place in the algorithm where we need a numerical approximation.
Indeed, the payoff function and the transition density both have a closed form formula,
and at each $x_j$, the two statements in line 5
\begin{eqnarray*}
\text{put}(t_i, x_j) &\leftarrow& \max(\text{ put}(t_i, x_j), \text{ payoff}(x_j)~)\\
\text{call}(t_i, x_j) &\leftarrow& \text{call}(t_i, x_j)\times 1_{\{x_j<\log(B/S_0)\}}
\end{eqnarray*}
will not introduce any numerical error.



\subsection{Modified Algorithms}
In every convolution pricing algorithm there is a convolution step like line 6 in Figures \ref{algo:conv_berm} and \ref{algo:conv_barrier}.
Since convolution is a linear operator, its numerical evaluation can be modified by the control variate technique,
and if we choose the control variate appropriately, the resulting algorithm will have better accuracy.
Although we only modify the algorithms in Figures \ref{algo:conv_berm} and \ref{algo:conv_barrier} as examples,
the statement below is general and works for all convolution pricing algorithms.


% �U���n�A���p�`���g�k
%
% Recall that every convolution pricing algorithm has a similar form as Figures \ref{algo:conv_berm} and \ref{algo:conv_barrier}, just line 3 and 5 are different.
% Below, we will first give a statement on how the control variate technique can be applied to a convolution pricing algorithm,
% where only line 6 is modified.
% Thus the statement is general and works for all convolution pricing algorithms.
% We then discuss possible control variates for Bermudan put and discrete barrier call option separately.





\begin{figure}[!t]
\begin{framed}
\begin{algorithmic}[1]
    \STATE payoff$(x) \leftarrow \max(K - S_0e^x, 0)$
    \STATE put$(t_n, x) \leftarrow$ payoff$(x)$
    \STATE $p(x) \leftarrow e^{-(x+(r-\sigma^2/2)\Delta t)^2/(2\sigma^2\Delta t)}/(\sigma\sqrt{2\pi\Delta t})$
    \FOR{$i=n$ \textbf{downto} $1$ }
        \STATE $\text{put}(t_i, x) \leftarrow \max(\text{ put}(t_i, x), \text{ payoff}(x)~)$
        \STATE $\text{put}_{\text{Diff}}(t_{i}, x) \leftarrow \text{put}(t_{i}, x) - \text{put}_{\text{CV}}(t_{i}, x)$
        \STATE $ \text{put}(t_{i-1}, x) \leftarrow e^{-r\Delta t}( \text{put}_{\text{Diff}}(t_i, x)\otimes p(x) ) + e^{-r\Delta t}( \text{put}_{\text{CV}}(t_i, x)\otimes p(x) )$
    \ENDFOR
    \STATE \textbf{return} $\text{put}(t_0, 0)$
\end{algorithmic}
\end{framed}
\caption{Bermudan Put Pricing Algorithm with Control Variate.
The control variate $\text{put}_{\text{CV}}(t_{i}, x)$ is chosen in a way that $e^{-r\Delta t}( \text{put}_{\text{CV}}(t_i, x)\otimes p(x) )$ is known analytically.
Thus only the first convolution in line 7 needs a numerical approximation.
}
\label{algo:conv_berm_CV}
\end{figure}




As mentioned earlier, the convolution step in line 6 in Figures \ref{algo:conv_berm} and \ref{algo:conv_barrier} is the only place in the algorithms where a numerical approximation is needed.
Take the Bermudan put pricing as an example.
We use a discrete convolution to approximate (\ref{eq:line6conv}),
which is the only part that needs to be taken care of in order to improve the accuracy of the algorithm.
Set $\mathsf T(f) = f\otimes p$, the ``convolve with $p(x)$'' operator.
The operator $\mathbb T$ that approximates $\mathsf T$ is a discrete convolution in this example.
Since $\mathsf T$ is linear, we can apply the control variate technique and use the better approximation $\mathbb T(f-g) + \mathsf T(g)$ in stead of $\mathbb T(f)$, as illustrated in (\ref{eq:GCVapprox}).
In the context here, to evaluate (\ref{eq:line6conv}),
first find a control variate $\text{put}_{\text{CV}}(t_i, y)$, evaluate $\text{put}_{\text{Diff}}(t_i, y) = \text{put}(t_i, y) - \text{put}_{\text{CV}}(t_i, y)$,
and then rewrite (\ref{eq:line6conv}) as
\begin{eqnarray}
\text{put}(t_{i-1}, x) = e^{-r\Delta t}\int_{-\infty}^\infty \text{put}_{\text{Diff}}(t_{i}, y)p(x-y)\,dy + e^{-r\Delta t}\int_{-\infty}^\infty \text{put}_{\text{CV}}(t_{i}, y)p(x-y)\,dy.\label{eq:line6GCV}
\end{eqnarray}
Now the discrete convolution is only needed for the first integral.
As have been discussed in earlier sections, there are two conditions a control variate should satisfy.
In the example here, the conditions $\text{put}_{\text{CV}}(t_i, y)$ should satisfy are
\begin{enumerate}
\item $\text{put}_{\text{CV}}(t_i, y)$ should behave similarly to $\text{put}(t_i, y)$, and
\item $\mathsf T( \text{put}_{\text{CV}} ) = \int_{-\infty}^\infty \text{put}_{\text{CV}}(t_{i}, y)p(x-y)\,dy$ is known in closed form.
\end{enumerate}
%The second condition ensures that we do not need a numerical approximation for the second integral in (\ref{eq:line6GCV}),
The first condition guarantees a small error when we apply the discrete convolution on the first integral in (\ref{eq:line6GCV}),
and the second condition precludes the need of a numerical approximation for the second integral. % in (\ref{eq:line6GCV}),
Unlike modified one-step algorithms where only one control variate is needed,
here we need $n$ control variates $\text{put}_{\text{CV}}(t_1, y), \text{put}_{\text{CV}}(t_2, y), \ldots, \text{put}_{\text{CV}}(t_n, y)$,
one for each observation point $t_i$.
The modified Bermudan put pricing algorithm is given in Figure \ref{algo:conv_berm_CV}.




The only difference between the original algorithm given in Figures \ref{algo:conv_berm} and \ref{algo:conv_barrier} and the modified version given in Figure \ref{algo:conv_berm_CV} is that
line 6 in the original algorithm is now rewritten as line 6 and 7 in the modified algorithm.
We emphasize that in the modified algorithm, $e^{-r\Delta t}( \text{put}_{\text{CV}}(t_i, x)\otimes p(x) )$ is known in closed form.
This is the second condition that must be satisfied by $\text{put}_{\text{CV}}(t_{i}, x)$.
We must have the closed form formulas for $\text{put}_{\text{CV}}(t_{i}, x)$ and $e^{-r\Delta t}( \text{put}_{\text{CV}}(t_i, x)\otimes p(x) )$
so that we can take the grid functions when implementing line 6 and 7.




The discrete barrier option pricing algorithm in Figure \ref{algo:conv_barrier} can also be modified in a similar manner.
One only needs to rewrite (\ref{eq:line6convBarrier}) as
\begin{eqnarray}
\text{call}(t_{i-1}, x) = e^{-r\Delta t}\int_{-\infty}^\infty \text{call}_{\text{Diff}}(t_{i}, y)p(x-y)\,dy + e^{-r\Delta t}\int_{-\infty}^\infty \text{call}_{\text{CV}}(t_{i}, y)p(x-y)\,dy, \label{eq:line6GCVBarrier}
\end{eqnarray}
where $\text{call}_{\text{CV}}(t_{i}, y)$ is the control variate,
and $\text{call}_{\text{Diff}}(t_i, y) = \text{call}(t_i, y) - \text{call}_{\text{CV}}(t_i, y)$ the difference.
The control variate $\text{call}_{\text{CV}}(t_{i}, y)$ should be chosen in a way that the following two conditions are satisfied:
\begin{enumerate}
\item $\text{call}_{\text{CV}}(t_i, y)$ should behave similarly to $\text{call}(t_i, y)$, and
\item $\mathsf T( \text{call}_{\text{CV}} ) = \int_{-\infty}^\infty \text{call}_{\text{CV}}(t_{i}, y)p(x-y)\,dy$ is known in closed form.
\end{enumerate}
The modified algorithm is shown in Figure \ref{algo:conv_barrier_CV}.
Obviously the same methodology can be used to modify all convolution pricing algorithms.







\begin{figure}[!t]
\begin{framed}
\begin{algorithmic}[1]
    \STATE payoff$(x) \leftarrow \max(S_0e^x - K, 0)$
    \STATE call$(t_n, x) \leftarrow$ payoff$(x)$
    \STATE $p(x) \leftarrow e^{-(x+(r-\sigma^2/2)\Delta t)^2/(2\sigma^2\Delta t)}/(\sigma\sqrt{2\pi\Delta t})$
    \FOR{$i=n$ \textbf{downto} $1$ }
        \STATE $\text{call}(t_i, x) \leftarrow \text{call}(t_i, x)\times 1_{\{x<\log(B/S_0)\}}$
        \STATE $\text{call}_{\text{Diff}}(t_{i}, x) \leftarrow \text{call}(t_{i}, x) - \text{call}_{\text{CV}}(t_{i}, x)$
        \STATE $ \text{call}(t_{i-1}, x) \leftarrow e^{-r\Delta t}( \text{call}_{\text{Diff}}(t_i, x)\otimes p(x) ) + e^{-r\Delta t}( \text{call}_{\text{CV}}(t_i, x)\otimes p(x) )$
    \ENDFOR
    \STATE \textbf{return} $\text{call}(t_0, 0)$
\end{algorithmic}
\end{framed}
\caption{Discrete Up-And-Out Call Pricing Algorithm with Control Variate.
The control variate $\text{call}_{\text{CV}}(t_{i}, x)$ is chosen in a way that $e^{-r\Delta t}( \text{call}_{\text{CV}}(t_i, x)\otimes p(x) )$ is known analytically.
Thus only the first convolution in line 7 needs a numerical approximation.
}
\label{algo:conv_barrier_CV}
\end{figure}



\subsection{Finding a Control Variate for Bermudan Puts Pricing}




\begin{figure}[!t]
\begin{center}
    \parbox[t]{0.45\textwidth}{$\text{put}(t_{i}, x)$ \& $\text{put}_{\text{CV}}(t_{i}, x)$\\}
    \parbox[t]{0.45\textwidth}{$\text{put}_{\text{Diff}}(t_{i}, x)$\\}
    \parbox[t]{0.45\textwidth}{
        \centerline{\epsfig{figure=eps/nsBSCV.eps, width=0.45\textwidth}}
    }
    \parbox[t]{0.45\textwidth}{
        \centerline{\epsfig{figure=eps/nsBSDiff.eps, width=0.45\textwidth}}
    }
\end{center}
\caption{Bermudan Put Pricing with European Put as a Control.
In the left panel are the graphs of the time $t_i$ value of a Bermudan put (solid line) and a European put (dashed line).
The graph of the difference is shown in the right panel.
}
\label{fig:cvBS}
\end{figure}




For Bermudan puts pricing, a control variate of put$(t_i, x)$ is a function $\text{put}_{\text{CV}}(t_i, x)$ that behaves similarly to put$(t_i, x)$ and
$e^{-r\Delta t}(\text{put}_{CV}(t_i, x)\otimes p(x))$ is known in closed form.
A possible choice is the Black-Scholes formula at time $t_i$, although we need to rewrite it in terms of $x = \log(S_{t_i}/S_0)$, that is
\begin{eqnarray*}
\text{put}_{\text{CV}}(t_i, x) &=& K e^{-r\tau_i} N\left( - \frac{\log\frac{S_{t_i}}{K} + \left(r - \frac{\sigma^2}{2}\right)\tau_i}{\sigma\sqrt{\tau_i}} \right)
                                    - S_{t_i} N\left( - \frac{ \log\frac{S_{t_i}}{K} + \left(r + \frac{\sigma^2}{2}\right)\tau_i}{\sigma\sqrt{\tau_i}} \right)\\
                                &=& K e^{-r\tau_i} N\left( - \frac{x + \log\frac{S_0}{K} + \left(r - \frac{\sigma^2}{2}\right)\tau_i}{\sigma\sqrt{\tau_i}} \right)
                                    -S_0e^x N\left( - \frac{x + \log\frac{S_0}{K} + \left(r + \frac{\sigma^2}{2}\right)\tau_i}{\sigma\sqrt{\tau_i}} \right),
\end{eqnarray*}
where $\tau_i = T - t_i$.
By using this control variate, we are approximating a Bermudan puts by European puts.
At first glance it seems not obvious why $e^{-r\Delta t}(\text{put}_{CV}(t_i, x)\otimes p(x))$ is known in closed form,
but due to Theorem \ref{thm:CONV} we know the convolution is equivalent to taking a conditional expectation.
Thus
\begin{eqnarray*}
e^{-r\Delta t}( \text{put}_{\text{CV}}(t_i, x)\otimes p(x) )    &=& e^{-r\Delta t}\tilde{\mathsf E}[\text{European put at $t_i$}|\mathcal F_{t_{i-1}}]\\
                                                                &=& e^{-r\Delta t}\tilde{\mathsf E}[ e^{-r \tau_i}\tilde{\mathsf E}[ (K-S_T)^+ |\mathcal F_{t_{i}}] |\mathcal F_{t_{i-1}}]\\
                                                                &=& e^{-r\tau_{i-1}}\tilde{\mathsf E}[(K-S_T)^+|\mathcal F_{t_{i-1}}],
\end{eqnarray*}
which is a European put at time $t_{i-1}$, so
\begin{eqnarray*}
&&e^{-r\Delta t}(\text{put}_{\text{CV}}(t_i, x)\otimes p(x))\\
&&= K e^{-r\tau_{i-1}} N\left( - \frac{x + \log\frac{S_0}{K} + \left(r - \frac{\sigma^2}{2}\right)\tau_{i-1}}{\sigma\sqrt{\tau_{i-1}}} \right)
-S_0e^x N\left( - \frac{x + \log\frac{S_0}{K} + \left(r + \frac{\sigma^2}{2}\right)\tau_{i-1}}{\sigma\sqrt{\tau_{i-1}}} \right).
\end{eqnarray*}
We just derived the closed form formula of the convolution by writing it as a conditional expectation first.
We will use this trick again later when we derive the closed form formula for the convolution for another control variate.



Numerical results provided later show that with this control, the modified algorithm is already better than the plain convolution algorithm.
However, from Figure \ref{fig:cvBS} we can see that the control variate and the Bermudan put are not very similar.
Below we construct a better control variate, a control variate that makes the difference smaller.
As the numerical results show us, we get even more accuracy with this new control variate.




Recall that at each observation point $t_i$, the option holder gets to decide if he or she wants to exercise the contract.
If $S_{t_i}$ is too low, the contract will be exercised to get $K - S_{t_i}$ immediately.
There is a key value $S_{t_i}^*$, the so-called ``early exercise boundary'',
such that if $S_{t_i}$ drops below this key value then the Bermudan put should be exercised and its value is just the payoff function $K - S_{t_i}$.
The key value in log scale is $x^* = \log(S_{t_i}/S_0)$.
In line 5 of the algorithm in Figure \ref{algo:conv_berm}, put$(t_i, x)$ is set to payoff$(x)$ on the left hand side of $x^*$;
on the right hand side, put$(t_i, x)$ remains unchanged.
We use a piecewise smooth function as the control variate.
First, with the grid functions of put$(t_i, x)$ and payoff$(x)$, we numerically determine the value of $x^*$.
On the left hand side of $x^*$ the control variate is set to the payoff, and hence the difference is zero.
As for the right hand side of $x^*$, from Figure \ref{fig:cvBS} we see that the Bermudan put is a decreasing convex function,
so we use a scaled exponential function $c_1 e^{c_2 x}$ with $c2 < 0$ as the control.
To sum up, the control is
\begin{eqnarray}
(K - S_0 e^x)1_{\{x<x^*\}} + c_1 e^{c_2 x}1_{\{x\geq x^*\}}.\label{eq:CVem}
\end{eqnarray}


We choose $c_1$, $c_2$ such that the control variate is continuous and differentiable.
To achieve the continuity, we must have $c_1 e^{c_2 x^*} = K - S_0e^{x^*}$;
for the control variate to be differentiable at $x^*$, we must have $c_2 c_1 e^{c_2 x^*} = - S_0e^{x^*}$.
This would give us two equations to solve for $c_1$ and $c_2$.
Plain algebra shows that the solution is
$$
\begin{cases}
c_2 = - \frac{S_0e^{x^*}}{K - S_0e^{x^*}}\\
c_1 = (K - S_0e^{x^*}) e^{-c_2 x^*}
\end{cases}.
$$
The graphs of the time $t_i$ value of a Bermudan put, the control variate and the difference are shown in Figure \ref{fig:CVem}.




\begin{figure}[!t]
\begin{center}
    \parbox[t]{0.45\textwidth}{$\text{put}(t_{i}, x)$ \& $\text{put}_{\text{CV}}(t_{i}, x)$\\}
    \parbox[t]{0.45\textwidth}{$\text{put}_{\text{Diff}}(t_{i}, x)$\\}
    \parbox[t]{0.45\textwidth}{
        \centerline{\epsfig{figure=eps/emCV.eps, width=0.45\textwidth}}
    }
    \parbox[t]{0.45\textwidth}{
        \centerline{\epsfig{figure=eps/emDiff.eps, width=0.45\textwidth}}
    }
\end{center}
\caption{Bermudan Put with (\ref{eq:CVem}) as a Control.
In the left panel are the graphs of a Bermudan put (solid line) and the control variate given by (\ref{eq:CVem}) (dashed line).
The graph of the difference is shown in the right panel.
}
\label{fig:CVem}
\end{figure}





To use (\ref{eq:CVem}) as a control variate, we still need the closed form formula for
$$
e^{-r\Delta t}\left(p(x) \otimes \left((K - S_0 e^x)1_{\{x<x^*\}} + c_1 e^{c_2 x}1_{\{x\geq x^*\}}\right)\right).
$$
Since the control variate is piecewise defined, and the ``convolve with $p(x)$'' operator is linear,
the closed form formula can be found term by term, meaning we find
$$
e^{-r\Delta t}\left(p(x) \otimes (K-S_0e^x)1_{\{x<x^*\}}\right) \qquad\text{and}\qquad e^{-r\Delta t}\left(p(x) \otimes c_1e^{c_2x}1_{\{x\geq x^*\}}\right)
$$
separately.
The second convolution can be evaluated directly as follows
\begin{eqnarray*}
p(x) \otimes c_1e^{c_2x}1_{\{x\geq x^*\}} &=&   c_1 \left( p(x) \otimes e^{c_2x}1_{\{x\geq x^*\}} \right)\\
&=&  c_1 \int_{-\infty}^{\infty} \frac{1}{\sigma\sqrt{2\pi\Delta t}} e^{-\frac{1}{2} \left( \frac{y-x+(r-\sigma^2 /2)\Delta t}{\sigma\sqrt{\Delta t}}\right)^2}e^{c_2x}1_{\{x\geq x^*\}} \,dx\\
&=& c_1 \int_{x^*}^{\infty} \frac{1}{\sigma\sqrt{2\pi\Delta t}} e^{-\frac{1}{2} \left( \frac{x-(r-\sigma^2 /2)\Delta t -y-c_2\sigma^2{\Delta t}}{\sigma\sqrt{\Delta t}}\right)^2} e^{-\frac{c_2}{2} \left( -2(r-\sigma^2 /2)\Delta t -2y -c_2\sigma^2{\Delta t}\right)}\,dx\\
&=& c_1 e^{c_2\left(y+(r-\sigma^2 /2)\Delta t +\frac{c_2 \sigma^2{\Delta t}}{2}\right)} N\left(\frac{y+c_2\sigma^2{\Delta t} + (r-\sigma^2 /2)\Delta t-x^*}{\sigma\sqrt{\Delta t}}\right).
\end{eqnarray*}
For the first convolution $e^{-r\Delta t}\left(p(x) \otimes (K-S_0e^x)1_{\{x<x^*\}}\right)$,
by Theorem \ref{thm:CONV} it can be rewritten as the conditional expectation
$$
e^{-r\Delta t} \tilde{\mathsf E}[(K-S_{t_i})1_{\{x<x^*\}}|\mathcal F_{t_{i-1}}].
$$
This is similar to a vanilla put option.
The closed form formula is
\begin{eqnarray*}
&& K e^{-r\Delta t} N\left( - \frac{\log\frac{S_{t_{i-1}}}{S_{t_i}^*} + \left(r - \frac{\sigma^2}{2}\right)\Delta t}{\sigma\sqrt{\Delta t}} \right)
-S_{t_{i-1}}N\left( - \frac{\log\frac{S_{t_{i-1}}}{S_{t_i}^*} + \left(r + \frac{\sigma^2}{2}\right)\Delta t}{\sigma\sqrt{\Delta t}} \right)\\
&&= K e^{-r\Delta t} N\left( - \frac{ x - x^* + \left(r - \frac{\sigma^2}{2}\right)\Delta t}{\sigma\sqrt{\Delta t}} \right)
-S_0e^xN\left( - \frac{ x - x^* + \left(r + \frac{\sigma^2}{2}\right)\Delta t}{\sigma\sqrt{\Delta t}} \right).
\end{eqnarray*}





\subsection{Finding a Control Variate for Discrete Barrier Options Pricing}
\label{sec:findingCVforBarrier}
For discrete up-and-out call options pricing, one possible choice of the control variate is to
use another up-and-out call option that is only allowed to knock out at one observation point,
that is
\begin{eqnarray*}
\text{call}_{\text{CV}}(t_{i}, x) &=& \left[S_0e^x N\left( \frac{x + \log\frac{S_0}{K} + \left(r + \frac{\sigma^2}{2}\right)\tau_i}{\sigma\sqrt{\tau_i}} \right)\right.\\
&&\qquad \left.- K e^{-r\tau_i} N\left( \frac{x + \log\frac{S_0}{K} + \left(r - \frac{\sigma^2}{2}\right)\tau_i}{\sigma\sqrt{\tau_i}} \right)\right]\times 1_{\{x<\log(B/S_0)\}},
\end{eqnarray*}
which is the value at $t_i-$ of an up-and-out call option that is only allowed to knock out at $t_i$.
With Theorem \ref{thm:CONV}, $e^{-r\Delta t}( \text{call}_{\text{CV}}(t_i, x)\otimes p(x) )$ can be rewritten as
\begin{eqnarray*}
e^{-r\Delta t}( \text{call}_{\text{CV}}(t_i, x)\otimes p(x) )    &=& e^{-r\Delta t}\tilde{\mathsf E}\left[(\text{European call at $t_i$})\times 1_{\{x<\log(B/S_0)\}}\big|\mathcal F_{t_{i-1}}\right]\\
                                                                &=& e^{-r\tau_{i-1}}\tilde{\mathsf E}[ \tilde{\mathsf E}\left[ (S_T - K)^+ |\mathcal F_{t_{i}}]\times 1_{\{x<\log(B/S_0)\}} \big|\mathcal F_{t_{i-1}}\right],
\end{eqnarray*}
which is the value of the same option at $t_{i-1}$, also known in closed form.
The derivation of the pricing formula is similar to that of a two-period barrier option pricing formula.



Another way to construct a control variate is to use a scaled normal density function with numerically estimated parameters.
The same technique can also be used to construct the control variate for Asian options pricing, as will be discussed in later sections.
This idea relies on the assumption that the function to be approximated by the control variate has a bell shape,
which is the case for the up-and-out barrier call options pricing algorithm.
Note that in the original convolution pricing algorithm in Figure \ref{algo:conv_barrier},
before the loop call$(t_n, x)$ goes up exponentially fast as $x$ increases.
Line 5 sets the function values to zero for $x>\log(B/S_0)$.
After line 5, the graph of call$(t_n, x)$ would be a triangle that is only nonzero in the interval $(0, \log(B/S_0)]$.
It is then sent to be convolved with a gaussian kernel $p(x)$ in line 6.
The resulting function call$(t_{n-1}, x)$ will be bell-shaped and can be approximated by a scaled normal density function of the form
\begin{eqnarray}
\frac{a_{n-1}}{s_{n-1}\sqrt{2\pi}}e^{-\frac{1}{2}\left(\frac{x-m_{n-1}}{s_{n-1}}\right)^2}.\label{eq:scaledBellShape}
\end{eqnarray}
This approximation has three parameters $a$, $m$ and $s$,
representing the area under the curve, the mean and the standard deviation of the normal density, respectively.
We estimate the parameters by numerically integrating the grid functions.
For example, as an approximation of the area under call$(t_{n-1}, x)$,
$a_{n-1}$ can be found by numerically integrating call$(t_{n-1}, x)$.
Once the value of $a_{n-1}$ is determined, the normalized bell-shaped curve $\text{call}(t_{n-1}, x)/a_{n-1}$ has area 1 and therefore is a probability density function,
the first and the second raw moments of which can be found by numerically integrating $x\cdot\text{call}(t_{n-1}, x)/a_{n-1}$ and $x^2\cdot\text{call}(t_{n-1}, x)/a_{n-1}$, respectively.
We get $m_{n-1}$ directly from the first moment.
To get an estimate of $s_{n-1}$, one just need to switch from the raw moment to the central moment.




Having found the parameters of (\ref{eq:scaledBellShape}),
we have a closed form approximation of the $\text{call}(t_{n-1}, x)$ function in the end of the first iteration of the loop.
The following function can be used as a control variate in the second iteration of the loop
\begin{eqnarray*}
\text{call}_{\text{CV}}(t_{n-1}, x) = \left(\frac{a_{n-1}}{s_{n-1}\sqrt{2\pi}}e^{-\frac{1}{2}\left(\frac{x-m_{n-1}}{s_{n-1}}\right)^2}\right)\times 1_{\{x<\log(B/S_0)\}}.
\end{eqnarray*}
In general, a control variate of this form can be used for all indices $i \leq n-1$, that is
$$
\text{call}_{\text{CV}}(t_{i}, x) = \left(\frac{a_{i}}{s_{i}\sqrt{2\pi}}e^{-\frac{1}{2}\left(\frac{x-m_{i}}{s_{i}}\right)^2}\right)\times 1_{\{x<\log(B/S_0)\}}\qquad\forall i = n-1, n-2, \ldots, 1,
$$
because after the convolution step in line 6 of Figure \ref{algo:conv_barrier} we always get a bell-shaped curve.
The parameters $a_i$, $m_i$ and $s_i$ can be estimated by numerically integrating the bell-shaped function, as described above.
In the first iteration with index $i=n$, however, we can not use control variate of this from
since no convolution step has been executed and therefore no bell-shaped function can be approximated.
The closed form formula of $e^{-r\Delta t}(\text{call}_{\text{CV}}(t_{i}, x)\otimes p(x))$ can be evaluated by the following lemma.
The derivation is straightforward.



\begin{lemma}
Let $f$ and $g$ be the probability density function of the normal distributions $N(\mu_1, \sigma_1)$ and $N(\mu_2, \sigma_2)$, respectively.
Then for constant $L$,
$$
f \otimes \left(g\cdot 1_{\{x>L\}}\right) = \frac{1}{\sqrt{2\pi(\sigma_1^2 + \sigma_2^2)}} e^{-\frac12\frac{(x - m_1 - m_2)^2}{\sigma_1^2 + \sigma_2^2}}
\left[1 - N\left(\frac{(L-m_2)\sigma_1^2 + (L-x+m_1)\sigma_2^2 }{\sigma_1\sigma_2\sqrt{\sigma_1^2 + \sigma_2^2}}\right)\right].
$$
\end{lemma}




\subsection{Convolution Pricing Algorithms for Asian Options}



A classic convolution-based algorithm is the Asian options pricing algorithm proposed by \cite{carverhill1990flexible}.
In the following sections we introduce this algorithm and its variant, and the modification with the control variate technique.




An Asian option is an option on the average value of the underlying asset prices through out the life of this option.
As an example, consider an Asian call option with maturity $T$ and strike price $K$.
As usual, we assume that the underlying asset price at time $t$ is $S_t$.
For an $n$-period Asian option, the underlying asset price is observed $n$ times through out the life of this option.
Say it is observed at time points $t_1 < t_2 < \ldots < t_n = T$.
Together with the initial underlying asset price $S_{t_0}$, the average value of the $n+1$ prices are evaluated and the payoff of this option is
\begin{eqnarray*}
\left(\frac{1}{n+1} \sum_{j=0}^n S_{t_{j}} - K\right)^+. %\label{eq:payoffasian}
\end{eqnarray*}
Therefore, by (\ref{eq:universal_pricing_formula}), its no-arbitrage price is
\begin{eqnarray*}
e^{-rT}\tilde{\mathsf E}\left[\left(\frac{1}{n+1} \sum_{j=0}^n S_{t_{j}} - K\right)^+\right].\label{eq:AsianOptionValue}
\end{eqnarray*}
For simplicity, we assume that the observation points are equally distant,
so there is a constant time interval $\Delta t = t_i - t_{i-1}$ for all $i=1, 2, \ldots, n$.
In this setting, the underlying asset price is monitored discretely, and hence the option is called a discrete Asian option.
There are also continuously monitored Asian options with payoff $\left(\frac{1}{T}\int_0^T S_t\,dt - K\right)^+$,
whose pricing problem is discussed intensively in the literature.
In this thesis, we only focus on discrete Asian options pricing.
In principle it is not possible to derive a closed form pricing formula for Asian options,
not even under the most analytically tractable Black-Scholes model.
Some numerical scheme needs to be involved for the pricing problem.




The main idea of \cite{carverhill1990flexible} is to numerically approximate the probability density function of the average value.
Unlike previously discussed convolution pricing algorithms which evaluate the option values backward in time,
Carverhill and Clewlow's algorithm goes forward.
Given the grid function of the probability density of $\log(S_{t_0} + S_{t_1} + \cdots + S_{t_k})$,
the algorithm evaluates the density function of $\log(S_{t_0} + S_{t_1} + \cdots + S_{t_{k}} + S_{t_{k+1}})$.
The algorithm goes recursively until the $\log(S_{t_0} + S_{t_1} + \cdots + S_{T})$ is found.
Once this is done, using numerical integration to find the fair price (\ref{eq:AsianOptionValue}) is straightforward.




To understand Carverhill and Clewlow's algorithm, let us introduce some notations first.
Denote $X_i\equiv S_{t_i}/S_{{t_{i-1}}}$ the return of the underlying asset price, and $R_i = \log X_i$ the log return.
Under the Black-Scholes model, $\{X_i\}$ is a sequence of iid log-normal random variables, and $\{R_i\}$ a sequence of iid normal.
With these notations, the underlying asset price at any observation point can be written as the product of a sequence of $X$'s as follows
$$
S_{t_j} = S_{t_0} \frac{S_{t_1}}{S_{t_0}} \frac{S_{t_2}}{S_{t_1}} \cdots \frac{S_{t_j}}{S_{t_{j-1}}} = S_{t_0}X_1X_2\cdots X_j.
$$
The starting point of Carverhill and Clewlow's algorithm is the following decomposition of the average price $A$.
\begin{eqnarray}
   A &=& \frac{1}{n+1}\sum\limits^n_{i=0} S_{t_j} = \frac{1}{n+1} \left(S_{t_0} + S_{t_1} + S_{t_2} + \cdots + S_{t_n}\right)\label{eq:A_sum}\\
     &=& \frac{S_{t_0}}{n+1}\left(1+X_1+X_1X_2+\cdots+X_1X_2\cdots X_n\right)\nonumber\\
     &=& \frac{S_{t_0}}{n+1}\left(1+X_1\left(1+X_2\left(\cdots X_{n-1}\left(1+X_n\right)\right)\right)\right).\nonumber
%     &=& \frac{S_{t_0}}{n+1}\left(1+e^{R_1+\ln(1+\exp(\cdots+\ln(1+\exp(R_n))))}\right) = \frac{S_{t_0}}{n+1}\left(1+e^{B_0}\right),
\end{eqnarray}
Define
$$
B_j = \log(X_{j+1}(1+X_{j+2}(\ldots X_{n-2}(1+X_{n-1}(1+ X_n))))).
$$
Then we can write
\begin{eqnarray}
A = \frac{S_{t_0}}{n+1}\left(1+e^{B_0}\right).\label{eq:AB0}
\end{eqnarray}
It is not hard to verify the following recurrence relation
\begin{eqnarray}
B_{j-1} = R_{j} + \log ( 1 + \exp (B_{j})) \qquad j=n-1, n-2, \ldots , 1\label{eq:recurrenceCarverhill}
\end{eqnarray}
with the ``initial value'' $B_{n-1} = R_n$.
For simplicity, we denote $Y_{j} = \log ( 1 + \exp (B_{j}))$, or equivalently,
$$
Y_j = \log(1 + X_{j+1}(1+X_{j+2}(\ldots X_{n-2}(1+X_{n-1}(1+ X_n))))).
$$
Thus the recurrence relation (\ref{eq:recurrenceCarverhill}) can also be written as
$$
B_{j-1} = R_{j} + Y_j \qquad j=n-1, n-2, \ldots , 1.
$$
Our ultimate goal is to find the density function of $A$ so that we can apply numerical integration to find $e^{-rT}\tilde{\mathsf E}[(A-K)^+]$.
With the recurrence relation (\ref{eq:recurrenceCarverhill}) it is more clear how to obtain the density.
One can start from $B_{n-1} = R_n$, which is known to be normally distributed,
and then apply (\ref{eq:recurrenceCarverhill}) to find the density function of $B_{n-2}$.
With the density function of $B_{n-2}$, (\ref{eq:recurrenceCarverhill}) can be applied again to find the density function of $B_{n-3}$.
The process goes on until the density function of $B_0$ is found, from which only a change of variable is needed to obtain the density of $A$.




Closed form density functions are not possible to obtain.
All the density functions of $B_j$'s need to be approximated numerically by grid functions.
This is similar to the previously discussed convolution pricing algorithms.
First set up a wide interval $[-a, a]$ and a uniform grid $\{x_i\}$, where $-a = x_0 < x_1 < x_2\cdots < x_m = a$.
For any function $f$, the algorithm only keeps its values on the grid, that is $\{f(x_0), f(x_1), f(x_2), \ldots, f(x_m)\}$.
Given the grid function of the density of $B_j$, to obtain the grid function of the density of $Y_j = \log ( 1 + \exp (B_{j}))$
one can apply interpolation to evaluate
\begin{equation}
f_{Y_{j}}(x) =
\begin{cases}
f_{B_{j}}(\ln (e^x-1))\frac{e^x}{e^x-1}&\mbox{if }x>0\\
0 &\mbox{otherwise}
\end{cases}\label{eq:ChangeDen}
\end{equation}
at each grid point.
Since $R_{j}$ and $Y_j = \log ( 1 + \exp (B_{j}))$ is independent,
$B_{j-1} = R_{j} + Y_j$ is the sum of two independent random variables.
Its density function is the convolution of the density functions of $R_j$ and $Y_j$.
To find the grid function, one needs to use the discrete convolution algorithm.
To sum up, there are two steps to apply the recurrence relation (\ref{eq:recurrenceCarverhill})
to find the grid function of the density of $B_{j-1}$ from that of $B_{j}$.
First apply interpolation to evaluate the grid function of the density of $Y_j$ by (\ref{eq:ChangeDen}).
Then use the discrete convolution algorithm to evaluate the grid function of the density of $B_{j-1}$ given the density functions of $R_j$ and $Y_j$.
Carverhill and Clewlow's algorithm is shown in Figure \ref{algo:conv_asian}.




\begin{figure}[!t]
\begin{framed}
\begin{algorithmic}[1]
    \STATE $f_R \leftarrow \text{grid function of } e^{-(x-(r-\sigma^2/2)\Delta t)^2/(2\sigma^2\Delta t)}/(\sigma\sqrt{2\pi\Delta t})$
    \STATE $f_{B_{n-1}} \leftarrow f_R$
    \FOR{$j=n-1$ \textbf{downto} $1$ }
        \STATE Given $f_{B_{j}}$, evaluate $f_{Y_j}$ by (\ref{eq:ChangeDen}) and interpolation
        \STATE $f_{B_{j-1}} \leftarrow f_R \otimes f_{Y_j}$
    \ENDFOR
    \STATE Given $f_{B_0}$, evaluate $f_A$ by (\ref{eq:AB0}) and interpolation
    \STATE \textbf{return} $e^{-rT}\tilde{\mathsf E}[(A-K)^+]$
\end{algorithmic}
\end{framed}
\caption{Carverhill and Clewlow's Algorithm for Asian Options Pricing
} \label{algo:conv_asian}
\end{figure}




\citet{benhamou2002fast} argues that Carverhill and Clewlow's algorithm is inefficient and can be improved.
In Carverhill and Clewlow's algorithm, a very wide interval $[-a, a]$ is needed
because all the random variables $Y_j$'s and $B_j$'s considered in the algorithm are suppose to have most of their probability mass stay in the interval.
Indeed, by taking the grid function $\{f_{B_j}(x_0), f_{B_j}(x_1), f_{B_j}(x_2), \ldots, f_{B_j}(x_m)\}$ with $x_0 = -a$ and $x_m = a$,
the probability mass of $B_j$ out of the interval $[-a, a]$ is meeting in the algorithm.
We truncate the part of $f_{B_j}$ out of $[-a, a]$.
Hence $[-a, a]$ should be big enough so that the truncation error can be ignored.
But from the identities
\begin{eqnarray*}
B_j &=& \log(X_{j+1}(1+X_{j+2}(\ldots X_{n-2}(1+X_{n-1}(1+ X_n))))), \\
Y_j &=& \log(1 + X_{j+1}(1+X_{j+2}(\ldots X_{n-2}(1+X_{n-1}(1+ X_n))))).
\end{eqnarray*}
we can see that, since the log-normal random variable $X_j$ is positive,
when $j$ goes smaller and smaller, the probability mass of $B_j$ and $Y_j$ shifts to the right.
As a result, an interval $[-a, a]$ that contains most of the probability mass of all $Y_j$'s and $B_j$'s would be very big.
While the interval $[-a, a]$ is big, to obtain accurate pricing results we still need a fine grid.
Thus a huge number of grid points is needed, making Carverhill and Clewlow's algorithm inefficient.




The idea of \citet{benhamou2002fast} is simple.
First estimate the mean value of $B_j$ and shift the density by this amount.
Denote by $\mu_j$ the estimated mean value and denote $D_j = B_j - \mu_j$.
The mean value of such constructed $D_j$ will be close to zero, thus the majority of the probability mass is in the neighborhood of 0.
Instead of the grid function of the density of $B_j$,
modify the algorithm and write everything in terms of the grid function of the density of $D_j$.
This way the interval that needs to contain the majority of the probability mass of the grid function considered in the algorithm will be greatly reduced.
\citet{benhamou2002fast} uses the following estimated mean values of $B_j$'s
\begin{eqnarray}
     \mu_{n-1}   &=& \tilde{\mathsf E}[R_n]\label{eq:mu_init}\\
     \mu_{j-1}   &=& \tilde{\mathsf E}[R_{j}]+\log (1+\exp (\mu_{j}))\qquad j=n-1, n-2, \ldots , 1.\label{eq:mu_rec}
\end{eqnarray}
Recall that the distribution of $R_j$'s are known (to be iid normal), so $\mu_{n-1}$ is known in closed form.
All $\mu_j$'s can be found recursively.
Since $\log(1+e^x)$ is a convex function, by Jensen's inequality we have
\begin{eqnarray*}
\tilde{\mathsf E}[B_{j-1}] &=& \tilde{\mathsf E}[R_{i}] + \tilde{\mathsf E}[\log(1+\exp (B_j))]\\
&\geq& \tilde{\mathsf E}[R_{i}] + \log(1+\exp (\tilde{\mathsf E}[B_j])).
\end{eqnarray*}
Thus $\mu_j$ always underestimates the mean value of $B_j$.
But it does not matter much.
$\mu_j$ is still a good enough estimate that can shift the majority of the probability mass of $B_j$ to the neighborhood of 0.




Replacing all $B_j$ by $D_j + \mu_j$, the recurrence relation (\ref{eq:recurrenceCarverhill}) can be modified as
$$
D_{j-1} = R_{j} + \ln(1+\exp(D_{j}+\mu_{j})) - \mu_{j-1}\qquad j=n-1, n-2, \ldots , 1
$$
with the ``initial value'' $D_{n-1} = R_n - \mu_{n-1}$.
For simplicity we denote $Z_{j} = Y_{j} - \mu_{j-1} = \ln(1+\exp(D_{j}+\mu_{j})) - \mu_{j-1}$, whose density function in terms of the density function of $D_j$ is
\begin{equation}
f_{Z_{j}}(x) =
\begin{cases}
f_{D_{j}}\left(\ln (e^{x+\mu_{j-1}}-1) - \mu_j\right)\frac{e^{x+\mu_{j-1}}}{e^{x+\mu_{j-1}}-1}&\mbox{ if }x>0\\
0 &\mbox{ otherwise }
\end{cases}.                \label{eq:D2Zdensity}
\end{equation}
This is an analogue of (\ref{eq:ChangeDen}) that can be used with interpolation to find the grid function of $f_{Z_{j}}$ given the grid function of $f_{D_{j}}$.
Benhamou's algorithm is shown in Figure \ref{algo:conv_asian_benh}.





\begin{figure}[!t]
\begin{framed}
\begin{algorithmic}[1]
    \STATE Evaluate $\{\mu_j\}$ by (\ref{eq:mu_init}) and (\ref{eq:mu_rec})
    \STATE $f_R \leftarrow \text{grid function of } e^{-(x-(r-\sigma^2/2)\Delta t)^2/(2\sigma^2\Delta t)}/(\sigma\sqrt{2\pi\Delta t})$
    \STATE $f_{D_{n-1}} \leftarrow \text{grid function of } e^{-x^2/(2\sigma^2\Delta t)}/(\sigma\sqrt{2\pi\Delta t})$
    \FOR{$j=n-1$ \textbf{downto} $1$ }
        \STATE Given $f_{D_{j}}$, evaluate $f_{Z_j}$ by (\ref{eq:D2Zdensity}) and interpolation
        \STATE $f_{D_{j-1}} \leftarrow f_R \otimes f_{Z_j}$
    \ENDFOR
    \STATE Given $f_{D_0}$, evaluate $f_A$ by (\ref{eq:AB0}) and interpolation
    \STATE \textbf{return} $e^{-rT}\tilde{\mathsf E}[(A-K)^+]$
\end{algorithmic}
\end{framed}
\caption{Benhamou's Algorithm for Asian Options Pricing
} \label{algo:conv_asian_benh}
\end{figure}



\subsection{Modified Algorithms}


Benhamou's algorithm in Figure \ref{algo:conv_asian_benh} has a similar structure as previously discussed convolution pricing algorithms.
Thus it can be modified by the control variate technique in the same methodology as before.
Take the Bermudan options pricing algorithm in Figure \ref{algo:conv_berm} as an example.
In both algorithms in Figures \ref{algo:conv_asian_benh} and \ref{algo:conv_berm},
line 6 is the convolution step which is linear and can be modified by the control variate technique.
Specifically, line 6 in Benhamou's algorithm in Figure \ref{algo:conv_asian_benh} can be viewed as the ``convolve with $f_R$'' operator acting on $f_{Z_j}$.
Thus we can modify the convolution step like we did in the modified Bermudan options pricing algorithm in Figure \ref{algo:conv_berm_CV}.
First we find a control variate, a function $f_{Z_{j}}^{\text{CV}}$ that satisfies the two conditions:
\begin{enumerate}
  \item $f_{Z_{j}}^{\text{CV}}$ behaves similarly to $f_{Z_{j}}$ and
  \item both $f_{Z_{j}}^{\text{CV}}$ and $f_{Z_{j}}^{\text{CV}}\otimes f_{R_{j}}$ are known in closed form.
\end{enumerate}
We then evaluate the grid function of the difference $f_{Z_{j}}^{\text{Diff}} = f_{Z_{j}} - f_{Z_{j}}^{\text{CV}}$
and replace line 6 in Benhamou's algorithm by the decomposition
\begin{eqnarray*}
f_{D_{j-1}} = f_{R_{j}} \otimes f_{Z_{j}}^{\text{CV}} + f_{R_{j}} \otimes f_{Z_{j}}^{\text{Diff}}.
\end{eqnarray*}
The resulting algorithm is shown in Figure \ref{algo:conv_asian_benh_CV_1}.
This is very similar to the modified Bermudan options pricing algorithm in Figure \ref{algo:conv_berm_CV}.
Now we have two convolutions $f_{R_{j}} \otimes f_{Z_{j}}^{\text{CV}}$ and $f_{R_{j}} \otimes f_{Z_{j}}^{\text{Diff}}$.
The first one is known in closed form; it does not require any numerical approximation and hence gives no numerical error.
The second convolution needs to be approximated by discrete convolution algorithm with grid functions,
but the numerical error will be tiny since $f_{Z_{j}}^{\text{Diff}}$ only takes small values.
All these benefits are from the two conditions the control variate $f_{Z_{j}}^{\text{CV}}$ should satisfy.



\begin{figure}[!t]
\begin{framed}
\begin{algorithmic}[1]
    \STATE Evaluate $\{\mu_j\}$ by (\ref{eq:mu_init}) and (\ref{eq:mu_rec})
    \STATE $f_R \leftarrow \text{grid function of } e^{-(x-(r-\sigma^2/2)\Delta t)^2/(2\sigma^2\Delta t)}/(\sigma\sqrt{2\pi\Delta t})$
    \STATE $f_{D_{n-1}} \leftarrow \text{grid function of } e^{-x^2/(2\sigma^2\Delta t)}/(\sigma\sqrt{2\pi\Delta t})$
    \FOR{$j=n-1$ \textbf{downto} $1$ }
        \STATE Given $f_{D_{j}}$, evaluate $f_{Z_j}$ by (\ref{eq:D2Zdensity}) and interpolation
        \STATE $f_{Z_{j}}^{\text{Diff}} \leftarrow f_{Z_{j}} - f_{Z_{j}}^{\text{CV}}$
        \STATE $f_{D_{j-1}} \leftarrow f_{R_{j}} \otimes f_{Z_{j}}^{\text{CV}} + f_{R_{j}} \otimes f_{Z_{j}}^{\text{Diff}}$
    \ENDFOR
    \STATE Given $f_{D_0}$, evaluate $f_A$ by (\ref{eq:AB0}) and interpolation
    \STATE \textbf{return} $e^{-rT}\tilde{\mathsf E}[(A-K)^+]$
\end{algorithmic}
\end{framed}
\caption{Benhamou's Algorithm Enhanced by the Control Variate Technique
} \label{algo:conv_asian_benh_CV_1}
\end{figure}



Unlike previously discussed convolution pricing algorithms, however,
in Benhamou's algorithm the convolution is not the only step that causes numerical error.
The interpolation in line 5 and 8 in Figure \ref{algo:conv_asian_benh} also does.
Fortunately this part can be enhanced by the control variate technique as well.
Given the grid function of $f_{D_j}$, line 5 evaluates the grid function of $f_{Z_j}$ by interpolation and (\ref{eq:D2Zdensity}), which we repeat here.
\begin{equation}
f_{Z_{j}}(x) =
\begin{cases}
f_{D_{j}}\left(\ln (e^{x+\mu_{j-1}}-1) - \mu_j\right)\frac{e^{x+\mu_{j-1}}}{e^{x+\mu_{j-1}}-1}&\mbox{ if }x>0\\
0 &\mbox{ otherwise }
\end{cases}.                \label{eq:D2ZdensityAgain}
\end{equation}
To modify this step, we first find a control variate of $f_{D_j}$, a closed form function $f_{D_j}^{\text{CV}}$ that behaves similarly to $f_{D_j}$.
We then find the grid function of the difference $f_{D_j}^{\text{Diff}} = f_{D_j} - f_{D_j}^{\text{CV}}$ and evaluate $f_{Z_j}$  by
\begin{eqnarray}
f_{Z_{j}}(x) &=& \left( f_{D_{j}}^{\text{CV}}\left(\ln (e^{x+\mu_{j-1}}-1) - \mu_j\right)
+ f_{D_{j}}^{\text{Diff}}\left(\ln (e^{x+\mu_{j-1}}-1) - \mu_j\right) \right) \frac{e^{x+\mu_{j-1}}}{e^{x+\mu_{j-1}}-1}\qquad\qquad \label{eq:IntpDecomp}
\end{eqnarray}
instead of (\ref{eq:D2ZdensityAgain}).
This is only for positive $x$.
For negative $x$ or zero, we still set $f_{Z_j}(x)$ to zero.
On the right hand side of (\ref{eq:IntpDecomp}), Since the control variate $f_{D_{j}}^{\text{CV}}$ is known in closed form,
the first term in the parentheses at any grid point $x_i$ can be found exactly.
Only the second term requires an interpolation and will introduce numerical error.
But the error will be tiny, as long as the control variate $f_{D_{j}}^{\text{CV}}$ behaves similarly to $f_{D_{j}}$
so that the difference $f_{D_{j}}^{\text{Diff}}$ only takes small values.
In contrast, in the original Benhamou algorithm which uses interpolation directly with (\ref{eq:D2ZdensityAgain}),
since the variance of $D_j$ can be small, $f_{D_j}$ can have a very tall and narrow shape.
Applying interpolation on this kind of function can yield significant numerical error.
That is why when the interpolation step in Benhamou's algorithm is also modified by the control variate technique, the pricing error would be greatly improved.
The modified algorithm is shown in Figure \ref{algo:conv_asian_benh_CV_2}.




\begin{figure}[!t]
\begin{framed}
\begin{algorithmic}[1]
    \STATE Evaluate $\{\mu_j\}$ by (\ref{eq:mu_init}) and (\ref{eq:mu_rec})
    \STATE $f_R \leftarrow \text{grid function of } e^{-(x-(r-\sigma^2/2)\Delta t)^2/(2\sigma^2\Delta t)}/(\sigma\sqrt{2\pi\Delta t})$
    \STATE $f_{D_{n-1}} \leftarrow \text{grid function of } e^{-x^2/(2\sigma^2\Delta t)}/(\sigma\sqrt{2\pi\Delta t})$
    \FOR{$j=n-1$ \textbf{downto} $1$ }
        \STATE $f_{D_{j}}^{\text{Diff}} \leftarrow f_{D_{j}} - f_{D_{j}}^{\text{CV}}$
        \STATE Given $f_{D_{j}}^{\text{Diff}}$, evaluate $f_{Z_j}$ by (\ref{eq:IntpDecomp}) and interpolation
        \STATE $f_{Z_{j}}^{\text{Diff}} \leftarrow f_{Z_{j}} - f_{Z_{j}}^{\text{CV}}$
        \STATE $f_{D_{j-1}} \leftarrow f_{R_{j}} \otimes f_{Z_{j}}^{\text{CV}} + f_{R_{j}} \otimes f_{Z_{j}}^{\text{Diff}}$
    \ENDFOR
    \STATE Given $f_{D_0}$, evaluate $f_A$ by (\ref{eq:AB0}) and interpolation
    \STATE \textbf{return} $e^{-rT}\tilde{\mathsf E}[(A-K)^+]$
\end{algorithmic}
\end{framed}
\caption{Another Modification of Benhamou's Algorithm by the Control Variate Technique
} \label{algo:conv_asian_benh_CV_2}
\end{figure}



\subsection{Finding a Control Variate for Asian Options Pricing}
In the modified Benhamou algorithm in Figure \ref{algo:conv_asian_benh_CV_2},
we need to find the control variates of $f_{D_j}$ and $f_{Z_j}$.
As $f_{D_j}$ and $f_{Z_j}$ are both probability density functions,
one possible choice is to use normal density functions with parameters estimated numerically,
like we did in discrete barrier options pricing.
Of course, just because $f_{D_j}$ and $f_{Z_j}$ are probability density functions does not mean they can be well-approximated by normal density functions,
but this choice can actually be justified.
Many literatures, including \citet{hamdan1971logarithm}, \citet{naus1969distribution} and \citet{schwartz1982distribution}
suggest that a sum of correlated log-normal random variables like (\ref{eq:A_sum}) can be well-approximated by a log-normal distribution.
By approximating the average price $A$ by a log-normal distribution, \citet{levy1992pricing} derives a closed form approximation formula for Asian option prices.
Recall that each $e^{B_j}$ or $e^{Y_j}$ has the same distribution as a partial sum of $A$,
and $D_j$ and $Z_j$ are just shifted $B_j$ and $Y_j$.
The fact that sum of correlated log-normal random variables can be well-approximated by a log-normal distribution implies that
both $D_j$ and $Z_j$ can be well-approximated by normal distributions.




So we use normal density functions as the control variates.
We still need to estimate the parameters of the density function.
Given the grid function of $f_{Z_j}$, first numerically integrate $x f_{Z_j}$ to get an estimate of the mean value.
Say the estimate is $m_j$.
Then numerically integrate $(x-m_j)^2 f_{Z_j}$ to find an estimate of the variance and therefore the standard deviation.
Denote the estimated standard deviation by $s_j$.
We use
\begin{eqnarray}
f_{Z_j}^{\text{CV}}(x) = \frac{1}{s_j\sqrt{2\pi}}e^{-\frac{1}{2}\left(\frac{x-m_j}{s_j}\right)^2}\label{eq:fZCV}
\end{eqnarray}
as the control variate of $f_{Z_j}$.
The same idea works to find a control variate of $f_{D_j}$.




For $f_{D_j}$, we only need the control variate to be a closed form function
so that we can evaluate the first term of the right hand side of (\ref{eq:IntpDecomp}) without interpolation.
For $f_{Z_j}$, however, we need a control variate $f_{Z_j}^{\text{CV}}$ such that $f_{R_{j}} \otimes f_{Z_{j}}^{\text{CV}}$ is known in closed form.
We need the closed form because we do not want to use any numerical approximation
to find the first convolution in line 8 of the modified Benhamou algorithm in Figure \ref{algo:conv_asian_benh_CV_2}.
Since $R_j$ is normally distributed, with a control variate of the form (\ref{eq:fZCV}),
$f_{R_{j}} \otimes f_{Z_{j}}^{\text{CV}}$ is simply another normal density function and is obviously known in closed form.




Using numerical integration to estimate the parameters of the control variate requires $O(m)$ computational time, $m$ the number of grid points.
This is not a great cost in comparison with the discrete convolution step which takes $O(m\log m)$ computations if the FFT algorithm is used.
But in fact it is a unnecessary cost,
because accurate estimates of both the mean and the standard deviation of $D_j$ and $Z_j$ can be derived more explicitly, as shown below.
First note that, by the definition $D_j = B_j - \mu_j$, $Z_j = Y_j - \mu_j$ we have
\begin{eqnarray*}
&&\tilde{\mathsf E}[D_j] = \tilde{\mathsf E}[B_j] - \mu_j,\qquad \widetilde{\mathsf{Var}}[D_j] = \widetilde{\mathsf{Var}}[B_j],\\
&&\tilde{\mathsf E}[Z_j] = \tilde{\mathsf E}[Y_j] - \mu_j, \qquad \widetilde{\mathsf{Var}}[Z_j] = \widetilde{\mathsf{Var}}[Y_j].
\end{eqnarray*}
Thus one way of finding the mean and the standard deviation of $D_j$ and $Z_j$ is to find the moments of $B_j$ and $Y_j$ first.
Recall that we have the identities
\begin{eqnarray*}
B_j &=& \log(X_{j+1}(1+X_{j+2}(\ldots X_{n-2}(1+X_{n-1}(1+ X_n))))), \\
Y_j &=& \log(1 + X_{j+1}(1+X_{j+2}(\ldots X_{n-2}(1+X_{n-1}(1+ X_n))))),
\end{eqnarray*}
or equivalently
\begin{eqnarray*}
e^{B_j} &=& X_{j+1}(1+X_{j+2}(\ldots X_{n-2}(1+X_{n-1}(1+ X_n)))), \\
e^{Y_j} &=& 1 + X_{j+1}(1+X_{j+2}(\ldots X_{n-2}(1+X_{n-1}(1+ X_n)))).
\end{eqnarray*}
By definition $X_i = S_{t_i}/S_{t_{i-1}}$ are iid log-normal random variables
and the associated normal distribution has mean $(r-\sigma^2/2)\Delta t$ and the standard deviation $\sigma\sqrt{\Delta t}$.
The first two raw moments of $X_i$ can easily be found as
$$
\tilde{\mathsf E}[X_i] = e^{r\Delta t}, \qquad \tilde{\mathsf E}[X_i^2] = e^{(2r + \sigma^2)\Delta t}.
$$
To find the general formulas of the first two moments of $e^{B_j}$ and $e^{Y_j}$,
first we introduce two lemmas.
\begin{lemma}
\label{lemma:mmul}
Let $e_k$ be the $k$-th moment of $X$,
    $m_{k}$ the $k$-th moment of $U$,
and $m_{k}^\prime$ the $k$-th moment of $XU$.
If $X$ is independent of $U$, then $m_k^\prime = e_k m_k$.
In matrix form, we can write
$$
\begin{pmatrix}
m_0^\prime\\ m_1^\prime\\ m_2^\prime
\end{pmatrix}
=
\begin{pmatrix}
e_0 & 0   & 0 \\
0   & e_1 & 0 \\
0   & 0   & e_2
\end{pmatrix}
\begin{pmatrix}
m_0\\
m_1\\
m_2
\end{pmatrix}.
$$
\end{lemma}
\begin{lemma}
\label{lemma:madd}
Let $m_{k}$ be the $k$-th moment of $U$ and $m_{k}^\prime$ the $k$-th moment of $U+1$.
Then
$$
\begin{pmatrix}
m_0^\prime\\ m_1^\prime\\ m_2^\prime
\end{pmatrix}
=
\begin{pmatrix}
1 & 0& 0 \\
1 & 1& 0 \\
1 & 2& 1
\end{pmatrix}
\begin{pmatrix}
m_0\\ m_1\\ m_2
\end{pmatrix}.
$$
\end{lemma}
Both lemmas can be proved by straightforward computation.


The first two moments of $e^{B_j}$ and $e^{Y_j}$ can be evaluated as follows.
First we put the first two raw moments (starting from zero) of $e^{B_{n-1}} = X_n$ together as a vector.
$$
v =
\begin{pmatrix}
1\\ e^{r\Delta t}\\ e^{(2r+\sigma^2)\Delta t}
\end{pmatrix}.
$$
Then introduce two matrices
$$
C = \begin{pmatrix}
1 & 0& 0 \\
1 & 1& 0 \\
1 & 2& 1
\end{pmatrix}, \qquad
M = \begin{pmatrix}
1 & 0   & 0 \\
0   & e^{r\Delta t} & 0 \\
0   & 0   & e^{(2r+\sigma^2)\Delta t}
\end{pmatrix}.
$$
By Lemma \ref{lemma:madd}, the first two raw moments of $e^{Y_{n-1}} = 1 + X_n$ is given by the vector $Cv$.
Given the moments of $e^{Y_{n-1}}$, applying Lemma \ref{lemma:mmul} one can find that the moments of $e^{B_{n-2}} = X_{n-1}(1 + X_n)$ is given by the vector $MCv$.
Then apply Lemma \ref{lemma:madd} again to find the moments of $e^{Y_{n-2}} = 1 + X_{n-1}(1 + X_n)$ as $CMCv$.
Thus, by applying Lemmas \ref{lemma:mmul} and \ref{lemma:madd} alternately,
we can conclude
\begin{eqnarray*}
&&\text{the first two raw moments of }e^{B_j}\text{ is given by the vector }(MC)^{n-j-1}v, \text{ and }\\
&&\text{the first two raw moments of }e^{Y_j}\text{ is given by the vector }(CM)^{n-j-1}Cv.
\end{eqnarray*}
Through matrix diagonalization, it is possible to derive the explicit formula of the matrix power $(MC)^k$ and $(CM)^k$.
That way we can obtain lengthy closed form formulas of the first two raw moments of $e^{B_j}$ and $e^{Y_j}$.
Nevertheless, when implementing the algorithm a more efficient way to compute the moments of $e^{B_j}$ and $e^{Y_j}$
is to perform the matrix multiplication in every iteration in the loop,
as each matrix multiplication by $C$ or $M$ only takes a few (less than 5) floating point number's multiplication or addition operations,
while the explicit formulas of the moments involve the power function which is a much more expensive operation.



Given the moments of $e^{B_j}$ and $e^{Y_j}$, we still want to find estimates of the moments of ${B_j}$ and ${Y_j}$.
As aforementioned, ${B_j}$ and ${Y_j}$ can be well-approximated by normal distributions and $e^{B_j}$ and $e^{Y_j}$ can be well-approximated by log-normal.
The mean and the variance formulas of a log-normal distribution in terms of the mean and the variance of its associated normal distribution give us
\begin{eqnarray*}
\tilde{\mathsf E}(e^{B_j}) &\approx& e^{\tilde{\mathsf E}(B_j)+\widetilde{\mathsf{Var}}(B_j)/2},\\
\widetilde{\mathsf{Var}}(e^{B_j}) &\approx& \left(e^{\widetilde{\mathsf{Var}}(B_j)}-1\right)e^{2\tilde{\mathsf E}(B_j)+\widetilde{\mathsf{Var}}(B_j)},
\end{eqnarray*}
which can be solved to obtain
\begin{eqnarray*}
\tilde{\mathsf{E}}(B_j) &=& \log\frac{\tilde{\mathsf{E}}(e^{B_j})^2}{\sqrt{\widetilde{\mathsf{Var}}(e^{B_j}) + \tilde{\mathsf{E}}(e^{B_j})^2}},\\
\widetilde{\mathsf{Var}}(B_j) &=& \log\left(1 + \frac{\widetilde{\mathsf{Var}}(e^{B_j})}{\tilde{\mathsf{E}}(e^{B_j})^2}\right).
\end{eqnarray*}
This gives estimates of the mean and the variance of $B_j$, and therefore the standard deviation.
The estimated mean and variance of $Y_j$ are also given by the same formulas as above.
One just need to replace all $B_j$ in the formulas by $Y_j$.



\subsection{Byproduct}
Since we can also obtain the grid function of $f_{Y_0}$, there is a better way to evaluate the Asian call option value,
instead of the interpolation step in line 10 in the modified Benhamou algorithm in Figure \ref{algo:conv_asian_benh_CV_2}.
In fact, since the average price can be written as
$$
A = \frac{S_{t_0}}{n+1}e^{Y_0},
$$
one can rewrite the Asian call option price as follows
\begin{eqnarray}
C %&=& e^{-rT}E[(A-K)^+]\\
  &=& e^{-rT}\tilde{\mathsf E}\left[\left(\frac{S_{t_0}}{n+1}e^{Y_0}-K\right)^+\right]\nonumber\\
  &=& e^{-rT} \int_{-\infty}^{\infty} \left(\frac{S_{t_0}}{n+1}e^{x}-K\right)^+ f_{Y_0}(x) \,dx\nonumber\\
  &=& e^{-rT} \int_{-\infty}^{\infty} \left(\frac{S_{t_0}}{n+1}e^{x}-K\right)^+ f_{Y_0}^{\text{CV}}(x) \,dx + e^{-rT} \int_{-\infty}^{\infty} \left(\frac{S_{t_0}}{n+1}e^{x}-K\right)^+ f_{Y_0}^{\text{Diff}}(x) \,dx. \qquad\qquad\label{eq:AsianCallDecomp}
\end{eqnarray}



As have been discussed in the previous section,
$f_{Y_0}(x)$ is well-approximated by a normal density function.
We have also discussed how to derive closed form approximations of the mean and the standard deviation of $Y_0$.
Denote by $\mu_{Y_0}$ and $\sigma_{Y_0}$ the estimated mean and standard deviation, respectively.
Then the control variate is the normal density function
$$
f_{Y_0}^{\text{CV}}(x) = \frac{1}{\sigma_{Y_0}\sqrt{2\pi}}e^{-\frac{1}{2}\left(\frac{x-\mu_{Y_0}}{\sigma_{Y_0}}\right)^2}.
$$
Plug this into (\ref{eq:AsianCallDecomp}) to obtain
\begin{eqnarray}
C &=& e^{-rT} \int_{-\infty}^{\infty} \left(\frac{S_{t_0}}{n+1}e^{x}-K\right)^+ \frac{1}{\sigma_{Y_0}\sqrt{2\pi}}e^{-\frac{1}{2}\left(\frac{x-\mu_{Y_0}}{\sigma_{Y_0}}\right)^2} \,dx \label{eq:AsianCallCV} \\
&&+~ e^{-rT} \int_{-\infty}^{\infty} \left(\frac{S_{t_0}}{n+1}e^{x}-K\right)^+ f_{Y_0}^{\text{Diff}}(x) \,dx. \label{eq:AsianCallDiff}
\end{eqnarray}
Only the second integral requires numerical integration, and the error will be tiny since the difference $f_{Y_0}^{\text{Diff}}(x)$ only takes small values.
The first integral can be evaluated explicitly as follows
\begin{eqnarray*}
&& e^{-rT} \int_{-\infty}^{\infty} \left(\frac{S_{t_0}}{n+1}e^{x}-K\right)^+
\frac{1}{\sigma_{Y_0}\sqrt{2\pi}}e^{-\frac{1}{2}\left(\frac{x-\mu_{Y_0}}{\sigma_{Y_0}}\right)^2} dx \\
&=& e^{-rT} \int_{\xi}^{\infty} \left(\frac{S_{t_0}}{n+1}e^{x}\right)
\frac{1}{\sigma_{Y_0}\sqrt{2\pi}}e^{-\frac{1}{2}\left(\frac{x-\mu_{Y_0}}{\sigma_{Y_0}}\right)^2} dx
-K e^{-rT} \int_{\xi}^{\infty}
\frac{1}{\sigma_{Y_0}\sqrt{2\pi}}e^{-\frac{1}{2}\left(\frac{x-\mu_{Y_0}}{\sigma_{Y_0}}\right)^2} dx,
\end{eqnarray*}
where $\xi$ is the solution of $\frac{S_{t_0}}{n+1}e^{x} - K = 0$, that is $\xi = \log \frac{K(n+1)}{S_{t_0}}$.
The last integral on the right hand side is
$$
-K e^{-rT} N\left( -\frac{ \xi - \mu_{Y_0}}{\sigma_{Y_0}} \right) = -K e^{-rT} N\left( \frac{ \log \frac{S_{t_0}}{K(n+1)} + \mu_{Y_0}}{\sigma_{Y_0}} \right),
$$
and the first integral can be derived as follows
\begin{eqnarray*}
&& e^{-rT} \int_{\xi}^{\infty} \left(\frac{S_0}{n+1}e^{x}\right) \frac{1}{\sigma_{Y_0}\sqrt{2 \pi }}e^{-\frac{(x-{\mu_{Y_0}})^2}{2 {\sigma_{Y_0}^2}}}\,dx
%&=& \frac{S_0 e^{-rT} }{(n+1)}\int_{\xi}^{\infty} \frac{1}{\sigma_{Y_0}\sqrt{2 \pi}}e^{x-\frac{(x-{\mu_{Y_0}})^{2}}{2{\sigma_{Y_0}^2}}}\,dx\\
=\frac{S_0 e^{-rT} }{(n+1)}\int_{\xi}^{\infty} \frac{1}{\sigma_{Y_0}\sqrt{2\pi}} e^{-\left[\frac{(x-{\mu_{Y_0}})^2-2{\sigma_{Y_0}^2} x}{2 {\sigma_{Y_0}^2}}\right]}\,dx\\
&=&\frac{S_0 e^{-rT} }{(n+1)}\int_{\xi}^{\infty} \frac{1}{\sigma_{Y_0}\sqrt{2\pi }}e^{-\left[\frac{\left(x-\left({\mu_{Y_0}} +{\sigma_{Y_0}^2} \right)\right)^2}{2 {\sigma_{Y_0}^2}}\right] +\frac{-2 {\mu_{Y_0}} {\sigma_{Y_0}^2} -{\sigma_{Y_0}^4}}{-2 {\sigma_{Y_0}^2}}}\,dx\\
&=&\frac{S_0 e^{-rT} }{(n+1)}e^{\frac{2 {\mu_{Y_0}} +{\sigma_{Y_0}^2}}{2}}N\left(-\frac{\xi-({\mu_{Y_0}} +{\sigma_{Y_0}^2})}{{\sigma_{Y_0}}}\right) = \frac{S_0 e^{-rT} }{(n+1)}e^{\frac{2 {\mu_{Y_0}} +{\sigma_{Y_0}^2}}{2}}N\left( \frac{ \log \frac{S_{t_0}}{K(n+1)} + {\mu_{Y_0}} +{\sigma_{Y_0}^2}}{{\sigma_{Y_0}}}\right).
\end{eqnarray*}
To sum up, the integral (\ref{eq:AsianCallCV}) is
\begin{eqnarray*}
&& e^{-rT} \int_{-\infty}^{\infty} \left(\frac{S_{t_0}}{n+1}e^{x}-K\right)^+ f_{Y_0}^{\text{CV}}(x) dx \\
&&=\quad \frac{S_0 e^{-rT} }{(n+1)}e^{\frac{2 {\mu_{Y_0}} +{\sigma_{Y_0}^2}}{2}}N\left( \frac{ \log \frac{S_{t_0}}{K(n+1)} + {\mu_{Y_0}} +{\sigma_{Y_0}^2}}{{\sigma_{Y_0}}}\right) -K e^{-rT} N\left( \frac{ \log \frac{S_{t_0}}{K(n+1)} + \mu_{Y_0}}{\sigma_{Y_0}} \right).
\end{eqnarray*}
This is the control variate of the Asian call option price.
In fact, since the difference (\ref{eq:AsianCallDiff}) is tiny, one can just ignore it and use the above formula as a closed form approximation.



\subsection{Numerical Results}


As explained in Section \ref{sec:CONVBackgroundReview},
in any convolution based pricing algorithm we first set up an interval $[-a, a]$ and a grid $-a = x_0 < x_1 < x_2 \cdots < x_m = a$.
For all numerical results in this section we set $a=2$ and change the number of grid points.


First we compare the Bermudan options pricing algorithm to its control variate modifications.
Consider a Bermudan put option with strike price $K=100$ and maturity $T=1$,
whose underlying asset has initial price $S_0 = 100$, volatility $\sigma = 0.3$.
Let $r=0.1$ be the risk-free interest rate.
Assume that throughout the life of this one-year-long option,
there are 12 observation points at time $t = 1/12, 2/12, \ldots, 1$ where the option is allowed to be exercised.
For benchmark, we use the pricing result of the na\"ive convolution based pricing algorithm with $m=102400$, which is $8.24327435926$.
Figure \ref{tab:berm} shows an efficiency and accuracy comparison of three algorithms.
``Bermudan'' is the na\"ive convolution based pricing algorithm,
``BSCV'' its modification with the Black-Scholes put price as the control variate,
and ``expCV'' is the modification with (\ref{eq:CVem}) as the control variate.
The first column $m$ is the number of partitions of the interval $[-a, a]$.
``CPU (s)'' is the running time of the program in seconds, and ``log error'' columns give the pricing error taking common logarithm.
Because of the extra cost to evaluate the control variate,
given the same number of partitions, both control variate modifications take more time than the na\"ive algorithm to accomplish the task.
But the extra cost gets less significant as $m$ goes larger and larger,
because in each loop iteration in the modified algorithm in Figure \ref{algo:conv_berm_CV}, the control variate in line 6 only takes $O(m)$ computations,
while the convolution step in line 7 using the FFT algorithm takes $O(m\log m)$ computations.
When $m$ is large, the majority of the computational time is spent on the convolution step.
Besides, given the same number of grid points the two control variate modifications both give more accurate pricing results than the na\"ive algorithm.
In fact, with the same computational time, the control variate modifications also give more accuracy than the na\"ive algorithm.
This is obvious in Figure \ref{fig:berm}.



\begin{figure}[!t]
\begin{center}
\begin{tabular}{rrrrrrr}
\toprule
{} &      BSCV &           &  Bermudan &           &     expCV &           \\
{$m$} &   CPU (s) & log error &   CPU (s) & log error &   CPU (s) & log error \\
\midrule
100  &  0.001888 & -2.656520 &  0.001125 & -1.928496 &  0.002173 & -1.972770 \\
200  &  0.003415 & -3.380402 &  0.001569 & -2.548871 &  0.002956 & -2.846451 \\
400  &  0.005393 & -3.981319 &  0.002921 & -3.151388 &  0.005494 & -3.792331 \\
800  &  0.009425 & -4.651855 &  0.006419 & -3.889831 &  0.012747 & -4.842674 \\
1600 &  0.020544 & -5.353463 &  0.015812 & -4.478109 &  0.019162 & -5.547746 \\
3200 &  0.059600 & -6.315952 &  0.051682 & -5.048506 &  0.056450 & -6.448490 \\
6400 &  0.243433 & -6.890989 &  0.221945 & -5.604359 &  0.237604 & -7.224185 \\
\bottomrule
\end{tabular}
\end{center}
\caption{
Efficiency and Accuracy Comparison of the Convolution Based Bermudan Options Pricing Algorithm and Its Control Variate Modifications.
``Bermudan'' is the na\"ive algorithm,
``BSCV'' its modification with the Black-Scholes control,
and ``expCV'' is the modification with (\ref{eq:CVem}) as the control variate.
``CPU (s)'' is the computational time in seconds,
and ``log error'' is the pricing error taking common logarithm.
}\label{tab:berm}
\end{figure}



\begin{figure}[!t]
\begin{center}
    \parbox[t]{0.8\textwidth}{
        \centerline{\epsfig{figure=eps/conv.eps, width=0.8\textwidth}}
    }\hfill
\end{center}
\caption{
Efficiency and Accuracy Comparison of the Convolution Based Bermudan Options Pricing Algorithm and Its Control Variate Modifications.
}\label{fig:berm}
\end{figure}



\begin{figure}[!t]
\begin{center}
    \parbox[t]{0.8\textwidth}{
        \centerline{\epsfig{figure=eps/convn.eps, width=0.8\textwidth}}
    }\hfill
\end{center}
\caption{
Accuracy Comparison of the Convolution Based Bermudan Options Pricing Algorithm and Its Control Variate Modifications.
``num'' is the number of grid points.
}\label{fig:bermn}
\end{figure}



We get a similar numerical results for Asian and discrete barrier options.
To examine the barrier options pricing results, consider a discrete up and out call option with strike price $K = 120$, barrier $H=130$ and maturity $T=1$,
whose underly asset has initial price $S_0 = 100$ and volatility $\sigma = 0.3$.
The risk-free interest rate is assumed to be a constant $r = 0.1$.
Assumed that there are 12 uniform observation points where the option can be knocked out.
The benchmark is 0.176355450733, obtained with number of partitions $m=102400$.
For Asian options pricing algorithms examination, we use the same numerical settings $S_0 = 100$, $K = 120$, $T=1$, $\sigma = 0.3$, $r = 0.1$.
There 13 uniform time points where the underlying asset price is observed for average evaluation, including $t=0$ and $t=T$.
The numerical results are shown in Figures \ref{tab:barrier}, \ref{fig:conv_barrier}, \ref{fig:convn_barrier}, \ref{tab:conv_asian}, \ref{fig:conv_asian} and \ref{fig:convn_asian}.
The ``UOcall'' in Figures \ref{tab:barrier}, \ref{fig:conv_barrier} and \ref{fig:convn_barrier}
refers to the na\"ive convolution based pricing algorithm for discrete up and out call options,
and ``UOcallCV'' its control variate modification.
In Figures \ref{tab:conv_asian}, \ref{fig:conv_asian} and \ref{fig:convn_asian},
``Benh'' is Benhamou's algorithm, and ``BenhCV'' its control variate modification.
For discrete barrier option, we use the scaled normal density functions (\ref{eq:scaledBellShape}) as the CV, as described in Section \ref{sec:findingCVforBarrier}.
For Asian option, we use normal density functions (\ref{eq:fZCV}).




Same as in the Bermudan pricing, in both Asian and barrier options pricing, given the same number of partitions,
the control variate modifications are more time consuming algorithms because of the extra time spent on control variate evaluations,
but the modified algorithms also get much more accuracy out of the control variates.
Overall, the control variate evaluation is a worth-taking cost,
which can be seen in the graphs of pricing error against computational time in Figures \ref{fig:conv_barrier} and \ref{fig:conv_asian}.



\begin{figure}[!t]
\begin{center}
\begin{tabular}{lrrrr}
\toprule
{} &    UOcall &           &  UOcallCV &           \\
{$m$} &   CPU (s) & log error &   CPU (s) & log error \\
\midrule
200   &  0.001761 & -1.282222 &  0.007278 & -1.471718 \\
400   &  0.002679 & -1.588148 &  0.009405 & -1.762492 \\
800   &  0.005815 & -1.891260 &  0.013437 & -2.057356 \\
1600  &  0.014535 & -2.185156 &  0.025336 & -2.345303 \\
3200  &  0.049247 & -2.489426 &  0.063436 & -2.645831 \\
6400  &  0.203671 & -2.781629 &  0.217144 & -2.932702 \\
12800 &  0.642218 & -3.064788 &  0.696188 & -3.207904 \\
\bottomrule
\end{tabular}
\end{center}
\caption{
Comparison of Discrete Barrier Options Pricing Algorithms
}\label{tab:barrier}
\end{figure}



\begin{figure}[!t]
\begin{center}
    \parbox[t]{0.8\textwidth}{
        \centerline{\epsfig{figure=eps/conv_barrier.eps, width=0.8\textwidth}}
    }\hfill
\end{center}
\caption{
Efficiency and Accuracy Comparison of Discrete Barrier Options Pricing Algorithms
}\label{fig:conv_barrier}
\end{figure}



\begin{figure}[!t]
\begin{center}
    \parbox[t]{0.8\textwidth}{
        \centerline{\epsfig{figure=eps/convn_barrier.eps, width=0.8\textwidth}}
    }\hfill
\end{center}
\caption{
Accuracy Comparison of Discrete Barrier Options Pricing Algorithms
}\label{fig:convn_barrier}
\end{figure}





\begin{figure}[!t]
\begin{center}
\begin{tabular}{lrrrr}
\toprule
{} &      Benh &           &    BenhCV &           \\
{} &   CPU (s) & log error &   CPU (s) & log error \\
\midrule
200   &  0.002641 & -1.605930 &  0.010837 & -2.365654 \\
400   &  0.003277 & -2.125959 &  0.014881 & -2.953226 \\
800   &  0.005955 & -2.776188 &  0.017535 & -3.555072 \\
1600  &  0.013135 & -3.394242 &  0.027068 & -4.157442 \\
3200  &  0.048506 & -3.963035 &  0.064932 & -4.761088 \\
6400  &  0.203419 & -4.582184 &  0.236476 & -5.370957 \\
12800 &  0.676367 & -5.175409 &  0.789486 & -6.006516 \\
\bottomrule
\end{tabular}
\end{center}
\caption{
Comparison of Asian Options Pricing Algorithms
}\label{tab:conv_asian}
\end{figure}

\begin{figure}[!t]
\begin{center}
    \parbox[t]{0.8\textwidth}{
        \centerline{\epsfig{figure=eps/conv_asian.eps, width=0.8\textwidth}}
    }\hfill
\end{center}
\caption{
Efficiency and Accuracy Comparison of Asian Options Pricing Algorithms
}\label{fig:conv_asian}
\end{figure}

\begin{figure}[!t]
\begin{center}
    \parbox[t]{0.8\textwidth}{
        \centerline{\epsfig{figure=eps/convn_aisan.eps, width=0.8\textwidth}}
    }\hfill
\end{center}
\caption{
Accuracy Comparison of Asian Options Pricing Algorithms
}\label{fig:convn_asian}
\end{figure}






\section{Pricing American Puts with Binomial Tree}
\label{Sec:AmePut}

\subsection{Binomial Options Pricing Model}
The argument introduced in Section \ref{sec:martingale} for derivatives pricing is a continuous time argument,
and at any time in the future, the underlying asset price has infinitely many possible states.
There is also the binomial options pricing model which starts with a discrete time argument,
and at any time there can only be finitely many possible states for the price of the underlying asset.
The binomial options pricing model is easy to understand and is reasonably efficient as a numerical pricing algorithm, that is why it is popular in practice.



The argument of the binomial options pricing model starts with a two period model where the underlying asset has price $S_0$ at time $t_0$.
At time $t_1$, it is assumed that the only two possible states of the underlying asset price are $S_0 u$ and $S_0 d$ for some constants $u>1$ and $d<1$.
Assume that at time $t_1$ the derivative to be priced pays off $X_u$ if the underlying asset price goes up to $S_0u$,
and $X_d$ if it goes down to $S_0 d$.
Let $R$ be the time $t_1$ value of one unit of the money market account.
Consider the portfolio consisting of $B$ dollars cash and $h$ shares of the underlying asset initially.
The initial value is clearly $B + hS_0$.
At time $t_1$, the portfolio has value $BR + hS_0 u$ if the underlying asset price goes up, and $BR + hS_0 d$ if it goes down.
If we want the portfolio to replicate the derivative, we have to have
\begin{eqnarray*}
\begin{cases}
BR + hS_0 u = X_u\\
BR + hS_0 d = X_d
\end{cases}.
\end{eqnarray*}
This idea of replicating (the payoff of) a derivative is similar to the argument in Section \ref{sec:martingale}.
The only difference is that now we only have two period $t_0$ and $t_1$, and $t_1$ is the maturity of the derivative to be priced.
Once the portfolio is set up at $t_0$, there is no chance to change it.
Thus we do not bother setting up a self-financing strategy.




In matrix form, one can write
$$
\begin{pmatrix}
R & S_0 u\\
R & S_0 d\\
\end{pmatrix}
\begin{pmatrix}
B\\
h
\end{pmatrix}
 =
\begin{pmatrix}
X_u\\
X_d
\end{pmatrix},
$$
or equivalently,
$$
\begin{pmatrix}
B\\
h
\end{pmatrix}
 =
\begin{pmatrix}
R & S_0 u\\
R & S_0 d\\
\end{pmatrix}^{-1}
\begin{pmatrix}
X_u\\
X_d
\end{pmatrix}
=
-\frac{1}{RS_0 (u-d)}
\begin{pmatrix}
S_0 d & -S_0 u\\
-R & R\\
\end{pmatrix}
\begin{pmatrix}
X_u\\
X_d
\end{pmatrix}
.
$$
This is the holding vector of the portfolio that replicates the derivative.
The agent who sets up a portfolio with this holding vector will have the same payoff as if he or she holds the derivative.
By the no-arbitrage principle, this portfolio should have the same initial value as the derivative.
The initial value of the portfolio is
\begin{eqnarray*}
B+hS_0 &=&
\begin{pmatrix}
1 & S_0
\end{pmatrix}
\begin{pmatrix}
B\\
h
\end{pmatrix}
=
-\frac{1}{RS_0 (u-d)}
\begin{pmatrix}
1 & S_0
\end{pmatrix}
\begin{pmatrix}
S_0 d & -S_0 u\\
-R & R\\
\end{pmatrix}
\begin{pmatrix}
X_u\\
X_d
\end{pmatrix}\\
&=&
-\frac{1}{RS_0 (u-d)}
\begin{pmatrix}
S_0(d-R) & S_0(R-u)
\end{pmatrix}
\begin{pmatrix}
X_u\\
X_d
\end{pmatrix}\\
&=& \frac1R\left[ \left(\frac{R-d}{u-d}\right) X_u + \left(\frac{u-R}{u-d}\right) X_d \right].
\end{eqnarray*}
Thus we conclude that the fair value, or the no-arbitrage price, of the derivative at $t_0$ is
\begin{eqnarray}
V_0 = \frac1R\left[ \left(\frac{R-d}{u-d}\right) X_u + \left(\frac{u-R}{u-d}\right) X_d \right].\label{eq:universal_pricing_formula_discrete}
\end{eqnarray}




The above is the discrete version of the pricing formula (\ref{eq:universal_pricing_formula}) in Section \ref{sec:martingale}.
The two coefficients
\begin{eqnarray}
p = \left(\frac{R-d}{u-d}\right), \qquad q = \left(\frac{u-R}{u-d}\right)\label{eq:risk_neutral prob_discrete}
\end{eqnarray}
sum to one.
When they are both positive, that is when $d<R<u$, this pair is a discrete probability measure.
It is the risk neutral probability measure in this oversimplified two period discrete world.
$1/R$ is the discount factor.
The fair value of the derivative is the discounted expected value of the payoff under the risk neutral probability,
just like the continuous version pricing formula (\ref{eq:universal_pricing_formula}).



In reality, time is continuous, the value of an asset can change anytime,
and there are infinitely many possible states of an asset price at any specific time in the future.
In a binomial options pricing model, this is approximated by adding more time steps and more possible states of the underlying asset price.
One well-known specification by \citet{cox1979option} is as follows.
Let $r$ be the constant risk free interest rate, and $e^{rt}$ the time $t$ value of one unit of money market account.
Say we want to price a derivative with maturity $T$, and we have $n$ time steps from time $t=0$ to $t=T$.
Denote $\Delta t = T/n$.
Let
\begin{eqnarray}
u = \exp(\sigma \sqrt{\Delta t}), \qquad d = 1/u\label{eq:udCRR}
\end{eqnarray}
for some constant $\sigma >0$.
At time $t=\Delta t$, one unit of money market account has value $e^{r\Delta t}$,
and the underlying asset price can either go up to $S_0 u = S_0 \exp(\sigma \sqrt{\Delta t})$ or go down to $S_0 d = S_0 \exp(-\sigma \sqrt{\Delta t})$.
These two possible states are the two tree nodes at $t=\Delta t$.
At each node, going one time step further, the underlying asset price can again either go up or down, giving us tree possible states, or nodes, at $t = 2\Delta t$.
They are $S_0 u^2$, $S_0$, and $S_0 d^2$.
In general, at $t = j\Delta t$, there are $j+1$ nodes $S_0 u^j, S_0 u^{j-2}, S_0 u^{j-4}, \ldots, S_0 d^{j}$.
At each node at each time step, by (\ref{eq:risk_neutral prob_discrete}), the underlying asset price goes up and down with probabilities
$$
p = \left(\frac{R-d}{u-d}\right) = \frac{e^{r\Delta t} - e^{-\sqrt{\sigma \Delta t}}}{e^{\sqrt{\sigma \Delta t}} - e^{-\sqrt{\sigma \Delta t}}}, \qquad
q = \left(\frac{u-R}{u-d}\right) = \frac{e^{\sqrt{\sigma \Delta t}} - e^{r\Delta t}}{e^{\sqrt{\sigma \Delta t}} - e^{-\sqrt{\sigma \Delta t}}},
$$
respectively in the risk neutral world.
By (\ref{eq:universal_pricing_formula_discrete}),
the derivative price at a node at time $j\Delta t$ is the discounted weighted sum of the two following nodes at time $(j+1)\Delta t$ with weights $p$, $q$.
Thus, to price a derivative, one first set up the payoff at time $T = n\Delta t$.
Then (\ref{eq:universal_pricing_formula_discrete}) can be used to find the fair price of the derivative at time $(n-1)\Delta t$.
Having the price of the derivative at time $(n-1)\Delta t$,
(\ref{eq:universal_pricing_formula_discrete}) can be applied again to find the derivative price at time $(n-2)\Delta t$.
This procedure is called backward induction.
It can be used repeatedly and one obtains the derivative price backward in time until the time zero value of the derivative is found.
This option pricing algorithm is known as the Cox-Ross-Rubinstein tree, or the CRR tree for short, in honor of the researchers who developed this algorithm.


The CRR tree pricing algorithm for a vanilla call option is shown in Figure \ref{algo:CRR}.
The risk neutral probability is set up in line 4 and 5, and the payoff is set up in line 6.
Line 8 in the for loop is the backward induction step that evaluates the option values at the $(i-1)$th time step as a vector,
given the option value vector at the $i$th time step.
At time step $i$, there are $i+1$ possible states of the underlying asset value.
Thus the option value vector is $i+1$ dimensional, each element used to keep the put value for a different state.
At time step 0, there is only one state and therefore one option value, that is the initial fair price of the put returned in line 10.



It can be shown that as $n\rightarrow \infty$, the derivative price obtained by the CRR tree algorithm converges to the fair price evaluated under the Black-Scholes model
with the volatility of the underlying asset equal to $\sigma$.
Roughly speaking, at each tree node the underlying asset price goes up or down according to a fixed (risk-neutral) probability.
The log return at time $T = n\Delta t$ is therefore binomial distributed with $n+1$ possible states.
As $n\rightarrow \infty$, the central limit theorem guarantees that this binomial distribution converges to the same normal distribution as in the Black-Scholes model.



\begin{figure}[!t]
\begin{framed}
%\line(1,0){240}
\begin{algorithmic}[1]
    \STATE \textbf{input}: $S_0, K, r, s, T, n$
    \STATE $u \leftarrow e^{s\sqrt{T/n}}$
    \STATE $d \leftarrow 1/u$
    \STATE $p \leftarrow \frac{e^{rT/n} - d}{u-d}$
    \STATE $q \leftarrow 1-p $
    \STATE $call \leftarrow (\{S_0 u^n, S_0 u^{n-2}, S_0 u^{n-4}, \ldots, S_0 d^{n}\} - K)^+$
    \FOR{$i=n$ \textbf{downto} $1$ }
        \STATE $call \leftarrow e^{-rT/n}(p*call[0:i] + q*call[1 : i+1])$
    \ENDFOR
    \STATE \textbf{return} $call[0]$
\end{algorithmic}
\end{framed}
\caption{
CRR Tree Pricing Algorithm for European Call Options
}
\label{algo:CRR}
\end{figure}




\subsection{American Put and Its Pricing}
Unlike European options which can only be exercised at the maturity,
American options are options that can be exercised anytime before its maturity.
Consider a put option that gives its holder the right to sell the underlying asset at the strike price $K$.
Denote by $S_t$ the price of the underlying asset at time $t$.
If this put option is exercised at time $\tau$, the holder gets payoff $(K-S_{\tau})^+$.
Discounted, the present value of the payoff at time zero is $e^{-r\tau}(K-S_{\tau})^+$,
and therefore by (\ref{eq:universal_pricing_formula})
the fair value of this option at time zero would be the expectation under the risk neutral probability, that is $\tilde E[e^{-r\tau}(K-S_{\tau})^+]$.
This is derived under the assumption that this option would be exercised at $\tau$.
A rational option holder would choose the best time $\tau$ to exercise the option
the time that maximizes the option value based on all currently available information.
Thus the fair price of an American put is
\begin{eqnarray}
\max_{\tau \in \mathcal T_{[0, T]}} \tilde E[e^{-r\tau}(K-S_{\tau})^+],\label{eq:AmericanPut}
\end{eqnarray}
where $\mathcal T_{[0, T]}$ is the set of all stopping times in $[0, T]$.
Stopping time is a kind of random time whose definition involves advanced concepts in probability theory, which is beyond the scope of this thesis,
but intuitively, it just means whether or not the random time has already happened only depends on all currently available information.



The fair value of an American put can also be determined by the CRR tree pricing algorithm,
and the same as European option pricing, when the number of time steps goes to infinity,
the pricing result given by the CRR tree converges to the theoretical fair price (\ref{eq:AmericanPut}) under the Black-Scholes model.
The only difference in the algorithm is that for American options pricing,
after each backward induction step one should check the option value for all current possible states and update it if necessary.
If the option value is lower than if it is exercised, a rational investor would decide to exercise the option right away.
Specifically, let $P$ be the put value given by the backward induction for some possible state at some time step,
and $S$ the underlying asset value at the same time at the same state.
If the option is exercised and its holder sells the underlying asset at the strike price $K$ at this time, the payoff is the price difference $(K - S)$.
The CRR tree algorithm considers two scenarios.
Either keep the option with value $P$ or exercise it to get the payoff $(K - S)$.
The algorithm takes whichever is greater to be the updated put value for the state at that time.




The CRR tree pricing algorithm for American puts is shown in Figure \ref{algo:CRR_American}.
The risk neutral probability, the terminal option value vector, and the payoff vector are set up in line 4, 5, 6 and 7,
same as the European options pricing algorithm in Figure \ref{algo:CRR}.
The backward induction step in line 9 in the loop is also the same,
but in line 10 the put value is compared to the payoff and updated if too low.
The comparison is done element by element.
The syntax of line 10 is similar to the vector operation in many programming languages or scientific libraries, such as the Python package Numpy.




\begin{figure}[!t]
\begin{framed}
%\line(1,0){240}
\begin{algorithmic}[1]
    \STATE \textbf{input}: $S_0, K, r, s, T, n$
    \STATE $u \leftarrow e^{s\sqrt{T/n}}$
    \STATE $d \leftarrow 1/u$
    \STATE $p \leftarrow \frac{e^{rT/n} - d}{u-d}$
    \STATE $q \leftarrow 1-p $
    \STATE $payoff \leftarrow (K - \{S_0 u^n, S_0 u^{n-2}, S_0 u^{n-4}, \ldots, S_0 d^{n}\})^+$
    \STATE $put \leftarrow payoff$
    \FOR{$i=n$ \textbf{downto} $1$ }
        \STATE $put \leftarrow e^{-rT/n}(p*put[0:i] + q*put[1 : i+1])$
        \STATE $put \leftarrow \max(put, payoff)$
    \ENDFOR
    \STATE \textbf{return} $put[0]$
\end{algorithmic}
\end{framed}
\caption{
CRR Tree Pricing Algorithm for American Put Options
}
\label{algo:CRR_American}
\end{figure}




\subsection{Modified Algorithms with Control Variates}


A numerical pricing algorithm like Figure \ref{algo:CRR} or its improvement is not necessary in practice,
since the vanilla options have closed form pricing formulas under the Black-Scholes model.
Its existence in the literature only serves educational purposes,
in order to demonstrate how the binomial options pricing model and the binomial tree pricing algorithm work.
For this reason, the focus of this section is only on the improvement of the American put pricing algorithm in Figure \ref{algo:CRR_American} with the control variate technique.



The backward induction step in any binomial tree algorithm is a coarse approximation of the convolution step in the convolution pricing algorithms,
and it is obviously a linear operation.
Thus it can be modified by the control variate technique to greatly reduce the numerical error.
In the American put pricing algorithm in Figure \ref{algo:CRR_American},
in the $i$th iteration of the loop, before the backward induction step in line 9 the vector $put$ keeps the American put values at time step $i$.
We need to find a control variate that is reasonably close to the American put value, and that has a closed form result when the convolution step is applied.
We use the fair value of the European put at time step $i$, given by the Black-Scholes formula.
If there were infinitely many time steps between time $t_i$ and $t_{i-1}$,
repeated backward induction will turn the European put value at time $t_i$ to its value at time $t_{i-1}$.
This is because the result of a CRR tree pricing algorithm for vanilla options converges to the fair price under the Black-Scholes model.
Thus the result given by the convolution step comes in closed form and is given by the Black-Scholes put value at time $t_{i-1}$.



To apply the control variate technique to the standard CRR tree algorithm in Figure \ref{algo:CRR_American},
we only modify the backward induction step in line 9, because it is the only linear operator (on the $put$ vector).
The early exercise checking step in line 10 is not always linear.
Depending on the values kept in it, the $put$ vector can be partially replaced by the $payoff$ vector in this step.
In the modified algorithm,
before sending the $put$ vector to the backward induction at time step $i$,
it is first subtracted by a vector that keeps the Black-Scholes put prices at time step $i$.
The backward induction is then applied to the difference.
The resulting vector is added by a vector of the Black-Scholes put prices at time step $i-1$ as the new $put$ vector to be checked for exercise.
The algorithm is shown in Figure \ref{algo:CRR_CV}.







\begin{figure}[!t]
\begin{framed}
%\line(1,0){240}
\begin{algorithmic}[1]
    \STATE \textbf{input}: $S_0, K, r, s, T, n$
    \STATE $u \leftarrow e^{s\sqrt{T/n}}$
    \STATE $d \leftarrow 1/u$
    \STATE $p \leftarrow \frac{e^{rT/n} - d}{u-d}$
    \STATE $q \leftarrow 1-p $
    \STATE $payoff \leftarrow (K - \{S_0 u^n, S_0 u^{n-2}, S_0 u^{n-4}, \ldots, S_0 d^{n}\})^+$
    \STATE $put \leftarrow payoff$
%    \FOR{$i=n$ \textbf{downto} $1$ }
%        \STATE $CV_i \leftarrow \text{vector of the Black-Scholes put values at time step $i$}$
%    \ENDFOR
    \FOR{$i=n$ \textbf{downto} $1$ }
        \STATE $put_{diff} \leftarrow put - CV^i$
        \STATE $put_{diff} \leftarrow e^{-rT/n}(p*put_{diff}[0:i] + q*put_{diff}[1 : i+1])$
        \STATE $put \leftarrow put_{diff} + CV^{i-1}$
        \STATE $put \leftarrow \max(put, payoff)$
    \ENDFOR
    \STATE \textbf{return} $put[0]$
\end{algorithmic}
\end{framed}
\caption{
Modified American Put Pricing Algorithm with Control Variate.
$CV_i$ is a vector that keeps the Black-Scholes put values at time step $i$.
}
\label{algo:CRR_CV}
\end{figure}



In comparison with the original CRR tree algorithm in Figure \ref{algo:CRR_American}, its CV modification in Figure \ref{algo:CRR_CV} yields more accurate pricing results.
The backward induction step in line 9 in Figure \ref{algo:CRR_American} is replaced by line 9, 10, and 11 in Figure \ref{algo:CRR_CV}.
When we choose the Black-Scholes put prices as the control variate and subtract it from the $put$ vector,
the result (saved in the $put_{diff}$ vector) is known as the the early exercise premium.
Financially it means the premium the option holder pays for the right to exercise the option prior to the maturity.
Also, the financial interpretation of the identity $put = put_{diff} + CV$ is
$$
\text{American option price} = \text{early exercise premium} + \text{European option price},
$$
except, of course, due to the accumulated error,
the early exercise premium kept in the $put_{diff}$ vector is not the theoretical true value but simply a numerical approximation.
In line 10 in Figure \ref{algo:CRR_American}, only the early exercise premium is sent to the backward induction step but not the European option price.
The backward induction step is a coarse approximation of the convolution step in the convolution pricing algorithm,
and it yields significant numerical errors.
Sending the entire American option price to such an operator would yield greater numerical error than sending the early exercise premium only.
Since the European option price in line 11 is given by the Black-Scholes formula, its value is exact.




The algorithm in Figure \ref{algo:CRR_CV} is however very inefficient, mainly because the Black-Scholes formula is evaluated at every node in the tree.
In the original CRR tree pricing algorithm in Figure \ref{algo:CRR_American},
at each node we take the discounted expected value of the put values in the previous time step (see line 9),
which only consists of very cheap arithmetic operations.
In contrast, evaluating the Black-Scholes formula requires taking logarithm, square root, exponential functions,
and two special function evaluations for the cumulated distribution function of the standard normal distribution.
Those are computationally much more expensive than backward induction.



It is possible to obtain an efficient modified algorithm without loosing the accuracy we get from the control variate technique.
First let us take another look at the original CRR tree pricing algorithm in Figure \ref{algo:CRR_American}.
In each loop iteration, the $put$ vector is sent to the backward induction step in line 9,
and the result is then compared with the payoff function in line 10
to determine if some of the option values kept in the $put$ vector are too low and need to be replaced by the payoff.
At $t = j\Delta t$, $put$ is a $j+1$ dimensional vector that keeps put values corresponding to the underlying asset prices
$S_0 u^j, S_0 u^{j-2}, S_0 u^{j-4}, \ldots, S_0 d^{j}$.
In general, the underlying asset price is in the form $S_0 u^{j-2i}$.
The greater $i$ is, the smaller the underlying asset price is, and the more likely the corresponding put value is too low and will be replaced by the payoff.
The smallest underlying asset price $S_0 u^{j-2i}$ whose corresponding put value is replaced by the payoff is known as the early exercise boundary.
It is a function of time.
Having the concept of early exercise boundary, one can achieve the same thing line 10 in Figure \ref{algo:CRR_American} does another way.
One just needs to locate the early exercise boundary and replaces all the put values with corresponding underlying asset price lower than the boundary by the payoff.
Line 12 in the modified algorithm in Figure \ref{algo:CRR_CV} can be done in the same manner.
The modified algorithm evaluates the Black-Scholes formula at all nodes just to locate the early exercise boundary, that is why it is so inefficient.
To speed up the algorithm, we only evaluate the Black-Scholes formula at the nodes that are close to the early exercise boundary,
only for the purpose of locating the boundary.



If we unwind the loop in Figure \ref{algo:CRR_CV}, the first few iterations would be as follows.
\begin{eqnarray*}
/* &&\text{(iteration $i=n$)} \qquad */\\
put_{diff} &\leftarrow& put - CV^n\\
put_{diff} &\leftarrow& e^{-rT/n}(p*put_{diff}[0:n] + q*put_{diff}[1 : n+1])\\
put &\leftarrow& put_{diff} + CV^{n-1}\\
put &\leftarrow& \max(put, payoff)\\
/* &&\text{(iteration $i=n-1$)} \qquad */\\
put_{diff} &\leftarrow& put - CV^{n-1}\\
put_{diff} &\leftarrow& e^{-rT/n}(p*put_{diff}[0:{n-1}] + q*put_{diff}[1 : n])\\
put &\leftarrow& put_{diff} + CV^{n-2}\\
put &\leftarrow& \max(put, payoff)\\
/* &&\text{(iteration $i=n-2$)} \qquad */\\
put_{diff} &\leftarrow& put - CV^{n-2}\\
put_{diff} &\leftarrow& e^{-rT/n}(p*put_{diff}[0:{n-1}] + q*put_{diff}[1 : {n-2}])\\
put &\leftarrow& put_{diff} + CV^{n-3}\\
put &\leftarrow& \max(put, payoff)\\
&&\qquad\vdots
\end{eqnarray*}
We can see that before and after the early exercise checking step in iteration $i$, there are two opposite statements
\begin{eqnarray*}
put &\leftarrow& put_{diff} + CV^{i-1}\\
put_{diff} &\leftarrow& put - CV^{i-1}
\end{eqnarray*}
each evaluating $i-1$ Black-Scholes formulas for the control variate vector $CV^{i-1}$.
This is unnecessary waste of computational time.
To speed up the algorithm, we only deal with the $put_{diff}$ vector after it is evaluated in the first iteration.
The statement
$$
put \leftarrow \max(put, payoff)
$$
can be replaced by
$$
put - CV^{i-1} \leftarrow \max(put - CV^{i-1}, payoff - CV^{i-1})
$$
or equivalently
$$
put_{diff} \leftarrow \max(put_{diff}, payoff - CV^{i-1}).
$$
The control variate (the Black-Scholes formula) is only evaluated near the early exercise boundary for the comparison of $put_{diff}$ and $payoff - CV^{i-1}$.
As a result, the procedure becomes
\begin{eqnarray*}
put_{diff} &\leftarrow& \vec 0\\
/* &&\text{(iteration $i=n$)} \qquad */\\
put_{diff} &\leftarrow& e^{-rT/n}(p*put_{diff}[0:n] + q*put_{diff}[1 : n+1])\\
put_{diff} &\leftarrow& \max(put_{diff}, payoff - CV^{n-1})\\
/* &&\text{(iteration $i=n-1$)} \qquad */\\
put_{diff} &\leftarrow& e^{-rT/n}(p*put_{diff}[0:{n-1}] + q*put_{diff}[1 : n])\\
put_{diff} &\leftarrow& \max(put_{diff}, payoff - CV^{n-2})\\
/* &&\text{(iteration $i=n-2$)} \qquad */\\
put_{diff} &\leftarrow& e^{-rT/n}(p*put_{diff}[0:{n-1}] + q*put_{diff}[1 : {n-2}])\\
put_{diff} &\leftarrow& \max(put_{diff}, payoff - CV^{n-3})\\
&&\qquad\vdots
\end{eqnarray*}
In the first line, in stead of $put - CV^n$, we assign a zero vector to $put_{diff}$ because of two reasons.
First, this can be justified since the $put_{diff}$ vector is used to keep the early exercise premium, which is zero at maturity.
Secondly, $put - CV^n$ still requires $n$ Black-Scholes formula evaluations.
Setting $put_{diff}$ to zero vector is more efficient.
Such modified algorithm is shown in Figure \ref{algo:CRR_CV_fast},
where in line 10 the control variate is only evaluated near the early exercise boundary.
This can be viewed as a tree of the early exercise premium, instead of the put value.




\begin{figure}[!t]
\begin{framed}
%\line(1,0){240}
\begin{algorithmic}[1]
    \STATE \textbf{input}: $S_0, K, r, s, T, n$
    \STATE $u \leftarrow e^{s\sqrt{T/n}}$
    \STATE $d \leftarrow 1/u$
    \STATE $p \leftarrow \frac{e^{rT/n} - d}{u-d}$
    \STATE $q \leftarrow 1-p $
    \STATE $payoff \leftarrow (K - \{S_0 u^n, S_0 u^{n-2}, S_0 u^{n-4}, \ldots, S_0 d^{n}\})^+$
    \STATE $put_{diff} \leftarrow \text{$n$ dimensional zero vector}$
    \FOR{$i=n$ \textbf{downto} $1$ }
        \STATE $put_{diff} \leftarrow e^{-rT/n}(p*put_{diff}[0:{n-1}] + q*put_{diff}[1 : {n-2}])$
        \STATE $put_{diff} \leftarrow \max(put_{diff}, payoff - CV^{i-1})$
    \ENDFOR
    \STATE \textbf{return} $(put_{diff}[0] + \text{Black-Scholes price at 0})$
\end{algorithmic}
\end{framed}
\caption{
Efficient Modification of the American Put Pricing Algorithm with Control Variate.
$CV_i$ is a vector that keeps the Black-Scholes put values at time step $i$.
}
\label{algo:CRR_CV_fast}
\end{figure}


Only the put values corresponding to underlying asset prices that are greater than the early exercise boundary need to be evaluated.
Below the early exercise boundary, the corresponding put values at all nodes are replaced by the payoff.
Since the early exercise boundary is continuous,
every time the algorithm goes one step backward, to locate the boundary
we just need to evaluate $payoff - CV^{i-1}$ (and hence the Black-Scholes formula) at the nodes that are close to the boundary of the previous time step.
In a way, we find the early exercise boundary recursively.
The initial value of this recursion is the early exercise boundary at maturity, which is $S=K$.




\subsection{Hull and White's Algorithm}



There is another American put pricing algorithm proposed by \citet{hull1988use} which is based on binomial tree and the control variate technique,
but the framework is different from ours.
While both Hull and White and our algorithms use the Black-Scholes formula as the control variate,
in our algorithm a different control variate is applied in every time step.
Overall $n$ control variates are used, $n$ the number of time steps.
In Hull and White's algorithm, only one CV is applied to the CRR tree algorithm as a whole.
In their algorithm, a CRR tree similar to Figure \ref{algo:CRR} is first used to determine the value of the European put with the same set of parameters.
Say the result is $P_E$.
Another CRR tree is than used to determine the American put value, just like in Figure \ref{algo:CRR_American}.
Suppose the result is $P_A$.
Theoretically, when the number of time steps goes to infinity, $P_E$ converges to the put value evaluated by the Black-Scholes formula,
and $P_A$ also converges to the theoretical no-arbitrage price (\ref{eq:AmericanPut}).
Thus $P_A - P_E$ converges to the early exercise premium.
Denote by $P_{BS}$ the time 0 put value given by the Black-Scholes formula, then
$$
P_A - P_E + P_{BS}
$$
converges to the American put value.
The algorithm is shown in Figure \ref{algo:CRR_Hull}.
Line 8, 9 and 10 is the loop evaluating $P_E$, and the loop in line 12, 13, 14 and 15 evaluates $P_A$.


With the same finite number of time steps $n$, if the numerical errors of $P_A$ and $P_E$ are in the same range,
we find the difference $P_A - P_E$ in the hope that the errors will cancel out.
This algorithm, and the explanation why it could give more accurate pricing result, can also be found in \citet{hull2006options}.




\begin{figure}[!t]
\begin{framed}
%\line(1,0){240}
\begin{algorithmic}[1]
    \STATE \textbf{input}: $S_0, K, r, s, T, n$
    \STATE $u \leftarrow e^{s\sqrt{T/n}}$
    \STATE $d \leftarrow 1/u$
    \STATE $p \leftarrow \frac{e^{rT/n} - d}{u-d}$
    \STATE $q \leftarrow 1-p $
    \STATE $payoff \leftarrow (K - \{S_0 u^n, S_0 u^{n-2}, S_0 u^{n-4}, \ldots, S_0 d^{n}\})^+$
    \STATE $put^{(E)} \leftarrow payoff$
    \FOR{$i=n$ \textbf{downto} $1$ }
        \STATE $put^{(E)} \leftarrow e^{-rT/n}(p*put^{(E)}[0:i] + q*put^{(E)}[1 : i+1])$
    \ENDFOR
    \STATE $put^{(A)} \leftarrow payoff$
    \FOR{$i=n$ \textbf{downto} $1$ }
        \STATE $put^{(A)} \leftarrow e^{-rT/n}(p*put^{(A)}[0:i] + q*put^{(A)}[1 : i+1])$
        \STATE $put^{(A)} \leftarrow \max(put^{(A)}, payoff)$
    \ENDFOR
    \STATE \textbf{return} $(put^{(A)}[0] - put^{(E)}[0] + \text{Black-Scholes price at 0})$
\end{algorithmic}
\end{framed}
\caption{
Hull and White's Pricing Algorithm for American Put Options
}
\label{algo:CRR_Hull}
\end{figure}






\subsection{Numerical Results}



For the benchmark consider an American put option with strike price $K=40$ and maturity $T=3$.
Suppose the underlying asset has initial price $S_0 = 40$ and volatility $\sigma = 0.3$.
Let the constant risk-free interest rate be $r = 0.1$.
We run the traditional CRR tree and the numerical results is shown in Figure \ref{tab:CRR}.
The results suggest that the put option has value 1.237687, which we will use as a benchmark.




\begin{figure}[!t]
\begin{center}
\begin{tabular}{rrr}
\toprule
%{} &         CRR &           \\
$n$ &     CPU (s) &    price \\
\midrule
100    &    0.008587 &  1.234086 \\
200    &    0.017313 &  1.235891 \\
400    &    0.038991 &  1.236748 \\
800    &    0.096507 &  1.237184 \\
1600   &    0.266663 &  1.237440 \\
3200   &    0.833163 &  1.237536 \\
6400   &    2.604832 &  1.237619 \\
12800  &    9.203689 &  1.237658 \\
25600  &   33.772759 &  1.237674 \\
51200  &  131.556080 &  1.237683 \\
102400 &  527.379351 &  1.237687 \\
\bottomrule
\end{tabular}
\end{center}
\caption{
American Put Pricing Results by the CRR Tree.
The first column is the number of time steps, the second column is the running time in seconds,
and the third column is the computed option value.
}\label{tab:CRR}
\end{figure}




First we compare the pricing results of the na\"ive control variate modification of the CRR tree algorithm shown in Figure \ref{algo:CRR_CV},
and efficient modification in Figure \ref{algo:CRR_CV_fast}.
Recall that the na\"ive modification is inefficient because the Black-Scholes formula is evaluated at every node.
The efficiency can be enhanced by building a tree of the early exercise boundary which only evaluates the Black-Scholes formula around the early exercise boundary.
That is the enhanced algorithm in Figure \ref{algo:CRR_CV_fast}.
In theory the two pricing algorithms should give exactly the same results
because the efficiency is achieved by saving unnecessary Black-Scholes formula evaluation.
Figure \ref{tab:CVfastCVslow} shows that this is true.
While the pricing results are always identical, the computational time is very different.
When the number of time steps is 800, the enhanced algorithm is more than 100 times faster than the na\"ive modification.
From now on, by control variate modification we always refer to the enhanced efficient algorithm in Figure \ref{algo:CRR_CV_fast}.




\begin{figure}[!t]
\begin{center}
\begin{tabular}{rrrrr}
\toprule
{} &       \multicolumn{2}{c}{Efficient CV}           &     \multicolumn{2}{c}{Na\"ive CV}     \\
{$n$} &   CPU (s) &    result &    CPU (s) &    result \\
\midrule
100 &  0.027450 &  1.235274 &   0.639714 &  1.235274 \\
200 &  0.054531 &  1.236470 &   2.508056 &  1.236470 \\
400 &  0.109282 &  1.237030 &   9.942845 &  1.237030 \\
800 &  0.218897 &  1.237324 &  39.198139 &  1.237324 \\
\bottomrule
\end{tabular}
\end{center}
\caption{
Comparison of the Na\"ive and the Efficient Control Variate Modification of a CRR Tree.
This is a comparison of the algorithms in Figures \ref{algo:CRR_CV} and \ref{algo:CRR_CV_fast}.
The two algorithms gives the same pricing results, but with very different efficiency.
$n$ is the number of time steps, CPU(s) the running time in seconds, the results the computed option value.
}\label{tab:CVfastCVslow}
\end{figure}




Next we compare the pricing results of the traditional CRR tree, its control variate modification, and Hull and While's algorithm.
The computational times and the absolute errors are shown in Figure \ref{tab:Tree} for different numbers of time steps.
The pricing results by Hull and White's algorithm are less accurate for small $n$, but for $n>400$, the accuracy is always better than the CRR tree.
Given the same number of time steps, the control variate modification always gives better pricing results than both the CRR tree and Hull and White's algorithm.
The only exception is when $n = 3200$ Hull and White's algorithm gives a bit more accuracy than the control variate modification.
This is in fact an example of the instability of Hull and White's algorithm, which will be discussed later.




The running times listed in Figure \ref{tab:Tree} shows that
for small $n$ the control variate modification is slower than both the CRR tree and Hull and White's algorithm.
This is because the CRR tree and Hull and White's algorithm only perform the backward induction step at each node,
which is a much computationally cheaper operation than the Black-Scholes formula evaluation needed in our control variate modification.
This difference in speed is less significant when $n$ goes large.
The reason is that in the control variate modification the Black-Scholes formula is only evaluated around the early exercise boundary,
so it is only evaluated $O(n)$ times,
while in all three algorithms the backward induction at all nodes requires $O(n^2)$ computations.
When $n>1600$, the CV-enhanced algorithm starts to go faster than Hull and White's.
When $n>3200$, it gives the best numerical results, in terms of both efficiency and accuracy.





\begin{figure}[!t]
\begin{center}
\begin{tabular}{rrrrrrr}
\toprule
{} &       CRR &            &        CV &           &        HW &            \\
{$n$} & Abs error &    CPU (s) & Abs error &   CPU (s) & Abs error &    CPU (s) \\
\midrule
100   &  0.003601 &   0.007769 &  0.002413 &  0.034853 &  0.004081 &   0.011390 \\
200   &  0.001796 &   0.016581 &  0.001217 &  0.056165 &  0.002048 &   0.024034 \\
400   &  0.000939 &   0.036914 &  0.000657 &  0.108347 &  0.000984 &   0.052816 \\
800   &  0.000503 &   0.093100 &  0.000363 &  0.222466 &  0.000458 &   0.125053 \\
1600  &  0.000247 &   0.251381 &  0.000178 &  0.465793 &  0.000233 &   0.319739 \\
3200  &  0.000151 &   0.788970 &  0.000116 &  0.888058 &  0.000089 &   0.915061 \\
6400  &  0.000068 &   2.589824 &  0.000050 &  1.831340 &  0.000053 &   2.849856 \\
12800 &  0.000029 &   9.503834 &  0.000021 &  3.861430 &  0.000031 &   9.931539 \\
25600 &  0.000013 &  34.190770 &  0.000009 &  7.750417 &  0.000017 &  36.476493 \\
\bottomrule
\end{tabular}
\end{center}
\caption{
Efficiency and Accuracy Comparison of Various Binominal Tree Algorithms.
CRR refers to the traditional CRR tree algorithm in Figure \ref{algo:CRR_American},
HW is Hull and White's algorithm in Figure \ref{algo:CRR_Hull},
and CV is our control variate modification of the CRR tree, the algorithm in Figure \ref{algo:CRR_CV_fast}.
}\label{tab:Tree}
\end{figure}




Figure \ref{fig:Tree} is a picture of the pricing result against the number of time steps.
As shown in the figure, while the results by all three algorithms converge to the benchmark 1.237687,
our control variate modification is the most stable.
This is because our modified algorithm is a tree of the early exercise premiums instead of the put values.
While the put value has a discontinuity in the first derivative at $S_T = K$,
the early exercise boundary is just a zero function at maturity.
In the literature, the discontinuity in the first derivative has been identified as the source of the oscillatory convergence.
See for example \citet{leisen1998pricing}, \citet{klassen2001simple}.
\citet{klassen2001simple} suggests using
$$
u = e^{\sigma\sqrt{T/n} + \frac{\log(K/S_0)}{n}}, \qquad d = e^{-\sigma\sqrt{T/n} + \frac{\log(K/S_0)}{n}}
$$
instead of the CRR specification (\ref{eq:udCRR}),
and shows that this choice of $u$, $d$ values can eliminate the oscillation of the numerical results.
With a tree of the early exercise premiums, there is no discontinuity of the first derivative
and therefore no need to change the CRR specification.
That is why the convergence of our control variate modification shown in Figure \ref{fig:Tree} seems more stable than other algorithms.






\begin{figure}[!t]
\begin{center}
    \parbox[t]{0.8\textwidth}{
        \centerline{\epsfig{figure=eps/Tree.eps, width=0.8\textwidth}}
    }\hfill
\end{center}
\caption{
Pricing Results Against Number of Time Steps
}\label{fig:Tree}
\end{figure}








%
%\section{Numerical Results}\label{sec:nresult}
%
%
%The numerical results of the plain and the modified Carr and Madan's algorithms are shown in Figure \ref{fig:CMnum},
%where the target model is the Heston model.
%The two modified algorithms use the Black-Scholes model and the model based on the Edgeworth expansion as the control variates, respectively.
%From Figure \ref{fig:CMnum} we can see that the Edgeworth expansion leads us to a better control.
%
%
%The numerical results of the plain and the modified convolution Bermudan put pricing algorithms are shown in Figure \ref{tab:Berm} and \ref{fig:CONVnum}.
%The two modified algorithms use the Black-Scholes formula and (\ref{eq:cvem}) as the control variates, respectively.
%We have seen from Figure \ref{fig:cvBS} and \ref{fig:cvem} that, (\ref{eq:cvem}) behaves more like the value of Bermudan puts.
%As shown in the table in Figure \ref{tab:Berm}, given the same computational time,
%using (\ref{eq:cvem}) as a control gives much more accuracy than using the Black-Scholes formula as a control.
%Given the same number of grid points, the modified algorithms clearly have better accuracy, as shown in Figure \ref{fig:CONVnum},
%but a modified algorithm is more time-consuming compared to the plain convolution algorithm.
%This is because in each loop it takes time to prepare the grid functions of the control and the difference,
%but even that, the time spent is worth it because with a control variate we can get more accuracy with fewer grid points.
%Indeed, as shown in the table in Figure \ref{tab:Berm}, given the same accuracy level,
%the modified algorithm with the Black-Scholes formula as a control is about twice as faster as the plain convolution algorithm.




\section{Conclusion}
In a general framework we extend the CV technique to enhance various numerical option pricing algorithms.
Unlike in the Monte Carlo simulation where the CV is applied to the algorithm as a whole,
in our framework CVs can be applied to some steps in the algorithm, which makes the framework more general.
An algorithm as a whole might not satisfy the conditions under which it can benefit from the use of CVs, but some of its individual parts may.
As examples we demonstrate how to apply the CV technique to enhance Carr and Madan's algorithm,
the convolution-based pricing algorithms, and binomial trees.
Numerical results show that the CV-enhanced algorithms are more efficient and accurate than the original algorithms in all cases considered.




One advantage the general framework inherits from the traditional CV technique is the flexibility of choosing CV.
Using Carr and Madan's algorithm and the convolution-based Bermudan puts pricing algorithm as examples,
we show that different choices of CVs can lead to different pricing accuracy.
By choosing better CVs, it is possible to design more accurate pricing algorithms in this new framework.
Moreover, in some cases, like the convolution-based pricing algorithms, the target to be approximated by CVs is easy to visualize,
which makes it easier to find good CVs than the Monte Carlo simulation.










%%%%%%%%%%%%%%%%%%%%%%%%%%%%%%%%%%%%%%%%%%%%%%%%%%%%%%%%%%%%%%%%%%%%%%

%
%
%
%
%
%
%Suppose $D<K$.
%\begin{eqnarray}
%&& e^{-r(T_2 - T_1)} E\left[C(\text{Black-Scholes put at $T_2$})1_{\{S_{T_2}>D\}} + (K-S_{T_2})1_{\{S_{T_2}\leq D\}} \Big| \mathcal F_{T_1}\right]\nonumber\\
%&=& C e^{-r(T_2 - T_1)} E\left[ e^{-r(T - T_2)} E\left[(K-S_{T})^+|\mathcal F_{T_2}\right] 1_{\{S_{T_2}>D\}} \Big| \mathcal F_{T_1}\right] \label{eq:CV:BSput}\\
%&&+ e^{-r(T_2 - T_1)} E\left[(K-S_{T_2})1_{\{S_{T_2}\leq D\}} \Big| \mathcal F_{T_1}\right],  \label{eq:CV:exercise}
%\end{eqnarray}
%where (\ref{eq:CV:BSput}) can be written as
%\begin{eqnarray*}
%&& C e^{-r(T - T_1)} E\left[ E\left[(K-S_{T})^+ 1_{\{S_{T_2}>D\}} \Big|\mathcal F_{T_2}\right]  \Big| \mathcal F_{T_1}\right]\\
%&=& C e^{-r(T - T_1)} E\left[ (K-S_{T})^+ 1_{\{S_{T_2}>D\}} \Big| \mathcal F_{T_1}\right],
%\end{eqnarray*}
%which is similar to a two-step down-and-out put option and has a closed form formula
%\begin{eqnarray*}
%&&C\left(K e^{-r(T - T_1)} N\left( \frac{\log\frac{S_{T_1}}{D} + \left(r - \frac{\sigma^2}{2}\right)(T_2 - T_1)}{\sigma\sqrt{T_2 - T_1}},
%-\frac{\log\frac{S_{T_1}}{K} + \left(r - \frac{\sigma^2}{2}\right)(T - T_1)}{\sigma\sqrt{T - T_1}},
%-\sqrt{\frac{T_2 - T_1}{T-T_1}}   \right)\right.\\
%&&\qquad\left. -S_{T_1}N\left( \frac{\log\frac{S_{T_1}}{D} + \left(r + \frac{\sigma^2}{2}\right)(T_2 - T_1)}{\sigma\sqrt{T_2 - T_1}},
%-\frac{\log\frac{S_{T_1}}{K} + \left(r + \frac{\sigma^2}{2}\right)(T - T_1)}{\sigma\sqrt{T - T_1}},
%-\sqrt{\frac{T_2 - T_1}{T-T_1}}   \right)\right),
%\end{eqnarray*}
%where $N(x, y, \rho)$ is the two-dimensional standard normal distribution function with correlation $\rho$.
%Eq.\,(\ref{eq:CV:exercise}) is similar to a vanilla put option with maturity $T_2$.
%The closed form formula is
%$$
%K e^{-r(T_2 - T_1)} N\left( - \frac{\log\frac{S_{T_1}}{D} + \left(r - \frac{\sigma^2}{2}\right)(T_2 - T_1)}{\sigma\sqrt{T_2 - T_1}} \right)
%-S_{T_1}N\left( - \frac{\log\frac{S_{T_1}}{D} + \left(r + \frac{\sigma^2}{2}\right)(T_2 - T_1)}{\sigma\sqrt{T_2 - T_1}} \right),
%$$
%where $N(x)$ is the one-dimensional standard normal distribution function.
%
%\pagebreak
%\section{Numerical Results}
%\begin{multicols}{2}
%UO Richardson extrapolation, MAE
%$$
%\begin{array}{l|ll}
%n & \text{error} & \text{CPU} \\
%\hline
%200 & 0.0150151 & 0.022\\
%400 & 0.0147935 & 0.039\\
%800 & 0.000839207 & 0.08\\
%1600 & 0.000849827 & 0.16\\
%3200 & 0.000852532 & 0.3\\
%6400 & 0.000853829 & 0.581\\
%12800 & 0.000122063 & 1.212\\
%25600 & 0.000122009 & 2.376\\
%51200 & 0.000121992 & 4.731\\
%102400 & 1.20147e-009 & 9.516\\
%204800 & 0 & 20.053\\
%\end{array}
%$$
%
%Bermudan Richardson extrapolation, MAE
%$$
%\begin{array}{l|ll}
%n & \text{error} & \text{CPU} \\
%\hline
%200 & 0.010028 & 0.02\\
%400 & 0.00265791 & 0.037\\
%800 & 0.00062668 & 0.079\\
%1600 & 0.00015667 & 0.15\\
%3200 & 3.91705e-005 & 0.308\\
%6400 & 0.00257759 & 0.593\\
%12800 & 2.45152e-006 & 1.235\\
%25600 & 0.00232802 & 2.425\\
%51200 & 1.56691e-007 & 4.853\\
%102400 & 4.19352e-008 & 9.688\\
%204800 & 0 & 20.398\\
%\end{array}
%$$
%
%UO MAE using Richardson with $m=10240$ as a benchmark, where the grid points are not adjusted
%
%$$
%\begin{array}{l|ll}
%n & \text{error} & \text{CPU} \\
%\hline
%200 & 0.0168317 & 0.006\\
%400 & 0.000935918 & 0.013\\
%800 & 0.0069257 & 0.025\\
%1600 & 0.00304248 & 0.053\\
%3200 & 0.0010961 & 0.088\\
%6400 & 0.000121727 & 0.201\\
%\end{array}
%$$
%
%UO MAE with adjusted grid using Richardson with $m=10240$ as a benchmark.
%
%$$
%\begin{array}{l|ll}
%n & \text{error} & \text{CPU} \\
%\hline
%200 & 0.0205927 & 0.006\\
%400 & 0.0102851 & 0.012\\
%800 & 0.00507752 & 0.032\\
%1600 & 0.00253768 & 0.064\\
%3200 & 0.00126844 & 0.092\\
%6400 & 0.000634018 & 0.207\\
%\end{array}
%$$
%
%Bermudan MAE using Richardson with $m=10240$ as a benchmark.
%
%$$
%\begin{array}{l|ll}
%n & \text{error} & \text{CPU} \\
%\hline
%200 & 1.00957 & 0.006\\
%400 & 0.504801 & 0.012\\
%800 & 0.252404 & 0.024\\
%1600 & 0.126203 & 0.053\\
%3200 & 0.0631019 & 0.096\\
%6400 & 0.031551 & 0.207\\
%\end{array}
%$$
%\end{multicols}
%
%
%\section{Bermudan}
%
%
%   nsBS�GBS put as CV
%   em�Gpt value and derivative on the left
%   BS�Ga multiple of BS put CV
%   emc�Gtwo pts on the left and right
%
%
%
%
%
%
%
%\begin{figure}[!h]
%\begin{center}
%    \parbox[t]{0.45\textwidth}{
%        \centerline{\epsfig{figure=eps/BSCV.eps, width=0.45\textwidth}}
%    }
%    \parbox[t]{0.45\textwidth}{
%        \centerline{\epsfig{figure=eps/BSDiff.eps, width=0.45\textwidth}}
%    }
%\end{center}
%\caption{BermCVBS}
%\label{fig:BermCVBS}
%\end{figure}
%
%
%
%\begin{figure}[!h]
%\begin{center}
%    \parbox[t]{0.45\textwidth}{
%        \centerline{\epsfig{figure=eps/emcCV.eps, width=0.45\textwidth}}
%    }
%    \parbox[t]{0.45\textwidth}{
%        \centerline{\epsfig{figure=eps/emcDiff.eps, width=0.45\textwidth}}
%    }
%\end{center}
%\caption{BermCVemc}
%\label{fig:BermCVemc}
%\end{figure}
%
%
%
%
%$$
%\begin{array}{l|llll}
%n & \text{BS err} & \text{BS CPU} & \text{emc err} & \text{emc CPU} \\
%\hline
%200 & 0.00160123 & 0.047 & 0.00165544 & 0.029\\
%400 & 0.000411737 & 0.091 & 0.00032464 & 0.056\\
%800 & 9.95462e-005 & 0.183 & 9.05748e-005 & 0.106\\
%1600 & 2.53013e-005 & 0.371 & 2.06212e-005 & 0.213\\
%3200 & 6.40176e-006 & 0.733 & 4.39801e-006 & 0.43\\
%6400 & 1.55036e-006 & 1.502 & 1.66559e-006 & 0.885\\
%12800 & 3.88023e-007 & 2.994 & 6.45726e-007 & 1.766\\
%25600 & 9.9298e-008 & 6.174 & 1.78323e-007 & 3.637\\
%\end{array}
%$$
%


%
%Fitting line equation of Berm: $y = -0.999988x+2.30515$.
%
%Fitting line equation of BermCVnsBS: $y = -1.0485x+1.54004$.
%
%Fitting line equation of BermCVem: $y = -1.04381x+0.0222945$.
%
%Fitting line equation of BermCVBS: $y = -2.00111x+1.81333$.
%
%Fitting line equation of BermCVemc: $y = -1.86495x+1.37219$.
%


\appendix






\section{Heston Model and Its Characteristic Function}
\label{appendix:heston}
Under the Heston stochastic volatility model, the price process $S_t$ of the underlying asset of an option is modeled by
\begin{eqnarray*}
\begin{cases}
dS_t &= r S_tdt + \sqrt{V_t} S_td\tilde W^S_t\\
dV_t &= \kappa(\theta - V_t) dt + \sigma\sqrt{V_t} d\tilde W^V_t\\
\end{cases},
\end{eqnarray*}
where $r>0$ is the risk-free interest rate,
$\tilde W^S_t$ and $\tilde W^V_t$ are standard Brownian motions with correlation $d\tilde W^S_td\tilde W^V_t = \rho dt$, $\rho\in(-1, 1)$,
and the parameters $\kappa>0$, $\theta>0$, $\sigma>0$.
The characteristic function of the log-return $X_T = \log(S_T/S_0)$ is known in closed form as
\begin{eqnarray*}
\phi_T(u) = \tilde E\left[e^{iuX_T}\right] = e^{C(u, T)+D(u, T)V_0},%\label{eq:chfHeston}
\end{eqnarray*}
where
\begin{eqnarray*}
C(u, \tau) &=& iru\tau + \frac{\kappa \theta  }{\sigma^2} \left( \left(\kappa -i\rho\sigma u - d\right)\tau - 2\log\left[\frac{1-g e^{-d\tau}}{1-g}\right] \right),\\
D(u, \tau) &=& \frac{ \kappa -i\rho\sigma u - d}{\sigma^2} \left[\frac{1-e^{-d\tau}}{1 - g e^{-d\tau}}\right],\\
g &=& \frac{ \kappa -i\rho\sigma u - d}{\kappa -i\rho\sigma u + d}~,\\
d &=& \sqrt{\left(\kappa -i\rho\sigma u \right)^2 + \sigma^2\left(iu+ u^2\right)}.
\end{eqnarray*}
This is a different formulation from the characteristic function formula first derived in the appendix of \citet{heston1993closed}.
\citet{lord2010complex} prove that with this formulation
taking the principal branch of the the complex logarithm in $C(\tau)$ will always give us the correct characteristic function value.

%
%
%
%As mentioned in Section \ref{sec:find_cumulants},
%since the characteristic function is of the form $\phi_T(u) = e^{\psi_T(u)}$
%with the characteristic exponent $\psi_T(u) = C(u, T) + D(u, T)V_0$,
%to find the first few cumulants $m_1, m_2, m_3, \ldots$ of the Heston model we can use the identity
%$$
%\sum_{n=1}^\infty \frac{m_n}{n!}u^n = \psi_T(-i u) = C(-iu, T) + D(-iu, T)V_0.
%$$
%One just needs to find the power series of $\psi_T(-i u)$.
%We find the power series of $C(-iu, T)$ and $D(-iu, T)V_0$ separately and then find the sum.
%Specifically, we write
%$$
%\sum_{n=1}^\infty \frac{m_n}{n!}u^n = \sum_{n=1}^\infty \frac{(c_n + d_nV_0)}{n!}u^n = C(-iu, T) + D(-iu, T)V_0,
%$$
%where $c_n$ and $d_n$ are from the coefficients of the power series of $C(-iu, T)$ and $D(-iu, T)$, respectively.
%Direct computation to expand $D(-iu, T)$ shows that
%\begin{eqnarray*}
%d_1 &=& \frac{e^{-\kappa T} - 1}{2\kappa}, \\
%d_2 &=& \frac{1}{4\kappa^3}\left( p_{20}(\sigma) + 2 e^{-\kappa T}p_{21}(\sigma) -\sigma^2 e^{-2\kappa T}\right), \\
%d_3 &=& \frac{3}{16\kappa^5}\left( 2 p_{30}(\sigma) + e^{-\kappa T}p_{31}(\sigma) + 2 e^{-2\kappa T}p_{32}(\sigma) + \sigma^3 e^{-3\kappa T} \right), \\
%d_4 &=& \frac{1}{16\kappa^7}\left( 3p_{40}(\sigma) + 2e^{-\kappa T}p_{41}(\sigma) - 12e^{-2\kappa T}p_{42}(\sigma) - 6e^{-3\kappa T}p_{43}(\sigma) - 3\sigma^4  \right),
%\end{eqnarray*}
%where $p_{i, j}$'s are polynomials of $\sigma$ given below.
%The polynomials in $d_2$ are
%\begin{eqnarray*}
%p_{21}(\sigma) &=& (-2 \kappa ^2) + \sigma (2 \kappa  \rho +2 \kappa ^2 \rho  T) + \sigma^2 (-\kappa T), \\
%p_{20}(\sigma) &=& (4 \kappa ^2) + \sigma (-4 \kappa  \rho ) + \sigma^2.
%\end{eqnarray*}
%The polynomials in $d_3$ are
%\begin{eqnarray*}
%p_{32}(\sigma) &=& \sigma^2 (4 \kappa ^2) +  \sigma^3 (-6 \kappa  \rho -4 \kappa ^2 \rho  T) +  \sigma^4 (2 \kappa  T+1), \\
%p_{31}(\sigma) &=& \sigma (-16 \kappa ^3 \rho -16 \kappa ^4 \rho  T) + \sigma^2 (16 \kappa ^2 \rho ^2+8 \kappa ^4 \rho ^2 T^2+16 \kappa ^3 \rho ^2 T+16 \kappa ^3 T)\\
%&&+~ \sigma^3 (-8 \kappa ^3 \rho  T^2-16 \kappa ^2 \rho  T) + \sigma^4 (2 \kappa ^2 T^2+2 \kappa  T-1), \\
%p_{30}(\sigma) &=& \sigma (8 \kappa ^3 \rho ) + \sigma^2 (-8 \kappa ^2 \rho ^2-4 \kappa ^2) + \sigma^3 (6 \kappa  \rho ) - \sigma^4,
%\end{eqnarray*}
%and in $d_4$
%\begin{eqnarray*}
%p_{43}(\sigma) &=& \sigma^4 (6 \kappa ^2) + \sigma^5 (-10 \kappa  \rho -6 \kappa ^2 \rho  T) + \sigma^6 (3 \kappa  T+2),\\
%p_{42}(\sigma) &=& \sigma^2 (4 \kappa ^4) + \sigma^3 (-24 \kappa ^3 \rho -16 \kappa ^4 \rho  T) + \sigma^4 (24 \kappa ^2 \rho ^2+6 \kappa ^2+8 \kappa ^4 \rho ^2 T^2+24 \kappa ^3 \rho ^2 T+12 \kappa ^3 T)\\
%&&+~ \sigma^5 (-10 \kappa  \rho -8 \kappa ^3 \rho  T^2-20 \kappa ^2 \rho  T) + \sigma^6 (2 \kappa ^2 T^2+3 \kappa  T+1),\\
%p_{41}(\sigma) &=& \sigma^2 (-96 \kappa ^4 \rho ^2-48 \kappa ^6 \rho ^2 T^2-96 \kappa ^5 \rho ^2 T-48 \kappa ^5 T)\\
%&&+~ \sigma^3 (96 \kappa ^3 \rho ^3+16 \kappa ^6 \rho ^3 T^3+48 \kappa ^5 \rho ^3 T^2+96 \kappa ^5 \rho  T^2+96 \kappa ^4 \rho ^3 T+192 \kappa ^4 \rho  T)\\
%&&+~ \sigma^4 (18 \kappa ^2-24 \kappa ^5 \rho ^2 T^3-96 \kappa ^4 \rho ^2 T^2-36 \kappa ^4 T^2-144 \kappa ^3 \rho ^2 T-36 \kappa ^3 T)\\
%&&+~ \sigma^5 (-30 \kappa  \rho +12 \kappa ^4 \rho  T^3+48 \kappa ^3 \rho  T^2+42 \kappa ^2 \rho  T)\\
%&&+~ \sigma^6 (-2 \kappa ^3 T^3-6 \kappa ^2 T^2-3 \kappa  T+6), \\
%p_{40}(\sigma) &=& \sigma^2 (64 \kappa ^4 \rho ^2+16 \kappa ^4) + \sigma^3 (-64 \kappa ^3 \rho ^3-96 \kappa ^3 \rho ) + \sigma^4 (96 \kappa ^2 \rho ^2+24 \kappa ^2) + \sigma^5 (-40 \kappa  \rho ) + 5\sigma^6.
%\end{eqnarray*}
%
%
%
%To find the coefficients $c_n$, we use the fact
%$$
%C(\tau) = \int_0^\tau iru + \kappa \theta D(s)\,ds
%$$
%from the derivation of the characteristic function.
%This identity can be found in both \citet{heston1993closed} and \citet{lord2010complex}.
%
%
%
%


%\bibliographystyle{plain}
\bibliographystyle{chicago}
\bibliography{GCV_db}








\end{document}
