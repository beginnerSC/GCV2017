\documentclass[12pt]{article}


%\title{Improving Pricing Accuracy of Various Numerical Methods with the Control Variate Technique}
\title{Extension of the Control Variate Technique to Option Pricing beyond Monte Carlo Simulation}
\date{}%\today}

%replace Levy by L\'evy


\usepackage[latin1]{inputenc}
\usepackage[margin=0.8in]{geometry}
\usepackage{graphicx}
% \usepackage{latexsym}
% \usepackage{algorithm}
% \usepackage{algorithmic}
\usepackage{epsfig}
\usepackage{color}
% \usepackage{rotating}
\usepackage{natbib}
\usepackage{algorithm}
\usepackage{algorithmic}
\usepackage{amsmath}
\usepackage{amsfonts}
\usepackage{amssymb}
\usepackage{hyperref}
\usepackage{multicol}
\usepackage{setspace}
\usepackage{booktabs}
\usepackage{framed}


\newtheorem{theorem}{Theorem}[section]
\newtheorem{lemma}[theorem]{Lemma}
\newtheorem{proposition}[theorem]{Proposition}
\newtheorem{corollary}[theorem]{Corollary}

\newenvironment{proof}[1][Proof]{\begin{trivlist}
\item[\hskip \labelsep {\bfseries #1}]}{\end{trivlist}}
\newenvironment{definition}[1][Definition]{\begin{trivlist}
\item[\hskip \labelsep {\bfseries #1}]}{\end{trivlist}}
\newenvironment{example}[1][Example]{\begin{trivlist}
\item[\hskip \labelsep {\bfseries #1}]}{\end{trivlist}}
\newenvironment{remark}[1][Remark]{\begin{trivlist}
\item[\hskip \labelsep {\bfseries #1}]}{\end{trivlist}}

\newcommand{\qed}{\nobreak \ifvmode \relax \else
      \ifdim\lastskip<1.5em \hskip-\lastskip
      \hskip1.5em plus0em minus0.5em \fi \nobreak
      \vrule height0.75em width0.5em depth0.25em\fi}




%\author{Student: Chun-Yuan Chiu\qquad Advisor: Dr. Tian-Shyr Dai}



\begin{document}
\maketitle



$$$$

\begin{abstract}
Although mostly used in connection with the Monte Carlo simulation,
we find that the control-variate (CV) technique has applications in other numerical algorithms in option pricing.
This paper studies the condition under which a numerical method can benefit from the CV technique
and the kind of approximation that can serve as a CV.
We demonstrate the ideas with Fourier transform-based pricing algorithms, convolution-based pricing algorithms, and binomial trees.
Numerical results are provided to show the efficiency and accuracy of the CV-enhanced algorithms.
\end{abstract}

%\newpage
%\tableofcontents
%\newpage


%\doublespacing


\section{Introduction}



Option pricing is key to the capital markets.
This is because options have large daily trading volumes,
and many important quantities such as default probabilities and the values of many popular financial derivatives can be estimated from their prices.
Under the popular Black-Scholes model, closed-form pricing formulas exist for some European options.
They include call and put options, binary options and some path-dependent options like barrier options.
But such options as American options and Asian options do not admit of closed-form formulas.
More complex and realistic models than the Black-Scholes model include the stochastic-volatility models developed by, for example, \citet{heston1993closed}
and models based on the L\'evy-type jump process by \citet{merton1976option}, \citet{kou2002jump} and \citet{carr2002fine}.
Option pricing under those models are usually not analytically tractable;
even the simplest options lack closed-form pricing formulas.
When pricing formulas are not available, closed-form approximation formulas and numerical algorithms are the two alternatives.
But each has its own advantages and disadvantages.
The former can be efficient but inaccurate,
whereas the latter can be time-consuming although their accuracy improves with increasing computational resources.




The control-variate (CV) technique is standard in reducing the variance of Monte Carlo simulation \citep{Glasserman2004MCM}.
By combining simulation with a closed-form formula,
CV is often able to reduce the overall variance and hence improve upon the efficiency of the original algorithm.
A good example from \citet{kemna1990pricing} is pricing arithmetic Asian options by simulation with the geometric Asian option as the CV,
which lowers the variance dramatically.




Although mostly used in connection with the Monte Carlo simulation, the CV technique can in fact be used to enhance almost any numerical method.
This paper presents a general framework of the CV technique.
It lays down the conditions a CV and the numerical method should jointly satisfy for the CV technique to work.
As illustrations, we demonstrate how to apply it to (1) Fourier transform (FT)-based pricing algorithms, (2) convolution-based algorithms, and (3) binomial trees.




The CV technique has been applied to binomial trees by \citet{hull1988use}.
\citet{tankov2003financial} also derives an FT-based pricing algorithm that implicitly uses a CV.
Other than the above works, the CV technique seems to have been always associated with the Monte Carlo simulation.
This paper will substantially broaden the applicability of the CV technique beyond Monte Carlo.
We will apply it to FT- and convolution-based pricing algorithms.
FT-based pricing algorithms and convolution-based algorithms are widely used in the literature (e.g., \citet{benhamou2002fast}, \citet{carr1999option}, \citet{lord2008fast}, \citet{schmelzle2010option}).
An FT-based pricing algorithm can deal with European options under any model with a closed-form characteristic function,
which includes some stochastic-volatility models and all exponential L\'evy models (see \citet{tankov2003financial}).
Convolution-based pricing algorithms can be used to evaluate many path-dependent options.
We also discuss how to choose the CVs.



In the literature, the CV technique always runs alongside the target algorithms (e.g., \citet{hull1988use}).
In the case of convolution-based algorithms and binomial trees, in contrast, our CV technique will run {\em within} the target algorithms.
In fact, several different CVs can be used in one algorithm.
Even if an algorithm as a whole does not satisfy the aforesaid conditions under which it can benefit from the use of CVs, some parts may.




This paper is organized as follows.
Section 2 explains the CV technique and the conditions under which it can be used.
Section 3 discusses how to apply the CV technique to FT-based pricing algorithms and a methodology to pick good CVs.
Section 4 uses the convolution-based pricing algorithms to illustrate how to apply the CV technique within them.
Section 5 demonstrates how to apply the CV technique to binomial trees.
Numerical results are given at the end of each section to show the improvements over the original algorithms.
Section 6 concludes.




\section{Control Variates as a General Method}
\label{sec:gcv}




The CV technique is an effective variance-reduction technique for the Monte Carlo method.
That it can be extended to many other numerical pricing methods is a major focus of this paper.
To better explain the general methodology, we first review its use in the Monte Carlo method.
Denote by $\mathsf E[Y]$ the expectation of a random variable $Y$,
and $\mathbb E[Y]$ the Monte Carlo estimator $\sum_{i=1}^nY(\omega_i)/n$, where $\omega_i$ are random samples.
The crude Monte Carlo method uses $\mathbb E[Y]$ as an estimate of $\mathsf E[Y]$.
The error of this estimate depends on the variance of $\mathbb E[Y]$.
The smaller the variance is, the more narrowly bounded the estimate.



The CV technique is based on the decomposition $\mathsf E[Y] = \mathsf E[Y-C] + \mathsf E[C]$.
The idea is that if one can pick a random variable $C$ such that $\mathsf E[C]$ is known in closed form
and the variance of $Y-C$ is smaller than the the variance of $Y$,
then one can use the estimate
\begin{eqnarray*}
\mathsf E[Y] &\approx& \mathbb E[Y-C] + \mathsf E[C] \\
            &=& \left(\sum_{i=1}^n(Y(\omega_i) - C(\omega_i))/n\right) + \mathsf E[C],
\end{eqnarray*}
which has a smaller variance compared to $\mathbb E[Y]$ and hence is a better estimate.
The random variable $C$ is called the CV, or control for short, and $Y-C$ is called the difference.
In other words, in the CV technique $Y$ is written as the sum of the difference and the control.
Since $\mathsf E[C]$ is known, the Monte Carlo estimator is applied to the difference.



We can extend the above idea to other numerical methods and under a more general setting.
Suppose we are to evaluate $\mathsf T(f)$, a linear operator $\mathsf T$ acting on the function $f$, and have access to a numerical approximation $\mathbb T(f)$ with error $|\mathbb T(f) - \mathsf T(f)|$.
An example: $\mathsf T$ is the integral operator $\int_a^b$ so $\mathsf T(f) = \int_a^b f\,dx$, while $\mathbb T$ is the trapezoidal rule approximation of $\mathsf T$.
The linearity of $\mathsf T$ is important for any CV method to work
because it guarantees $\mathsf T(f) = \mathsf T(f-g) + \mathsf T(g)$ for any function $g$, the control.
% For the general CV technique, w
We shall also pick a $g$ whose $\mathsf T(g)$ is known in closed form
and the error of $\mathbb T(f)$ is smaller than that of $\mathbb T(f-g)$.
Then,
\begin{eqnarray}
%\mathsf T(f) \approx
\mathbb T(f-g) + \mathsf T(g)\label{eq:GCVapprox}
\end{eqnarray}
would be a better approximation than $\mathbb T(f)$.
%The only additional assumption for our general method to work is the linearity of $\mathsf T$, which is needed in $\mathsf T(f) = \mathsf T(f-g) + \mathsf T(g)$.
This is a nutshell in our general CV technique.
Figure \ref{fig:CV_GCV_comp} compares two versions of the CV technique.



\begin{figure}[t!]
\rule{\textwidth}{0.4pt}\\
{\scriptsize
\parbox[t]{0.43\textwidth}{
Control Variate Method in Monte Carlo
\begin{itemize}
    \item Want to evaluate $\mathsf E[Y]$
        \begin{itemize}
            \item[]
        \end{itemize}
    \item Monte Carlo estimator
        \begin{eqnarray*}
        \textstyle \mathsf E[Y] \approx \mathbb E[Y] = \sum_{i=1}^n Y(\omega_i)/n
        \end{eqnarray*}
    \item Control variate method
        \begin{eqnarray*}
        \mathsf E[Y] = \mathsf E[Y-C] + \mathsf E[C]
        \end{eqnarray*}
    \begin{itemize}
        \item $C$ is the CV
        \item $Y-C$ is the difference
    \end{itemize}
    \item Find a $C$ such that
    \begin{itemize}
        \item $\text{Var}(Y-C) < \text{Var}(Y)$
        \item $\mathsf E[C]$ is known analytically
    \end{itemize}
    \item Apply the estimator on $Y-C$ instead of $Y$
\end{itemize}
}
\parbox[t]{0.57\textwidth}{
General Control Variate Method
\begin{itemize}
    \item Want to evaluate $\mathsf T(f)$
    \begin{itemize}
        \item linear operator $\mathsf T$ on any function $f$
    \end{itemize}
    \item Approximating operator
        \begin{eqnarray*}
        \mathsf T(f) \approx \mathbb T(f)
        \end{eqnarray*}
    \item General CV method
        \begin{eqnarray*}
        \mathsf T(f) = \mathsf T(f-g) + \mathsf T(g)
        \end{eqnarray*}
    \begin{itemize}
        \item $g$ is the CV
        \item $f-g$ is the difference
    \end{itemize}
    \item Find a $g$ such that
    \begin{itemize}
        \item error of $\mathbb T(f)$ $<$ error of $\mathbb T(f-g)$
        \item $\mathsf T(g)$ is known analytically
    \end{itemize}
    \item Apply the approximating operator $\mathbb T$ on $f-g$ instead of $f$
\end{itemize}
}}
\rule{\textwidth}{0.4pt}
\caption{Control Variate Method in Monte Carlo and the General Version}\label{fig:CV_GCV_comp}
\end{figure}












\section{FT-based Pricing Algorithms}
FT-based option pricing algorithms are popular.
An important reason is that many option pricing models have closed-form characteristic functions but no pricing formulas.
An FT-based pricing algorithm first derives the FT of the option price or tail probabilities, which is in closed-form in terms of the characteristic function in most cases.
An inverse FT is then performed numerically to back out the option price.
When the target model has a closed-form characteristic function, the pricing algorithm is a numerical integration to evaluate the inverse FT, with the closed-form FT as the integrand.
Since the integral operator is linear, the discussion in the previous section means that the CV method could potentially improve the numerical integration, thus the pricing algorithm and the prices.
A very popular FT-based pricing algorithm is developed by \citet{carr1999option}.
This section develops a CV-enhanced version of their algorithm.
The possibility of improving other FT-based pricing algorithms using CVs is also discussed.
Although we focus on vanilla calls, the same principle can be easily applied to other options.
Throughout this section, the risk-free interest rate is assumed to be a constant $r$.




\subsection{Carr and Madan's Algorithm and Its CV-Enhancement}
\label{sec:CMCV}



The FT-based pricing algorithm developed by \citet{carr1999option} works as follows.
Consider a vanilla call option with strike price $K$ and maturity $T$.
Let the underlying asset price process be $S_t$, and $\phi_T(u)$ be the characteristic function of the log-return $\log(S_T/S_0)$.
Denote $k = \log K$, $b = 1.5$, and $a(u) = e^{i(u - (b+1)i)\log S_0}$.
Then the time-0 value of the call option equals
\begin{eqnarray}
c = \frac{e^{-bk}}{\pi}\int_0^\infty e^{-iku} \frac{~e^{-rT}a(u)\phi_T(u-(b+1)i)~}{b^2+b-u^2+i(2b+1)u}\,du.\label{eq:PCarr}
\end{eqnarray}
With this formula, the call option price can be obtained by numerical integration.
\citet{carr1999option} evaluate (\ref{eq:PCarr}) for all $k$ values with one application of the fast Fourier transform (FFT).
Option prices with different strike prices can thus be found efficiently. %without evaluating (\ref{eq:PCarr}) with numerical integration multiple times.
This section only focuses on how to evaluate the integral (\ref{eq:PCarr}) efficiently and accurately by introducing a CV.
%For more details on how the FFT algorithm can be applied, see \citet{carr1999option}.




Suppose we are to price call options with (\ref{eq:PCarr}) under a target model that does not have a closed-form pricing formula for $c$ but only a closed-form characteristic function $\phi_T(u)$.
%meaning (\ref{eq:PCarr}) is not known in closed form; only the integrand is.
To evaluate (\ref{eq:PCarr}) one will apply numerical integration with, for instance, the trapezoidal rule.
We proceed to show that this can be enhanced by the CV technique with a suitable CV.
A CV is a closed-form approximation of the integrand in (\ref{eq:PCarr}) whose integral is in closed form. %, which may hold for various models.
By the notations in Section \ref{sec:gcv}, the linear operator $\mathsf T$ is $\int_0^\infty$ in (\ref{eq:PCarr}), $\mathbb T$ the trapezoidal rule (e.g.), $f$ the integrand,
and the CV $g$ an approximation of $f$ such that $\mathsf T(g) = \int_0^\infty g\,du$ is known in closed form.
%Although with the integrand in (\ref{eq:PCarr}) it does not seem obvious how to choose a CV,
We will provide a general methodology to pick good $g$'s for all target models with closed-form characteristic functions but without closed-form option pricing formulas.



%The integral in (\ref{eq:PCarr}) is in closed form when the option price $c$ is.
%The same is true for many other FT-based pricing algorithms.
For any model whose $\phi_T(u)$ {\em and} call option price come in closed form,
%with being the characteristic function of any model that has a closed-form call option price,
(\ref{eq:PCarr}) is in closed form and its integrand can serve as a CV.
This model will be referred to as the CV model from now on.
Among possible CV models %with closed-form call option prices
we shall choose the one suggested by \citet{jarrow1982approximate}, which is implied by the Edgeworth series expansion.
Furthermore, this CV model is easy to calibrate, which makes it a good CV as we will explain later.




To derive a CV from the Edgeworth series, we introduce the random variable $X = \log(S_T/S_0)$.
When using FT-based pricing algorithms, $X$'s characteristic function $\phi_T(u)$ is assumed to have had a closed-form expression.
The Edgeworth series expansion requires the first few moments of $X$, which are known in closed form:
The $k$-th raw moment $m_k$ is simply $(-i)^n d^n\phi_T(0)/du^n$.
Let $m$ and $s^2$ be the mean and variance of $X$, respectively.
Denote by $q(x)$ the normal density function with mean $m$ and variance $s^2$.
%Given the first few cumulants $c_k$ of $X$,
We follow \citet{jarrow1982approximate} to use
the 5-th order Edgeworth series approximation of the density function of $X$,
\begin{eqnarray}
f^{\text{CV}}(x) = q(x) -\frac{c_3}{3!}\frac{d^3q}{dx^3}(x) + \frac{c_4}{4!}\frac{d^4q}{dx^4}(x) - \frac{c_5}{5!}\frac{d^5q}{dx^5}(x),\label{eq:fCV_ew}
\end{eqnarray}
as an approximation to the density function of $X$,
where $c_k$ is the $k$-th cumulant of $X$ and can be evaluated in terms of raw moments:
\begin{eqnarray*}
c_3 &=& m_3-3m_2m_1+2{m_1}^3\\
c_4 &=& m_4-4m_3m_1-3{m_2}^2+12m_2{m_1}^2-6{m_1}^4\\
c_5 &=& m_5-5m_4m_1-10m_3m_2+20m_3{m_1}^2+30{m_2}^2m_1-60m_2{m_1}^3+24{m_1}^5.
\end{eqnarray*}
%This is the CV of the density function.
Note that $f^{\text{CV}}(x)$ is not a probability density function since it can take on negative values,
but its FT is known in closed form as
\begin{eqnarray}
\phi_T^{\text{CV}}(u)   &=& \int_{-\infty}^{\infty}e^{iux}\left(q(x) -\frac{c_3}{3!}\frac{d^3q}{dx^3}(x) + \frac{c_4}{4!}\frac{d^4q}{dx^4}(x) - \frac{c_5}{5!}\frac{d^5q}{dx^5}(x)\right)\,dx\nonumber\\
                        &=& e^{imu-\frac{s^2 u^2}{2}}\left[\left(1+\frac{c_4}{4!}u^4\right) - i\left(\frac{c_3}{3!}u^3-\frac{c_5}{5!}u^5\right)\right].\label{eq:ctrl_chf}
\end{eqnarray}
%which will serve as our CV of $\phi_T(u)$.
If the density function of $X$ is well-approximated by $f^{\text{CV}}(x)$,
its FT, $\phi_T(u)$, is likely to be well-approximated by $\phi_T^{\text{CV}}(u)$, the FT of $f^{\text{CV}}(x)$.
If true, when the characteristic function $\phi_T(u)$ in (\ref{eq:PCarr}) is replaced by $\phi_T^{\text{CV}}(u)$,
the integrand should not change by much.
Define the difference of the characteristic functions as $\phi_T^{\text{diff}}(u) = \phi_T(u) - \phi_T^{\text{CV}}(u)$.
Now the integral (\ref{eq:PCarr}) can be decomposed as
\begin{eqnarray}
c &=& \frac{e^{-dk}}{\pi}\int_0^\infty e^{-iuk}\frac{~e^{-rT}a(u)\phi_T^{\text{diff}}(u-(d+1)i)~}{d^2+d-u^2+i(2d+1)u}\, du \label{eq:PCarrDiff}\\
    &&+ \frac{e^{-dk}}{\pi}\int_0^\infty e^{-iuk}\frac{~e^{-rT}a(u)\phi_T^{\text{CV}}(u-(d+1)i)~}{d^2+d-u^2+i(2d+1)u}\, du, \label{eq:PCarrCV}
\end{eqnarray}
which is the integral of the difference in (\ref{eq:PCarrDiff}) plus the integral of the CV in (\ref{eq:PCarrCV}) up to a multiplicative factor of $e^{-dk}/\pi$.
%The first and the second integrals (\ref{eq:PCarrDiff}) and (\ref{eq:PCarrCV}) are the difference and the CV of the option price, respectively.
While numerical integration is still needed to evaluate (\ref{eq:PCarrDiff}), (\ref{eq:PCarrCV}) at least is known in closed form by design.



Let $c^{\text{CV}}$ stand for (\ref{eq:PCarrCV}) for brevity.
%Think of it as the CV of the option price $c$.
%Although $c^{\text{CV}}$ does have a closed-form expression, it is hard to derive the formula directly from (\ref{eq:PCarrCV}).
%To derive the closed-form formula of (\ref{eq:PCarrCV}),
We now proceed to derive its closed-form formula.
Observe that $c^{\text{CV}}$ is just Carr and Madan's pricing formula for a call option with characteristic function $\phi_T^{\text{CV}}(u)$.
%Thus it is the value of a call option under the model implied by the Edgeworth expansion.
Thus it is the value of a call option whose density function is $f^{\text{CV}}(x)$;
as a result, alternatively,
%Instead of the writing $c^{\text{CV}}$ in terms of $\phi_T^{\text{CV}}(u)$, we use $f^{\text{CV}}(x)$ to derive the closed-form formula:
\begin{eqnarray*}
c^{\text{CV}} = e^{-rT}\int_{-\infty}^\infty (S_0 e^x-K)^+\,f^{\text{CV}}(x)\,dx.%\label{eq:cCVint}
\end{eqnarray*}
Since $f^{\text{CV}}(s)$ is the sum of a normal density function and its derivatives, $c^{\text{CV}}$ has a closed-form formula.
Indeed, the formula is available in \citet{jarrow1982approximate} as
{\small
\begin{eqnarray}
c^{\text{CV}}   &=& S_0\left(1 + \frac{c_3}{3!} + \frac{c_4}{4!} +  \frac{c_5}{5!}\right)e^{m+\frac{s^2}{2}-rT}N\left(\frac{m+s^2-\log\frac{K}{S_0}}{s}\right) - Ke^{-rT} N\left(\frac{m-\log\frac{K}{S_0}}{s}\right)\nonumber\\
                && + Ke^{-rT}q\left(\log\frac{K}{S_0} \right)\left[-\left(\frac{c_5}{5!}\right)p_3\left(\log\frac{K}{S_0} \right) + \left(\frac{c_5}{5!} + \frac{c_4}{4!}\right)p_2\left(\log\frac{K}{S_0} \right)\right.\nonumber\\
                &&\quad\qquad\qquad\qquad\qquad\quad\left. - \left(\frac{c_5}{5!} + \frac{c_4}{4!}  + \frac{c_3}{3!}\right)p_1\left(\log\frac{K}{S_0} \right) + \left(\frac{c_5}{5!} + \frac{c_4}{4!}  + \frac{c_3}{3!}\right)\right],\label{eq:cCV}
\end{eqnarray}
}
where $N(x)$ is the standard normal distribution function, and
\begin{eqnarray*}
%q(x) &=& \frac{1}{\sqrt{2\pi}} e^{-\frac12\left(\frac{x-m}{s}\right)^2},\\
p_k(x) &=& \frac{d^kq(x)}{dx^k}/q(x),\\
p_1(x) &=& (m-x)/s^2, \\
p_2(x) &=& \left(m^2-s^2-2mx+x^2\right)/s^4, \\
p_3(x) &=& (m-x)\left(m^2-3s^2-2mx+x^2\right)/s^6.
\end{eqnarray*}
Finally, our CV-enhanced Carr and Madan algorithm is
\begin{eqnarray*}
c = \frac{e^{-dk}}{\pi}\int_0^\infty e^{-iuk}\frac{~e^{-rT}a(u)\phi_T^{\text{diff}}(u-(d+1)i)~}{d^2+d-u^2+i(2d+1)u}\, du + c^{\text{CV}}
\end{eqnarray*}
whose integral can be evaluated by numerical integration
and $c^{\text{CV}}$ is given by (\ref{eq:cCV}).




%Recall that a CV should satisfy two conditions:
%\begin{enumerate}
%\item $\phi_T(u)$ should be well-approximated by $\phi_T^{\text{CV}}(u)$, and
%\item the CV of the option value $c^{\text{CV}}$ is known in closed form.
%\end{enumerate}
%The second condition means that we need a model under which the option price is known in closed form,
%and the first one ensures that the difference of the integrand takes small values.




% before reversing the paragraph

%It is possible to choose CV models other than the one implied by the Edgeworth expansion,
%as long as the characteristic function and the call option price under the CV model are in closed form.
%However, finding the best set of parameters in the CV then can be more time-consuming.
%For any CV model, the parameters have to be determined so that $\phi_T^{\text{CV}}(u)$ approximates $\phi_T(u)$ well.
%This is a calibration problem.
%Our goal is to calibrate the CV model to the target model. %, the model under which $\phi_T(u)$ is derived.
%Since Carr and Madan's algorithms is already computationally cheap,
%it would not be economical if some rather expensive numerical optimization algorithm is involved for calibration.
%This suggests that the CV model implied by the Edgeworth series produces good CVs,
%because the parameters of an Edgeworth expansion are just the moments and cumulants of the target model, which are known in closed form.



Any model with a closed-form characteristic function and a closed-form call option price can serve as a CV model, e.g., the jump-diffusion model of \citet{merton1976option}.
But the CV model implied by the Edgeworth series has an advantage over many other models in that its coefficients are known in closed form:
They are just the moments and cumulants of the target model.
In general, the parameters of a CV model have to be determined so that $\phi_T^{\text{CV}}(u)$ approximates $\phi_T(u)$ well.
This is a calibration problem, which can be time-consuming for some CV models.
%Our goal is to calibrate the CV model to the target model. %, the model under which $\phi_T(u)$ is derived.
Since Carr and Madan's algorithm is efficient,
it would not be economical to augment it with a CV model which is expensive to calibrate.
The same consideration applies when the CV technique is added to all other FT-based pricing algorithms.




%The classic Black-Scholes model also has a closed-form option pricing formula, and the only parameter $\sigma$, the volatility, is easy to calibrate.
%Thus one can use it as the CV model.
%%With a similar idea, \citet{tankov2003financial} derives an FT-based pricing algorithm by taking the FT of the difference of the option price and the Black-Scholes formula.
%With a similar idea, \citet{tankov2003financial} prices options by numerically integrating the closed-form FT of the difference between the option price and the Black-Scholes formula.
%While their purpose is to obtain a smooth and integrable function before taking the FT,
%ours is to obtain an integral whose integrand only takes small values.
%Also, using the Black-Scholes model as the CV is actually a special case of the Edgeworth series expansion.
%Indeed, while the Black-Scholes model only fits the second moment (the volatility),
%the Edgeworth series expansion is able to capture the skewness of the target model by fitting higher order moments.
%Numerical results are provided later to show that adding the information about the 3-rd and 4-th moments of the target model really does yield more accuracy.




If one takes $f^{\text{CV}}(x) = q(x)$ in (\ref{eq:fCV_ew}), the CV model is the classic Black-Scholes model.
%\citet{tankov2003financial} derive an FT-based pricing algorithm by considering the difference between the option price and the Black-Scholes formula.
%While their purpose is to obtain a smooth and integrable function before taking the FT,
%ours is to obtain an integral, such as (\ref{eq:PCarrDiff}), whose integrand only takes on small values.
%Unlike the Black-Scholes model, which only fits the first two moments,
%the more general Edgeworth series expansion is able to capture also higher moments of the target model.
But the Black-Scholes model only fits the first two moments,
unlike the more general Edgeworth series expansion, which is able to capture also higher moments of the target model.
Numerical results are provided later to show that adding the 3-rd and 4-th moments of the target model increases accuracy.



\subsection{Other FT-Based Algorithms and Their CVs}





The CV technique can be added to all FT-based pricing algorithms.
%Any FT-based pricing algorithm can be modified by the CV technique.
Two more examples are the Black-Scholes-type formula and Lewis' formula.
For any option pricing models that do not have closed-form pricing formulas but only closed-form $\phi_T(u)$,
under the assumption that $\phi_T(-i) = e^{rT}$,
the so-called Black-Scholes-type formula for the vanilla call is given by
\begin{eqnarray}
c &=& S_0 \Pi_1 - K e^{-rT} \Pi_2,\label{eq:BSstyle}
\end{eqnarray}
where
\begin{eqnarray*}
\Pi_1 &=& \frac{1}{2} + \frac{1}{\pi}\int_0^{\infty} \text{Re}\left[\frac{e^{-iu \log\left(\frac{K}{S_0}\right)}\phi_T(u-i)}{iu\phi_T(-i)}\right]\,du,\\
\Pi_2 &=& \frac{1}{2} + \frac{1}{\pi}\int_0^{\infty} \text{Re}\left[\frac{e^{-iu \log\left(\frac{K}{S_0}\right)}\phi_T(u)}{iu}\right]\,du.
\end{eqnarray*}
Both $\Pi_1$ and $\Pi_2$ can be integrated numerically.
This formula is first derived by \citet{heston1993closed} for his stochastic-volatility model.
Under the same set of assumptions,
\citet{lewis2001simple} derives an FT-based pricing formula for any arbitrary function of the terminal underlying asset price $S_T$.
Thus pricing formulas of any path-independent European-type options can be easily derived.
For the vanilla call he obtains
\begin{eqnarray}
c &=& S_0 - \frac{\sqrt{S_0 K} e^{-rT}}{\pi} \int_0^\infty \text{Re}\left[e^{iu \log (S_0/K)}\phi_T\left(u-\frac{i}{2}\right)\right] \frac{du}{u^2 + (1/4)}\label{eq:Lewis}.
\end{eqnarray}
%\footnote{
%\citet{carr1999option} use the characteristic function of $\log S_T$ to derive the pricing algorithm,
%whereas \citet{lewis2001simple} uses the characteristic function of $\log(S_T/S_0) - rT$.
%We use the characteristic function of $\log(S_T/S_0)$.
%}
Pricing again requires numerical integration.


% ���� Lewis ���g�k
%
%To add the CV technique to Lewis' formula, first choose a model whose option price {\em and} characteristic function $\phi_T^{\text{CV}}(u)$ come in closed form,
%and impose the assumption $\phi_T^{\text{CV}}(-i) = e^{rT}$.
%Recall the decomposition $\phi_T(u) = \phi_T^{\text{diff}}(u) + \phi_T^{\text{CV}}(u)$.
%Then (\ref{eq:Lewis}) becomes
%\begin{eqnarray}
%c &=& - \frac{\sqrt{S_0 K} e^{-rT}}{\pi} \int_0^\infty \text{Re}\left[e^{iu\log (S_0/K)}{\phi_T^{\text{diff}}\left(u-\frac{i}{2}\right)}\right] \frac{du}{u^2+ (1/4)}\nonumber \\
%&& + S_0 - \frac{\sqrt{S_0 K} e^{-rT}}{\pi} \int_0^\infty \text{Re}\left[e^{iu\log (S_0/K)}{\phi_T^{\text{CV}}\left(u-\frac{i}{2}\right)}\right] \frac{du}{u^2+ (1/4)} \label{eq:cCVLewis}\\
%&=& - \frac{\sqrt{S_0 K} e^{-rT}}{\pi} \int_0^\infty \text{Re}\left[e^{iu\log (S_0/K)}{\phi_T^{\text{diff}}\left(u-\frac{i}{2}\right)}\right] \frac{du}{u^2+ (1/4)} + c^{\text{CV}},\label{eq:CVLewis}
%\end{eqnarray}
%where $c^{\text{CV}}$ stands for (\ref{eq:cCVLewis}).
%Since $c^{\text{CV}}$ is simply Lewis' formula for a call option with characteristic function $\phi_T^{\text{CV}}(u)$, it is known in closed form by choice.
%Equation (\ref{eq:CVLewis}) is the CV-version Lewis' formula.
%Numerical integration is needed only for the first term of (\ref{eq:CVLewis}). % as $c^{\text{CV}}$ is already in closed form.
%Similarly, the CV-version Black-Scholes-type formula (\ref{eq:BSstyle}) equals
%\begin{eqnarray*}
%c = S_0 \Pi_1^{\text{diff}} - K e^{-rT} \Pi_2^{\text{diff}} + c^{\text{CV}},
%\end{eqnarray*}
%where
%$$
%\Pi_1^{\text{diff}} = \frac{1}{\pi}\int_0^{\infty} \text{Re}\left[\frac{e^{-iu \log \left(\frac{K}{S_0}\right)}{\phi_T^{\text{diff}}(u-i)}}{iu\phi_T(-i)}\right]\,du,\qquad
%\Pi_2^{\text{diff}} = \frac{1}{\pi}\int_0^{\infty} \text{Re}\left[\frac{e^{-iu \log \left(\frac{K}{S_0}\right)}{\phi_T^{\text{diff}}(u)}}{iu}\right]\,du
%$$
%are evaluated by numerical integration.
%
%


To add the CV technique to the Black-Scholes-type formula (\ref{eq:BSstyle}), first choose a model whose option price {\em and} characteristic function $\phi_T^{\text{CV}}(u)$ come in closed form,
and impose the assumption $\phi_T^{\text{CV}}(-i) = e^{rT}$.
Recall the decomposition $\phi_T(u) = \phi_T^{\text{diff}}(u) + \phi_T^{\text{CV}}(u)$.
Then (\ref{eq:BSstyle}) becomes
\begin{eqnarray}
c &=& (S_0 \Pi_1^{\text{diff}} - K e^{-rT} \Pi_2^{\text{diff}}) + (S_0 \Pi_1^{\text{CV}} - K e^{-rT} \Pi_2^{\text{CV}}) \label{eq:CVBSstyle}\\
  &=& (S_0 \Pi_1^{\text{diff}} - K e^{-rT} \Pi_2^{\text{diff}}) + c^{\text{CV}},\nonumber
\end{eqnarray}
where
\begin{eqnarray*}
&& c^{\text{CV}} = S_0 \Pi_1^{\text{CV}} - K e^{-rT} \Pi_2^{\text{CV}}, \\
&& \Pi_1^{\text{diff}} = \frac{1}{\pi}\int_0^{\infty} \text{Re}\left[\frac{e^{-iu \log \left(\frac{K}{S_0}\right)}{\phi_T^{\text{diff}}(u-i)}}{iu\phi_T(-i)}\right]\,du,\qquad
\Pi_2^{\text{diff}} = \frac{1}{\pi}\int_0^{\infty} \text{Re}\left[\frac{e^{-iu \log \left(\frac{K}{S_0}\right)}{\phi_T^{\text{diff}}(u)}}{iu}\right]\,du,\\
&& \Pi_1^{\text{CV}} = \frac{1}{\pi}\int_0^{\infty} \text{Re}\left[\frac{e^{-iu \log \left(\frac{K}{S_0}\right)}{\phi_T^{\text{CV}}(u-i)}}{iu\phi_T(-i)}\right]\,du,\qquad
\Pi_2^{\text{CV}} = \frac{1}{\pi}\int_0^{\infty} \text{Re}\left[\frac{e^{-iu \log \left(\frac{K}{S_0}\right)}{\phi_T^{\text{CV}}(u)}}{iu}\right]\,du,
\end{eqnarray*}
and numerical integration is needed only for $\Pi_1^{\text{diff}}$ and $\Pi_2^{\text{diff}}$. % as $c^{\text{CV}}$ is already in closed form.
Since $c^{\text{CV}}$ is simply the Black-Scholes-type formula for a call option with characteristic function $\phi_T^{\text{CV}}(u)$, it is known in closed form by choice.
Equation (\ref{eq:CVBSstyle}) is the CV-version Black-Scholes-type formula.
Similarly, the CV version of Lewis' formula (\ref{eq:Lewis}) equals
\begin{eqnarray}
c &=& - \frac{\sqrt{S_0 K} e^{-rT}}{\pi} \int_0^\infty \text{Re}\left[e^{iu\log (S_0/K)}{\phi_T^{\text{diff}}\left(u-\frac{i}{2}\right)}\right] \frac{du}{u^2+ (1/4)}\nonumber \\
&& + S_0 - \frac{\sqrt{S_0 K} e^{-rT}}{\pi} \int_0^\infty \text{Re}\left[e^{iu\log (S_0/K)}{\phi_T^{\text{CV}}\left(u-\frac{i}{2}\right)}\right] \frac{du}{u^2+ (1/4)} \nonumber \\%\label{eq:cCVLewis}\\
&=& - \frac{\sqrt{S_0 K} e^{-rT}}{\pi} \int_0^\infty \text{Re}\left[e^{iu\log (S_0/K)}{\phi_T^{\text{diff}}\left(u-\frac{i}{2}\right)}\right] \frac{du}{u^2+ (1/4)} + c^{\text{CV}},\label{eq:CVLewis}
\end{eqnarray}
where the first term in (\ref{eq:CVLewis}) is evaluated by numerical integration.









%Any FT-based pricing algorithm can be modified by the CV technique.
%Two other popular FT-based pricing algorithms are Lewis' formula
%\begin{eqnarray}
%c &=& S_0 - \frac{\sqrt{S_0 K} e^{-rT/2}}{\pi} \int_0^\infty \text{Re}\left[e^{iu(\log (S_0/K) + rT)}\phi_T\left(u-\frac{i}{2}\right)\right] \frac{du}{u^2 + (1/4)}\label{eq:Lewis},
%\end{eqnarray}
%and the Black-Scholes-type formula
%\begin{eqnarray}
%c &=& S_0 \Pi_1 - K e^{-rT} \Pi_2,\label{eq:BSstyle}
%\end{eqnarray}
%where
%\begin{eqnarray*}
%\Pi_1 &=& \frac{1}{2} + \frac{1}{\pi}\int_0^{\infty} \text{Re}\left[\frac{e^{-iu \log\left(\frac{K}{S_0}\right)}\phi_T(u-i)}{iu\phi_T(-i)}\right]\,du,\\
%\Pi_2 &=& \frac{1}{2} + \frac{1}{\pi}\int_0^{\infty} \text{Re}\left[\frac{e^{-iu \log\left(\frac{K}{S_0}\right)}\phi_T(u)}{iu}\right]\,du.
%\end{eqnarray*}
%The Black-Scholes-type formula (\ref{eq:BSstyle}) is first derived in \citet{heston1993closed} to evaluate the option price in the Heston stochastic-volatility model.
%Lewis' formula (\ref{eq:Lewis}) is derived in a very general framework developed by \citet{lewis2001simple},
%which considers both $\phi_T(u)$ and the option payoff function to be evaluated.
%Thus pricing formulas for any European options can be easily derived, including put and binary options.\footnote{
%\citet{carr1999option} use the characteristic function of $\log S_T$ to derive the pricing algorithm,
%whereas \citet{lewis2001simple} uses the characteristic function of $\log(S_T/S_0) - rT$.
%We use the characteristic function of $\log(S_T/S_0)$.
%%That is why all pricing formulas are slightly different from the original paper.
%}
%With the decomposition $\phi_T(u) = \phi_T^{\text{diff}}(u) + \phi_T^{\text{CV}}(u)$,
%the two FT-based pricing algorithms (\ref{eq:Lewis}) and (\ref{eq:BSstyle}) can be rewritten as
%\begin{eqnarray*}
%c &=& - \frac{\sqrt{S_0 K} e^{-rT/2}}{\pi} \int_0^\infty \text{Re}\left[e^{iu(\log (S_0/K) + rT)}{\phi_T^{\text{diff}}\left(u-\frac{i}{2}\right)}\right] \frac{du}{u^2+(1/4)} + c^{\text{CV}}\\
% &=& S_0 \Pi_1^{\text{diff}} - K e^{-rT} \Pi_2^{\text{diff}} + c^{\text{CV}},
%\end{eqnarray*}
%where
%$$
%\Pi_1^{\text{diff}} = \frac{1}{\pi}\int_0^{\infty} \text{Re}\left[\frac{e^{-iu \log \left(\frac{K}{S_0}\right)}{\phi_T^{\text{diff}}(u-i)}}{iu\phi_T(-i)}\right]\,du,\qquad
%\Pi_2^{\text{diff}} = \frac{1}{\pi}\int_0^{\infty} \text{Re}\left[\frac{e^{-iu \log \left(\frac{K}{S_0}\right)}{\phi_T^{\text{diff}}(u)}}{iu}\right]\,du,
%$$
%and the closed-form formula $c^{\text{CV}}$ is given in (\ref{eq:cCV}).
%





Both (\ref{eq:BSstyle}) and (\ref{eq:Lewis}) are derived with the equality $\phi_T(-i) = e^{rT}$,
a consequence of the fact that $\log(S_t/S_0) - rt$ is a martingale in the risk-neutral probability measure.
Because of this additional requirement,
%one can not use the Edgeworth expansion the same way as we did in Section \ref{sec:CMCV}
the Edgeworth expansion approach in Section \ref{sec:CMCV} needs some revision
in finding the CVs for the CV-version Black-Scholes-type formula (\ref{eq:CVBSstyle}) and Lewis' formula (\ref{eq:CVLewis}).
%Take the CV-version Black-Style-type formula for example.
Take (\ref{eq:CVBSstyle}) for example.
When deriving its characteristic function $\phi_T^{\text{CV}}(u)$ with the Edgeworth expansion,
the coefficients $c_3$, $c_4$, $c_5$ must uphold the identity $\phi_T^{\text{CV}}(-i) = e^{rT}$
so that $(S_0\Pi_1^{\text{CV}} - K e^{-rT} \Pi_2^{\text{CV}})$ in (\ref{eq:CVBSstyle}) equals (\ref{eq:cCV}).
In other words, the choice of the coefficients must uphold the identity
or there will be an extra source of error arising from the inequality of $(S_0\Pi_1^{\text{CV}} - K e^{-rT} \Pi_2^{\text{CV}})$ and (\ref{eq:cCV}).
With this in mind, we shall set $c_3$ and $c_4$ to be the 3-rd and 4-th cumulants of the target model, respectively, as before,
but determine $c_5$ by solving $\phi_T(-i) = e^{rT}$,
which leads to
\begin{eqnarray}
c_5 = 5!\left(e^{rT-m-\frac{s^2}{2}} - \left(1 + \frac{c_3}{3!} + \frac{c_4}{4!}\right)\right).\label{eq:c5}
\end{eqnarray}
%Thus we lose one degree of freedom when constructing the CV.
In general, when adding the CV technique to an FT-based pricing algorithm,
if the derivation of the original algorithm uses the identity $\phi_T(-i) = e^{rT}$,
$c_5$ should be determined by (\ref{eq:c5}) instead of the 5-th cumulant of the target model.
In fact, we will always use (\ref{eq:c5}) even if the original algorithm may not have anything to do with the identity just to be safe --- even for Carr and Madan's algorithm.

%as it is not convenient to go through the derivation of an FT-based pricing algorithm just to check if it uses $\phi_T(-i) = e^{rT}$.
%So regardless of the type of the FT-based algorithm



%In general, the equality may not hold for the characteristic function of the distribution inferred from a set of moments and cumulants $m, s, c_3, c_4, c_5$.






Not all FT-based pricing algorithms benefit from the same technique.
With a CV, one needs to evaluate two characteristic functions in $\phi_T^{\text{diff}}(u) = \phi_T(u) - \phi_T^{\text{CV}}(u)$
instead of just $\phi_T(u)$ originally, even if $c^{\text{CV}}$ is easy to calculate.
So when the original algorithm already converges fast, these added costs may outweigh the increased accuracy.
Given the slow convergence of Carr and Madan's algorithm (see \citet{tankov2003financial}),
adding the CV technique improves its overall efficiency and accuracy, as we will show below.
%because, as remarked in \citet{tankov2003financial}, it has a slow convergence.
%We shall see the efficiency and accuracy comparison later in the numerical results.
%This claim will be confirmed in the end of this section.






%\subsection{Calibrating the Control Model}
%\label{sec:find_cumulants}
%
%Using the model based on the Edgeworth expansion as the control, we have five parameters to estimate -- $m, s, c_3, c_4$ and $c_5$.
%Our goal is to find a set of parameters so that the $\phi_T^{\text{CV}}(u)$ function in (\ref{eq:ctrl_chf})
%behaves similarly to the characteristic function used in pricing.
%The assumption of the Edgeworth expansion is that $m$ and $s$ are the mean and the standard deviation of the distribution to be approximated,
%and $c_3$, $c_4$ and $c_5$ are its 3-rd, 4-th and 5-th cumulants.
%In our case of interest the distribution to be approximated is the distribution of the log-return $\log(S_T/S_0)$.
%One way of finding the cumulants is to first find the raw moments through differentiating the closed-form characteristic function $\phi_T(u)$.
%Let $m_i$ be the $i$-th raw moment.
%The the mean and the standard deviation are obviously given by
%\begin{eqnarray*}
%m &=& m_1,\\
%s &=& \sqrt{m_2 - m_1^2}.
%\end{eqnarray*}
%The cumulants can be obtained by the following formulas
%\begin{eqnarray*}
%c_3 &=& 2 m_1^3-3 m_1 m_2+m_3, \\
%c_4 &=& -6 m_1^4+12 m_1^2 m_2-4 m_1 m_3-3 m_2^2+m_4, \\
%c_5 &=& 24 m_1^5-60 m_1^3 m_2+20 m_1^2 m_3+30 m_1 m_2^2-5 m_1 m_4-10 m_2 m_3+m_5.
%\end{eqnarray*}
%This way of getting the cumulants is, however, sometimes not very convenient.
%
%
%
%Many, if not all, popular derivatives pricing models have characteristic functions of the form $\phi_T(u) = e^{\psi_T(u)}$,
%where $\psi_T(u)$ is called the characteristic exponent.
%Examples include all exponential L\'evy models and the Heston stochastic-volatility model.
%When differentiating a characteristic function $\phi_T(u)$ of this form to get the raw moments, the expression soon gets complicated.
%A better way to find cumulants is to use the definition
%-- the $k$-th cumulant of a random variable $X$ is the $k$-th coefficient of the power series expansion of $\log E[e^{u X}]$ multiplied by $k!$.
%Now to find the cumulants of $\log(S_T/S_0)$
%given that the characteristic function is of the form $\phi_T(u) = E[e^{i u \log(S_T/S_0)}] = e^{\psi_T(u)}$,
%we just need the power series of
%$$
%\log E[e^{u \log(S_T/S_0)}] = \log \phi_T(-iu) = \psi_T(-iu),
%$$
%the first few terms of which will be much easier to write down explicitly than differentiating $\phi_T(u)$ directly.
%As an example, consider the exponential L\'evy model developed by \citet{merton1976option}, whose characteristic exponent is given by
%$$
%\psi_T(u) = iu\left( r- \lambda\left(e^{\alpha+\frac{\beta^2}{2}}-1\right)-\frac{\sigma^2}{2}\right)T -\frac{\sigma^2 u^2}{2}T + \lambda T \left(e^{i\alpha u -\frac{1}{2}\beta^2 u^2}-1\right),
%$$
%where $\alpha, \beta>0, \sigma>0$ and $\lambda>0$ are model parameters, and $r$ is the constant risk-free interest rate.
%Thus
%\begin{eqnarray}
%\psi_T(-iu) = u\left( r- \lambda\left(e^{\alpha+\frac{\beta^2}{2}}-1\right)-\frac{\sigma^2}{2}\right)T + \frac{\sigma^2 u^2}{2}T + \lambda T \left(e^{\alpha u +\frac{1}{2}\beta^2 u^2}-1\right).\label{eq:charExpJD}
%\end{eqnarray}
%We expand the last parentheses using the series $e^x - 1 = x + x^2/2! + x^3/3! + x^4/4! + \cdots$.
%We obtain
%\begin{eqnarray*}
%&&e^{\alpha u +\frac{1}{2}\beta^2 u^2} - 1 \\
%&=& \left(\alpha u +\frac{1}{2}\beta^2 u^2\right) + \frac12 \left(\alpha u +\frac{1}{2}\beta^2 u^2\right)^2 + \frac16 \left(\alpha u +\frac{1}{2}\beta^2 u^2\right)^3 + \frac{1}{24} \left(\alpha u +\frac{1}{2}\beta^2 u^2\right)^4 + \cdots\\
%&=& \alpha u + \frac12 (\alpha^2 + \beta^2) u^2 + \frac16 (\alpha^3 + 3 \alpha \beta^2) u^3 + \frac{1}{24} (\alpha^4 + 6 \alpha^2 \beta^2 + 3 \beta^4) u^4 + \cdots.
%\end{eqnarray*}
%Plugging this into (\ref{eq:charExpJD}) we obtain the power series
%\begin{eqnarray*}
%\psi_T(-iu) &=& \left( r- \lambda\left(e^{\alpha+\frac{\beta^2}{2}}-1-\alpha\right)-\frac{\sigma^2}{2}\right)Tu + \frac{1}{2}(\sigma^2 + \lambda(\alpha^2 + \beta^2))T u^2\\
%&&+ \frac16 \lambda \alpha(\alpha^2 + 3\beta^2)Tu^3 + \frac{1}{24} \lambda(\alpha^4 + 6\alpha^2\beta^2 + 3\beta^4)Tu^4 + \cdots.
%\end{eqnarray*}
%By definition this is equal to $\sum_{n=1}^\infty c_n\frac{u^n}{n!}$, with $c_n$ the $n$-th cumulant.
%Thus we have
%\begin{eqnarray*}
%c_1 &=& \left( r- \lambda\left(e^{\alpha+\frac{\beta^2}{2}}-1-\alpha\right)-\frac{\sigma^2}{2}\right)T, \\
%c_2 &=& (\sigma^2 + \lambda(\alpha^2 + \beta^2))T, \\
%c_3 &=& \lambda \alpha(\alpha^2 + 3\beta^2)T, \\
%c_4 &=& \lambda(\alpha^4 + 6\alpha^2\beta^2 + 3\beta^4)T,\\
%&& \vdots
%\end{eqnarray*}
%This is the first few cumulants of $\log(S_T/S_0)$ under Merton's model.
%The first two cumulants are also the mean and the variance, so we have
%$$
%m = c_1, \qquad s = \sqrt{c_2}.
%$$
%In general for any derivatives pricing model with a characteristic function of the form $\phi_T(u) = e^{\psi_T(u)}$,
%the first few cumulants can be easily computed by the same methodology.






%http://mathurl.com/h49smx7








\subsection{Numerical Results}


This subsection compares Carr and Madan's algorithm with its CV versions.
We use both the Black-Scholes CV model and the Edgeworth expansion-implied CV
to evaluate a vanilla call option under the Heston model.
(The Heston model and its characteristic function are covered in Appendix \ref{appendix:heston}.)
The option has maturity $T=1$ and strike price $K=120$, while the initial underlying asset price is $S_0=100$.
The Heston model's parameters for the experiments are $r = 0.1$, $\kappa = 2$, $\theta = 0.04$, $\sigma = 0.5$, $\rho = -0.7$, and $V_0 = 0.04$.
The benchmark is 3.29968857626, obtained from Carr and Madan's formula (\ref{eq:PCarr}) by the trapezoidal rule with the domain of integration $[0, 200]$ and 10,000 partitions.
A comparison of the efficiency and the accuracy is shown in Table \ref{tab:CM}.
``CM'' refers to the original Carr and Madan's algorithm,
``BS-CV'' the CV version with the Black-Scholes CV model,
and ``EW-CV'' the CV version implied by the Edgeworth expansion.
For all three algorithms, the trapezoidal rule is used to integrate over the interval $[0, 200]$.
``CPU (ms)'' lists the running times in milliseconds,
``log error'' tabulates the base-10 logarithms of the pricing errors,
and $n$ denotes the number of partitions.
Roughly speaking, a log error of about $-(1+m)$ means the result has $m$-decimal-place accuracy.



From Table \ref{tab:CM}, the CV versions are more time consuming than the original algorithm for all $n$, as mentioned earlier.
% always need more time to accomplish the task.
% This is because the most time consuming part of an FT-based algorithm is the characteristic function evaluation,
% and with CV we need $\phi_T^{\text{diff}}(u) = \phi_T(u) - \phi_T^{\text{CV}}(u)$, that is one more characteristic function to evaluate.
This extra cost is, however, more than compensated by the improved accuracy benefiting from the CVs.
% In around 3 milliseconds, we get one-decimal-place accuracy with the Black-Scholes CV model,
% three-decimal-place accuracy with the Edgeworth expansion-implied CV,
% while the pricing error of Carr and Madan's algorithm is just slightly less than 1.
% From this table it is obvious that using a CV in Carr and Madan's algorithm does improve the accuracy and efficiency,
% and that the CV implied by the Edgeworth expansion is better than the Black-Scholes model.
Figure \ref{fig:CM} visualizes the same numerical results.






\begin{table}[!t]
%\begin{figure}[!t]
\begin{center}
\begin{tabular}{|r|rr|rr|rr|}
%\toprule
\hline
{} &      \multicolumn{2}{|c|}{CM}           &   \multicolumn{2}{|c|}{CM-BS-CV}           &   \multicolumn{2}{|c|}{CM-EW-CV}           \\
\hline
{$n$} &  CPU (ms) & log error &  CPU (ms) & log error &  CPU (ms) & log error \\
%\midrule
\hline
20  &  0.759470 &  1.549590 &  0.963912 & $-$1.295336 &  1.577646 & $-$0.985337 \\
30  &  0.961859 &  1.317691 &  1.294384 & $-$1.233819 &  2.081359 & $-$2.013734 \\
40  &  1.243889 &  1.113891 &  1.654414 & $-$1.434872 &  2.564547 & $-$3.078093 \\
50  &  1.501288 &  0.918457 &  2.095728 & $-$1.659546 &  3.076060 & $-$4.121298 \\
60  &  1.789476 &  0.724625 &  2.454526 & $-$1.879688 &  3.592500 & $-$5.167961 \\
70  &  1.993918 &  0.529981 &  2.828925 & $-$2.094297 &  4.149171 & $-$6.218208 \\
80  &  2.282106 &  0.333783 &  3.207428 & $-$2.305139 &  4.639338 & $-$7.271658 \\
90  &  2.564547 &  0.135955 &  3.564585 & $-$2.513626 &  5.235830 & $-$8.327967 \\
100 &  2.837956 & $-$0.063333 &  3.883152 & $-$2.720658 &  5.723533 & $-$9.386850 \\
%\bottomrule
\hline
\end{tabular}
\end{center}
\caption{Efficiency and Accuracy of Carr and Madan's Algorithm and Its CV Versions.
``CM'' is the original Carr and Madan's algorithm,
``CM-BS-CV'' its CV version with the Black-Scholes model as the CV,
and ``CM-EW-CV'' its CV version with the Edgeworth expansion-implied CV.
CPU (ms) is the computational time in milliseconds,
log error the base-10 logarithm of the pricing error,
and $n$ the number of partitions in the trapezoidal rule.
}\label{tab:CM}
%\end{figure}
\end{table}




\begin{figure}[!t]
\begin{center}
    \parbox[t]{0.5\textwidth}{
        \centerline{\epsfig{figure=eps/CM.eps, width=0.5\textwidth}}
    }\hfill
    \parbox[t]{0.5\textwidth}{
        \centerline{\epsfig{figure=eps/CMn.eps, width=0.5\textwidth}}
    }\hfill
\end{center}
\caption{
Efficiency and Accuracy of Carr and Madan's Algorithm and Its CV Versions Plotted.
%``num'' is the number of partitions in the numerical integration.
}\label{fig:CM}
\end{figure}



\begin{table}[!t]
%\begin{figure}[!t]
\begin{center}
\begin{tabular}{rrrr}
\hline
{$n$} &      \multicolumn{1}{c}{CM}   &   \multicolumn{1}{c}{CM-BS-CV}  &    \multicolumn{1}{c}{CM-EW-CV} \\
\hline
20  & 46.67446801 & 0.052556521 & 0.065562188 \\
30  & 21.60625946 & 0.045093907 & 0.004654993 \\
40  & 10.44983355 & 0.022206689 & 0.000325759 \\
50  & 5.520685241 & 0.010450067 & 2.45871E$-$05 \\
60  & 2.964142797 & 0.005374578 & 1.89079E$-$06 \\
70  & 1.699314276 & 0.002844995 & 1.45825E$-$07 \\
80  & 0.945033464 & 0.001544202 & 1.15315E$-$08 \\
90  & 0.533266543 & 0.000859736 & 8.97527E$-$10 \\
100 & 0.30455192  & 0.000489957 & 7.16945E$-$11 \\
\hline
\end{tabular}
\end{center}
\caption{Error/Running Time Ratio of Carr and Madan's Algorithm and Its CV Versions.
For all algorithms we list the absolute pricing error divided by CPU time (ms).
}\label{tab:ErrCpuRatio}
\end{table}





\section{Convolution Pricing Algorithms}
\label{sec:CONV}
In all previous examples the CV technique runs alongside the target algorithm,
but the it can actually run within the target algorithms.
We use convolution-based pricing algorithms and binomial trees as examples.
In this section we discuss convolution-based pricing algorithms and their CV versions.
Binomial trees will be discussed in later sections.




\subsection{Background Review}
\label{sec:CONVBackgroundReview}
The convolution pricing algorithms developed by \citet{lord2008fast} are a enhanced version of the quadrature methods by \citet{andricopoulos2003universal}.
It can deal with the pricing problems of many exotic options including Bermudan options, discrete barrier options, look back options and Asian options (see \citet{vcerny2011improved}).
% under a wide class of option pricing models, including all exponential L\`evy models.
Although it only deals with discretely monitored exotic options,
a Richardson extrapolation can be applied to obtain the value of continuously monitored options.
The convolution pricing algorithm is based on the following theorem.
\begin{theorem}
\label{thm:CONV}
Let $W_t$ be the standard Brownian motion, and $\mathcal F_t$ a filtration for the Brownian motion.
Let $f(x)$ be a Borel measurable function, and $X_t = \mu t + \sigma W_t$.
Then, for all $\Delta t>0$, $t>0$,
$$
{\mathsf E}[f(X_{t+\Delta t})|\mathcal F_t] = \int_{-\infty}^\infty f(y)p(x-y)\,dy,
$$
where $x = X_t$, and
$$
p(z) = \frac{1}{\sigma\sqrt{2\pi\Delta t}}~e^{-\frac{(z + \mu\Delta t)^2}{2\sigma^2\Delta t}}
$$
is the transition density function.
In other words, the conditional expectation ${\mathsf E}[f(X_{t+\Delta t})|\mathcal F_t]$ is the convolution of $f(x)$ and a transition density $p(x)$, which we denote by
$$
{\mathsf E}[f(X_{t+\Delta t})|\mathcal F_t] = f(x)\otimes p(x).
$$
\end{theorem}
Due to the Markov property of the Brownian motion, the conditional expectation ${\mathsf E}[f(X_{t+\Delta t})|\mathcal F_t]$ ought to be a function of $X_t$.
The above theorem gives the function explicitly.
For a detailed proof of the theorem see, for example, \citet{shreve2004stochastic}.






The idea behind the convolution pricing algorithm is easy to understand.
Suppose, under the Black-Scholes model, we are to evaluate a discretely monitored exotic option with maturity $T$ and observation points $0 < t_1 < t_2 < \cdots < t_n = T$.
For simplicity assume that $\Delta t = t_{i+1} - t_i$ for all $i=1, 2, \ldots, n-1$.
Assume that under the risk-neutral probability measure the price of the underlying asset follows the log-normal diffusion process $dS_t = rS_tdt + \sigma S_tdW_t.$
Depending on the type of the option to be evaluated, at observation points the option value could jump.
For example, an up-and-out barrier option with barrier $B$ at time $t_i-$ is worthless if $S_t>B$ because it is going to knock out,
but at $t_i+$ it has positive time value since $S_t$ could drop below $B$ before next observation point.
Another example is that the value of a Bermudan put with strike $K$ at time $t_i-$ should be $K - S_{t_i}$ if $S_{t_i}$ is too low,
because the put holder has the right to exercise the contract right away,
but at $t_i+$ the contract can not be exercised until next observation point, so the value drops.
Apart from the observation points, in each time interval $(t_i , t_{i+1})$ the contract is just a vanilla European option, i.e., no early exercise, no knock in nor knock out,
so the price of the contract at $t_i+$ should be the discounted expected value of the price at $t_{i+1}-$ conditional on the filtration at $t_{i}$, which can be evaluated by a convolution due to Theorem \ref{thm:CONV}.
A convolution pricing algorithm begins with the payoff function in log scale and evaluates the option price backward in time.
The above argument gives an intuitive explanation of the algorithms in Figures \ref{algo:conv_berm} and \ref{algo:conv_barrier}.





\begin{figure}[!t]
\begin{framed}
\begin{algorithmic}[1]
    \STATE payoff$(x) \leftarrow \max(K - S_0e^x, 0)$
    \STATE put$(t_n, x) \leftarrow$ payoff$(x)$
    \STATE $p(x) \leftarrow e^{-(x+(r-\sigma^2/2)\Delta t)^2/(2\sigma^2\Delta t)}/(\sigma\sqrt{2\pi\Delta t})$
    \FOR{$i=n$ \textbf{downto} $1$ }
        \STATE $\text{put}(t_i, x) \leftarrow \max(\text{ put}(t_i, x), \text{ payoff}(x)~)$
        \STATE $\text{put}(t_{i-1}, x) \leftarrow e^{-r\Delta t}( \text{put}(t_i, x)\otimes p(x) )$
    \ENDFOR
    \STATE \textbf{return} $\text{put}(t_0, 0)$
\end{algorithmic}
\end{framed}
\caption{ Convolution Pricing Algorithm for Bermudan Put Options
} \label{algo:conv_berm}
\end{figure}



\begin{figure}[!th]
\begin{framed}
\begin{algorithmic}[1]
    \STATE payoff$(x) \leftarrow \max(S_0e^x - K, 0)$
    \STATE call$(t_n, x) \leftarrow$ payoff$(x)$
    \STATE $p(x) \leftarrow e^{-(x+(r-\sigma^2/2)\Delta t)^2/(2\sigma^2\Delta t)}/(\sigma\sqrt{2\pi\Delta t})$
    \FOR{$i=n$ \textbf{downto} $1$ }
        \STATE $\text{call}(t_i, x) \leftarrow \text{call}(t_i, x)\times 1_{\{x<\log(B/S_0)\}}$
        \STATE $\text{call}(t_{i-1}, x) \leftarrow e^{-r\Delta t}( \text{call}(t_i, x)\otimes p(x) )$
    \ENDFOR
    \STATE \textbf{return} $\text{call}(t_0, 0)$
\end{algorithmic}
\end{framed}
\caption{ Convolution Pricing Algorithm for Up-And-Out Barrier Call Options
} \label{algo:conv_barrier}
\end{figure}





In both Figures \ref{algo:conv_berm} and \ref{algo:conv_barrier}, $p(x)$ is the transition density function.
The value of the option is set to the payoff in log scale at time $T$ (see line 2).
The variable $x$ represents $\log(S_{t_i}/S_0)$ at time $t_i$.
In line 5 the option value is adjusted according to what should happen at observation point $t_i$.
A Bermudan put holder will decide if he or she wants to exercise the contract and get payoff$(x) = K-S_{t_i}$ in line 5, or keep the contract with the value unchanged.
An up-and-out call is set to 0 at observation point $t_i$ if $x>\log(B/S_0)$, or equivalently $S_{t_i} > B$.
Before line 5, the function put$(\cdot, x)$ or call$(\cdot, x)$ keeps the value of the option at $t_i+$;
after line 5, it keeps the value at $t_i-$.
The discounted conditional expected value is evaluated in line 6 by a convolution,
or more explicitly,
\begin{eqnarray}
\text{put}(t_{i-1}, x) &=& e^{-r\Delta t}\int_{-\infty}^\infty \text{put}(t_{i}, y)p(x-y)\,dy, \label{eq:line6conv}\\
\text{call}(t_{i-1}, x) &=& e^{-r\Delta t}\int_{-\infty}^\infty \text{call}(t_{i}, y)p(x-y)\,dy, \label{eq:line6convBarrier}
\end{eqnarray}
for Figures \ref{algo:conv_berm} and \ref{algo:conv_barrier}, respectively.
Finally when the loop terminates, the option value at $x=0$ is returned.
Many convolution pricing algorithm has a similar form as in Figures \ref{algo:conv_berm} and \ref{algo:conv_barrier}.
In those algorithms, one just needs to change the transition density function $p(x)$ according to the model,
and to change line 5 according to the type of the option to be evaluated.




To implement a convolution pricing algorithm, one first sets up a wide interval $[-a, a]$ and a uniform grid in the interval.
Suppose the grid points are $-a = x_0 < x_1 < x_2 \cdots < x_m = a$, where $x_j = x_0 + jh$, $h = x_1 - x_0$.
% For simplicity, further assume that $m$ is even so that $x_{m/2} = 0$.
% This way when we return $\text{call}(t_0, 0)$ or $\text{put}(t_0, 0)$ in line 8 in Figure \ref{algo:CONV}, since 0 is one of the grid points, no interpolation is needed.
In the implementation, all the functions are kept as an array of function values at $x_0, x_1, \ldots, x_m$.
This is called a grid function.
For example, the transition density $p(x)$ in line 3 in Figures \ref{algo:conv_berm} and \ref{algo:conv_barrier} is stored as an array $\{p(x_0), p(x_1), \ldots, p(x_m)\}$.
To evaluate the convolution in line 6, or (\ref{eq:line6conv}) and (\ref{eq:line6convBarrier}), numerical integration and discrete convolution are involved,
which is typically done by the FFT algorithm in $O(m\log m)$ computational time.
Note that line 6 is the only place in the algorithm where we need a numerical approximation.
Indeed, the payoff function and the transition density both have a closed-form formula,
and at each $x_j$, the two statements in line 5
\begin{eqnarray*}
\text{put}(t_i, x_j) &\leftarrow& \max(\text{ put}(t_i, x_j), \text{ payoff}(x_j)~)\\
\text{call}(t_i, x_j) &\leftarrow& \text{call}(t_i, x_j)\times 1_{\{x_j<\log(B/S_0)\}}
\end{eqnarray*}
will not introduce any numerical error.



\subsection{Control Variate-Enhanced Algorithms}
In every convolution-based pricing algorithm there is a convolution step like line 6 in Figures \ref{algo:conv_berm} and \ref{algo:conv_barrier}.
Since convolution is a linear operator, its numerical evaluation can be enhanced by the CV technique,
and if we choose the CV appropriately, the resulting algorithm will have better accuracy.
Although we only improve the algorithms in Figures \ref{algo:conv_berm} and \ref{algo:conv_barrier} as examples,
the statement below is general and works for all convolution-based pricing algorithms.


% �U���n�A���p�`���g�k
%
% Recall that every convolution pricing algorithm has a similar form as Figures \ref{algo:conv_berm} and \ref{algo:conv_barrier}, just line 3 and 5 are different.
% Below, we will first give a statement on how the CV technique can be applied to a convolution pricing algorithm,
% where only line 6 is modified.
% Thus the statement is general and works for all convolution pricing algorithms.
% We then discuss possible CVs for Bermudan put and discrete barrier call option separately.





\begin{figure}[!t]
\begin{framed}
\begin{algorithmic}[1]
    \STATE payoff$(x) \leftarrow \max(K - S_0e^x, 0)$
    \STATE put$(t_n, x) \leftarrow$ payoff$(x)$
    \STATE $p(x) \leftarrow e^{-(x+(r-\sigma^2/2)\Delta t)^2/(2\sigma^2\Delta t)}/(\sigma\sqrt{2\pi\Delta t})$
    \FOR{$i=n$ \textbf{downto} $1$ }
        \STATE $\text{put}(t_i, x) \leftarrow \max(\text{ put}(t_i, x), \text{ payoff}(x)~)$
        \STATE $\text{put}_{\text{diff}}(t_{i}, x) \leftarrow \text{put}(t_{i}, x) - \text{put}_{\text{CV}}(t_{i}, x)$
        \STATE $ \text{put}(t_{i-1}, x) \leftarrow e^{-r\Delta t}( \text{put}_{\text{diff}}(t_i, x)\otimes p(x) ) + e^{-r\Delta t}( \text{put}_{\text{CV}}(t_i, x)\otimes p(x) )$
    \ENDFOR
    \STATE \textbf{return} $\text{put}(t_0, 0)$
\end{algorithmic}
\end{framed}
\caption{Bermudan Put Pricing Algorithm with Control Variate.
The CV $\text{put}_{\text{CV}}(t_{i}, x)$ is chosen in a way that $e^{-r\Delta t}( \text{put}_{\text{CV}}(t_i, x)\otimes p(x) )$ is known in closed form.
Thus only the first convolution in line 7 needs a numerical approximation.
}
\label{algo:conv_berm_CV}
\end{figure}




%As mentioned earlier, the convolution step in line 6 in Figures \ref{algo:conv_berm} and \ref{algo:conv_barrier} is the only place in the algorithms where a numerical approximation is needed.
As mentioned earlier, the convolution step in line 6 in Figures \ref{algo:conv_berm} and \ref{algo:conv_barrier} is the only source of numerical error.
Take the Bermudan put pricing as an example.
We use a discrete convolution to approximate (\ref{eq:line6conv}),
which is the only part that needs to be taken care of in order to improve the accuracy of the algorithm.
Set $\mathsf T(f) = f\otimes p$, the ``convolve with $p(x)$'' operator.
Denote by $\mathbb T$ the operator that approximates $\mathsf T$.
It is now a discrete convolution with $p(x)$.
Since $\mathsf T$ is linear, we can apply the CV technique and use the better approximation $\mathbb T(f-g) + \mathsf T(g)$ instead of $\mathbb T(f)$, as illustrated in (\ref{eq:GCVapprox}).
In the context here, to evaluate (\ref{eq:line6conv}),
first pick a control $\text{put}_{\text{CV}}(t_i, y)$, evaluate the difference $\text{put}_{\text{diff}}(t_i, y) = \text{put}(t_i, y) - \text{put}_{\text{CV}}(t_i, y)$,
and then rewrite (\ref{eq:line6conv}) as
\begin{eqnarray}
\text{put}(t_{i-1}, x) = e^{-r\Delta t}\int_{-\infty}^\infty \text{put}_{\text{diff}}(t_{i}, y)p(x-y)\,dy + e^{-r\Delta t}\int_{-\infty}^\infty \text{put}_{\text{CV}}(t_{i}, y)p(x-y)\,dy.\label{eq:line6GCV}
\end{eqnarray}
Now the discrete convolution is only needed for the first integral.
%As have been discussed in earlier sections, there are two conditions a CV should satisfy.
%In the example here, the conditions $\text{put}_{\text{CV}}(t_i, y)$ should satisfy are
The control $\text{put}_{\text{CV}}(t_i, y)$ should satisfy
\begin{enumerate}
\item $\text{put}(t_i, y)$ is well-approximated by $\text{put}_{\text{CV}}(t_i, y)$, and
\item $\mathsf T( \text{put}_{\text{CV}} ) = \int_{-\infty}^\infty \text{put}_{\text{CV}}(t_{i}, y)p(x-y)\,dy$ is known in closed form.
\end{enumerate}
%The second condition ensures that we do not need a numerical approximation for the second integral in (\ref{eq:line6GCV}),
The first condition guarantees a small error when we apply the discrete convolution on the first integral in (\ref{eq:line6GCV}),
and the second condition precludes the need of a numerical approximation for the second integral. % in (\ref{eq:line6GCV}),
Unlike the enhanced FT-based pricing algorithms where only one CV is needed,
here we need $n$ CVs: $\text{put}_{\text{CV}}(t_1, y), \text{put}_{\text{CV}}(t_2, y), \ldots, \text{put}_{\text{CV}}(t_n, y)$,
one for each observation point $t_i$.
The enhanced Bermudan put pricing algorithm is given in Figure \ref{algo:conv_berm_CV}.




The only difference between the original algorithm given in Figures \ref{algo:conv_berm} and \ref{algo:conv_barrier} and the enhanced version given in Figure \ref{algo:conv_berm_CV} is that
line 6 in the original algorithm is now rewritten as line 6 and 7 in the CV-enhanced algorithm.
We emphasize that in the CV-enhanced algorithm, $e^{-r\Delta t}( \text{put}_{\text{CV}}(t_i, x)\otimes p(x) )$ is known in closed form.
This is the second condition that must be satisfied by $\text{put}_{\text{CV}}(t_{i}, x)$.
We must have the closed-form formulas for $\text{put}_{\text{CV}}(t_{i}, x)$ and $e^{-r\Delta t}( \text{put}_{\text{CV}}(t_i, x)\otimes p(x) )$
so that we can take the grid functions when implementing line 6 and 7.




The discrete barrier option pricing algorithm in Figure \ref{algo:conv_barrier} can also be enhanced in a similar manner.
One only needs to rewrite (\ref{eq:line6convBarrier}) as
\begin{eqnarray}
\text{call}(t_{i-1}, x) = e^{-r\Delta t}\int_{-\infty}^\infty \text{call}_{\text{diff}}(t_{i}, y)p(x-y)\,dy + e^{-r\Delta t}\int_{-\infty}^\infty \text{call}_{\text{CV}}(t_{i}, y)p(x-y)\,dy, \label{eq:line6GCVBarrier}
\end{eqnarray}
where $\text{call}_{\text{CV}}(t_{i}, y)$ is the CV,
and $\text{call}_{\text{diff}}(t_i, y) = \text{call}(t_i, y) - \text{call}_{\text{CV}}(t_i, y)$ the difference.
The functions $\text{call}_{\text{CV}}(t_{i}, y)$ should be chosen in a way that the following two conditions are satisfied:
\begin{enumerate}
\item $\text{call}(t_i, y)$ should be well-approximated by $\text{call}_{\text{CV}}(t_i, y)$, and
\item $\mathsf T( \text{call}_{\text{CV}} ) = \int_{-\infty}^\infty \text{call}_{\text{CV}}(t_{i}, y)p(x-y)\,dy$ is known in closed form.
\end{enumerate}
The CV-enhanced algorithm is shown in Figure \ref{algo:conv_barrier_CV}.
Obviously the same methodology can be used to enhance all convolution-based pricing algorithms.







\begin{figure}[!t]
\begin{framed}
\begin{algorithmic}[1]
    \STATE payoff$(x) \leftarrow \max(S_0e^x - K, 0)$
    \STATE call$(t_n, x) \leftarrow$ payoff$(x)$
    \STATE $p(x) \leftarrow e^{-(x+(r-\sigma^2/2)\Delta t)^2/(2\sigma^2\Delta t)}/(\sigma\sqrt{2\pi\Delta t})$
    \FOR{$i=n$ \textbf{downto} $1$ }
        \STATE $\text{call}(t_i, x) \leftarrow \text{call}(t_i, x)\times 1_{\{x<\log(B/S_0)\}}$
        \STATE $\text{call}_{\text{diff}}(t_{i}, x) \leftarrow \text{call}(t_{i}, x) - \text{call}_{\text{CV}}(t_{i}, x)$
        \STATE $ \text{call}(t_{i-1}, x) \leftarrow e^{-r\Delta t}( \text{call}_{\text{diff}}(t_i, x)\otimes p(x) ) + e^{-r\Delta t}( \text{call}_{\text{CV}}(t_i, x)\otimes p(x) )$
    \ENDFOR
    \STATE \textbf{return} $\text{call}(t_0, 0)$
\end{algorithmic}
\end{framed}
\caption{Discrete Up-And-Out Call Pricing Algorithm with Control Variate.
The CV $\text{call}_{\text{CV}}(t_{i}, x)$ is chosen in a way that $e^{-r\Delta t}( \text{call}_{\text{CV}}(t_i, x)\otimes p(x) )$ is known in closed form.
Thus only the first convolution in line 7 needs a numerical approximation.
}
\label{algo:conv_barrier_CV}
\end{figure}



\subsection{Picking a Control Variate for Bermudan Puts Pricing}




\begin{figure}[!t]
\begin{center}
    \parbox[t]{0.45\textwidth}{$\text{put}(t_{i}, x)$ \& $\text{put}_{\text{CV}}(t_{i}, x)$\\}
    \parbox[t]{0.45\textwidth}{$\text{put}_{\text{diff}}(t_{i}, x)$\\}
    \parbox[t]{0.45\textwidth}{
        \centerline{\epsfig{figure=eps/nsBSCV.eps, width=0.45\textwidth}}
    }
    \parbox[t]{0.45\textwidth}{
        \centerline{\epsfig{figure=eps/nsBSDiff.eps, width=0.45\textwidth}}
    }
\end{center}
\caption{Bermudan Put Pricing with European Put as a Control.
In the left panel are the graphs of the time $t_i$ value of a Bermudan put (solid line) and a European put (dashed line).
The graph of the difference is shown in the right panel.
}
\label{fig:cvBS}
\end{figure}




For Bermudan puts pricing, a CV of put$(t_i, x)$ is a function $\text{put}_{\text{CV}}(t_i, x)$ such that it approximates put$(t_i, x)$ and
$e^{-r\Delta t}(\text{put}_{\text{CV}}(t_i, x)\otimes p(x))$ is known in closed form.
A possible choice is the Black-Scholes formula at time $t_i$, although we need to rewrite it in terms of $x = \log(S_{t_i}/S_0)$, that is
\begin{eqnarray*}
\text{put}_{\text{CV}}(t_i, x) &=& K e^{-r\tau_i} N\left( - \frac{\log\frac{S_{t_i}}{K} + \left(r - \frac{\sigma^2}{2}\right)\tau_i}{\sigma\sqrt{\tau_i}} \right)
                                    - S_{t_i} N\left( - \frac{ \log\frac{S_{t_i}}{K} + \left(r + \frac{\sigma^2}{2}\right)\tau_i}{\sigma\sqrt{\tau_i}} \right)\\
                                &=& K e^{-r\tau_i} N\left( - \frac{x + \log\frac{S_0}{K} + \left(r - \frac{\sigma^2}{2}\right)\tau_i}{\sigma\sqrt{\tau_i}} \right)
                                    -S_0e^x N\left( - \frac{x + \log\frac{S_0}{K} + \left(r + \frac{\sigma^2}{2}\right)\tau_i}{\sigma\sqrt{\tau_i}} \right),
\end{eqnarray*}
where $\tau_i = T - t_i$.
By using this CV, we are approximating a Bermudan puts by European puts.
At first glance it seems not obvious why $e^{-r\Delta t}(\text{put}_{CV}(t_i, x)\otimes p(x))$ is known in closed form,
but due to Theorem \ref{thm:CONV} we know the convolution is equivalent to taking a conditional expectation.
Thus
\begin{eqnarray*}
e^{-r\Delta t}( \text{put}_{\text{CV}}(t_i, x)\otimes p(x) )    &=& e^{-r\Delta t}\tilde{\mathsf E}[\text{European put at $t_i$}|\mathcal F_{t_{i-1}}]\\
                                                                &=& e^{-r\Delta t}\tilde{\mathsf E}[ e^{-r \tau_i}\tilde{\mathsf E}[ (K-S_T)^+ |\mathcal F_{t_{i}}] |\mathcal F_{t_{i-1}}]\\
                                                                &=& e^{-r\tau_{i-1}}\tilde{\mathsf E}[(K-S_T)^+|\mathcal F_{t_{i-1}}],
\end{eqnarray*}
which is a European put at time $t_{i-1}$, so
\begin{eqnarray*}
&&e^{-r\Delta t}(\text{put}_{\text{CV}}(t_i, x)\otimes p(x))\\
&&= K e^{-r\tau_{i-1}} N\left( - \frac{x + \log\frac{S_0}{K} + \left(r - \frac{\sigma^2}{2}\right)\tau_{i-1}}{\sigma\sqrt{\tau_{i-1}}} \right)
-S_0e^x N\left( - \frac{x + \log\frac{S_0}{K} + \left(r + \frac{\sigma^2}{2}\right)\tau_{i-1}}{\sigma\sqrt{\tau_{i-1}}} \right).
\end{eqnarray*}
We just derived the closed-form formula of the convolution by writing it as a conditional expectation first.
We will use this trick again later when we derive the closed-form formula for the convolution for another CV.




Numerical results provided later show that with this control, the CV-enhanced algorithm is already better than the plain convolution algorithm.
However, from Figure \ref{fig:cvBS} we can see that the CV and the Bermudan put are not very similar.
Below we construct a better CV, a CV that makes the difference smaller.
As the numerical results show us, we get even more accuracy with this new CV.




Recall that at each observation point $t_i$, the option holder gets to decide if he or she wants to exercise the contract.
If $S_{t_i}$ is too low, the contract will be exercised to get $K - S_{t_i}$ immediately.
There is a key value $S_{t_i}^*$, the so-called ``early exercise boundary'',
such that if $S_{t_i}$ drops below this key value then the Bermudan put should be exercised and its value is just the payoff function $K - S_{t_i}$.
The key value in log scale is $x^* = \log(S_{t_i}/S_0)$.
In line 5 of the algorithm in Figure \ref{algo:conv_berm}, put$(t_i, x)$ is set to payoff$(x)$ on the left-hand side of $x^*$;
on the right-hand side, put$(t_i, x)$ remains unchanged.
We use a piecewise smooth function as the CV.
First, with the grid functions of put$(t_i, x)$ and payoff$(x)$, we numerically determine the value of $x^*$.
On the left-hand side of $x^*$ the CV is set to the payoff, and hence the difference is zero.
As for the right-hand side, from Figure \ref{fig:cvBS} we see that the Bermudan put is a decreasing convex function,
so we use a scaled exponential function $c_1 e^{c_2 x}$ with $c2 < 0$ as the control.
To sum up, the control is
\begin{eqnarray}
(K - S_0 e^x)1_{\{x<x^*\}} + c_1 e^{c_2 x}1_{\{x\geq x^*\}}.\label{eq:CVem}
\end{eqnarray}
We choose $c_1$, $c_2$ such that the difference is continuous and differentiable.
To achieve the continuity and smoothness we estimate put$(t_i, x^*)$ and put$_x(t_i, x^*)$ with interpolation and the finite difference method.
The graphs of the time $t_i$ value of a Bermudan put, the CV and the difference are shown in Figure \ref{fig:CVem}.




\begin{figure}[!t]
\begin{center}
    \parbox[t]{0.45\textwidth}{$\text{put}(t_{i}, x)$ \& $\text{put}_{\text{CV}}(t_{i}, x)$\\}
    \parbox[t]{0.45\textwidth}{$\text{put}_{\text{diff}}(t_{i}, x)$\\}
    \parbox[t]{0.45\textwidth}{
        \centerline{\epsfig{figure=eps/emCV.eps, width=0.45\textwidth}}
    }
    \parbox[t]{0.45\textwidth}{
        \centerline{\epsfig{figure=eps/emDiff.eps, width=0.45\textwidth}}
    }
\end{center}
\caption{Bermudan Put with (\ref{eq:CVem}) as a Control.
In the left panel are the graphs of a Bermudan put (solid line) and the CV given by (\ref{eq:CVem}) (dashed line).
The graph of the difference is shown in the right panel.
}
\label{fig:CVem}
\end{figure}





To use (\ref{eq:CVem}) as a CV, we still need the closed-form formula for
$$
e^{-r\Delta t}\left(p(x) \otimes \left((K - S_0 e^x)1_{\{x<x^*\}} + c_1 e^{c_2 x}1_{\{x\geq x^*\}}\right)\right).
$$
Since the CV is piecewise defined, and the ``convolve with $p(x)$'' operator is linear,
the closed-form formula can be found term by term, meaning we find
$$
e^{-r\Delta t}\left(p(x) \otimes (K-S_0e^x)1_{\{x<x^*\}}\right) \qquad\text{and}\qquad e^{-r\Delta t}\left(p(x) \otimes c_1e^{c_2x}1_{\{x\geq x^*\}}\right)
$$
separately.
The second convolution can be evaluated directly as follows
\begin{eqnarray*}
p(x) \otimes c_1e^{c_2x}1_{\{x\geq x^*\}} &=&   c_1 \left( p(x) \otimes e^{c_2x}1_{\{x\geq x^*\}} \right)\\
&=&  c_1 \int_{-\infty}^{\infty} \frac{1}{\sigma\sqrt{2\pi\Delta t}} e^{-\frac{1}{2} \left( \frac{y-x+(r-\sigma^2 /2)\Delta t}{\sigma\sqrt{\Delta t}}\right)^2}e^{c_2x}1_{\{x\geq x^*\}} \,dx\\
&=& c_1 \int_{x^*}^{\infty} \frac{1}{\sigma\sqrt{2\pi\Delta t}} e^{-\frac{1}{2} \left( \frac{x-(r-\sigma^2 /2)\Delta t -y-c_2\sigma^2{\Delta t}}{\sigma\sqrt{\Delta t}}\right)^2} e^{-\frac{c_2}{2} \left( -2(r-\sigma^2 /2)\Delta t -2y -c_2\sigma^2{\Delta t}\right)}\,dx\\
&=& c_1 e^{c_2\left(y+(r-\sigma^2 /2)\Delta t +\frac{c_2 \sigma^2{\Delta t}}{2}\right)} N\left(\frac{y+c_2\sigma^2{\Delta t} + (r-\sigma^2 /2)\Delta t-x^*}{\sigma\sqrt{\Delta t}}\right).
\end{eqnarray*}
For the first convolution $e^{-r\Delta t}\left(p(x) \otimes (K-S_0e^x)1_{\{x<x^*\}}\right)$,
by Theorem \ref{thm:CONV} it can be rewritten as the conditional expectation
$$
e^{-r\Delta t} \tilde{\mathsf E}[(K-S_{t_i})1_{\{x<x^*\}}|\mathcal F_{t_{i-1}}].
$$
This is similar to a vanilla put option.
The closed-form formula is
\begin{eqnarray*}
&& K e^{-r\Delta t} N\left( - \frac{\log\frac{S_{t_{i-1}}}{S_{t_i}^*} + \left(r - \frac{\sigma^2}{2}\right)\Delta t}{\sigma\sqrt{\Delta t}} \right)
-S_{t_{i-1}}N\left( - \frac{\log\frac{S_{t_{i-1}}}{S_{t_i}^*} + \left(r + \frac{\sigma^2}{2}\right)\Delta t}{\sigma\sqrt{\Delta t}} \right)\\
&&= K e^{-r\Delta t} N\left( - \frac{ x - x^* + \left(r - \frac{\sigma^2}{2}\right)\Delta t}{\sigma\sqrt{\Delta t}} \right)
-S_0e^xN\left( - \frac{ x - x^* + \left(r + \frac{\sigma^2}{2}\right)\Delta t}{\sigma\sqrt{\Delta t}} \right).
\end{eqnarray*}





\subsection{Picking a Control Variate for Discrete Barrier Options Pricing}
\label{sec:findingCVforBarrier}
%For discrete up-and-out call options pricing, one possible choice of the CV is to
%use another up-and-out call option that is only allowed to knock out at one observation point,
%that is
%\begin{eqnarray*}
%\text{call}_{\text{CV}}(t_{i}, x) &=& \left[S_0e^x N\left( \frac{x + \log\frac{S_0}{K} + \left(r + \frac{\sigma^2}{2}\right)\tau_i}{\sigma\sqrt{\tau_i}} \right)\right.\\
%&&\qquad \left.- K e^{-r\tau_i} N\left( \frac{x + \log\frac{S_0}{K} + \left(r - \frac{\sigma^2}{2}\right)\tau_i}{\sigma\sqrt{\tau_i}} \right)\right]\times 1_{\{x<\log(B/S_0)\}},
%\end{eqnarray*}
%which is the value at $t_i-$ of an up-and-out call option that is only allowed to knock out at $t_i$.
%With Theorem \ref{thm:CONV}, $e^{-r\Delta t}( \text{call}_{\text{CV}}(t_i, x)\otimes p(x) )$ can be rewritten as
%\begin{eqnarray*}
%e^{-r\Delta t}( \text{call}_{\text{CV}}(t_i, x)\otimes p(x) )    &=& e^{-r\Delta t}\tilde{\mathsf E}\left[(\text{European call at $t_i$})\times 1_{\{x<\log(B/S_0)\}}\big|\mathcal F_{t_{i-1}}\right]\\
%                                                                &=& e^{-r\tau_{i-1}}\tilde{\mathsf E}[ \tilde{\mathsf E}\left[ (S_T - K)^+ |\mathcal F_{t_{i}}]\times 1_{\{x<\log(B/S_0)\}} \big|\mathcal F_{t_{i-1}}\right],
%\end{eqnarray*}
%which is the value of the same option at $t_{i-1}$, also known in closed form.
%The derivation of the pricing formula is similar to that of a two-period barrier option pricing formula.
%
%
%
%Another way to construct a CV is to use a scaled normal density function with numerically estimated parameters.
For discrete up-and-out call options pricing, one possible choice of the CV is a scaled normal density function with numerically estimated parameters.
The same technique can also be used to construct the CV for Asian options pricing, as will be discussed in later sections.
This idea relies on the assumption that the function to be approximated by the CV has a bell shape,
which is the case for the up-and-out barrier call options pricing algorithm.
Indeed, in the original convolution pricing algorithm in Figure \ref{algo:conv_barrier},
before the loop call$(t_n, x)$ is the payoff function which goes up exponentially fast as $x$ increases.
Line 5 sets the function values to zero for $x>\log(B/S_0)$.
After line 5, the graph of call$(t_n, x)$ would be like a triangle that is only nonzero in the interval $(\log(K/S_0), \log(B/S_0)]$.
It is then sent to be convolved with a gaussian kernel $p(x)$ in line 6.
The resulting function call$(t_{n-1}, x)$ will be bell-shaped and can be approximated by a scaled normal density function of the form
\begin{eqnarray}
\frac{a_{n-1}}{s_{n-1}\sqrt{2\pi}}e^{-\frac{1}{2}\left(\frac{x-m_{n-1}}{s_{n-1}}\right)^2}.\label{eq:scaledBellShape}
\end{eqnarray}
This approximation has three parameters $a$, $m_{n-1}$ and $s_{n-1}$,
representing the area under the curve, the mean and the standard deviation of the normal density, respectively.
We estimate the parameters by numerically integrating the bell-shaped grid functions.
For example, as the area under call$(t_{n-1}, x)$,
$a_{n-1}$ can be estimated by numerically integrating call$(t_{n-1}, x)$.
Once the value of $a_{n-1}$ is determined, the normalized bell-shaped curve $\text{call}(t_{n-1}, x)/a_{n-1}$ has area 1 and therefore is a probability density function,
the first and the second raw moments of which can be found by numerically integrating $x\cdot\text{call}(t_{n-1}, x)/a_{n-1}$ and $x^2\cdot\text{call}(t_{n-1}, x)/a_{n-1}$, respectively.
We get $m_{n-1}$ directly from the first moment.
To get an estimate of $s_{n-1}$, one just need to switch from the raw moment to the central moment.




Having found the parameters of (\ref{eq:scaledBellShape}),
we have a closed-form approximation of the $\text{call}(t_{n-1}, x)$ function in the end of the first iteration of the loop.
The following function can be used as a CV in the second iteration of the loop
\begin{eqnarray*}
\text{call}_{\text{CV}}(t_{n-1}, x) = \left(\frac{a_{n-1}}{s_{n-1}\sqrt{2\pi}}e^{-\frac{1}{2}\left(\frac{x-m_{n-1}}{s_{n-1}}\right)^2}\right)\times 1_{\{x<\log(B/S_0)\}}.
\end{eqnarray*}
In general, a CV of this form can be used for all indices $i \leq n-1$, that is
$$
\text{call}_{\text{CV}}(t_{i}, x) = \left(\frac{a_{i}}{s_{i}\sqrt{2\pi}}e^{-\frac{1}{2}\left(\frac{x-m_{i}}{s_{i}}\right)^2}\right)\times 1_{\{x<\log(B/S_0)\}}\qquad\forall i = n-1, n-2, \ldots, 1,
$$
because after the convolution step in line 6 of Figure \ref{algo:conv_barrier} we always get a bell-shaped curve.
The parameters $a_i$, $m_i$ and $s_i$ can be estimated by numerically integrating the bell-shaped function, as described above.
In the first iteration with index $i=n$, however, we can not use CV of this from
since no convolution step has been executed and therefore no bell-shaped function can be approximated.
For $i\neq n$, the closed-form formula of $e^{-r\Delta t}(\text{call}_{\text{CV}}(t_{i}, x)\otimes p(x))$ can be evaluated by the following lemma,
which can be proved by straightforward computation.



\begin{lemma}
Let $f$ and $g$ be the probability density function of the normal distributions $N(\mu_1, \sigma_1)$ and $N(\mu_2, \sigma_2)$, respectively.
Then for constant $L$,
$$
f \otimes \left(g\cdot 1_{\{x<L\}}\right) = \frac{1}{\sqrt{2\pi(\sigma_1^2 + \sigma_2^2)}} e^{-\frac12\frac{(x - m_1 - m_2)^2}{\sigma_1^2 + \sigma_2^2}}
\left[1 - N\left( - \frac{(L-m_2)\sigma_1^2 + (L-x+m_1)\sigma_2^2 }{\sigma_1\sigma_2\sqrt{\sigma_1^2 + \sigma_2^2}}\right)\right].
$$
\end{lemma}




\subsection{Benhamou's Algorithm for Asian Options Pricing}
Unlike previously discussed convolution-based pricing algorithms where we evaluate the grid functions of the option values backward in time,
The Asian options pricing algorithm proposed by \citet{benhamou2002fast} evaluates probability density functions forward in time.
Benhamou's algorithm is a variant of the algorithm by \citet{carverhill1990flexible}, which is a classic example of convolution-based pricing algorithms.
Like other convolution-based pricing algorithms, Benhamou's algorithm can also be enhanced by the CV technique, as we discuss in the following sections.




Consider an discretely monitored Asian call option with maturity $T$ and strike price $K$.
As usual, we assume that the underlying asset price at time $t$ is $S_t$.
Assume that, through out the life of this option, the underlying asset price is observed $n$ times at $t_1 < t_2 < \cdots < t_n = T$.
The option is written on the average of the observed prices and the initial price $S_0$.
Thus the fair price is
\begin{eqnarray}
e^{-rT}\tilde{\mathsf E}\left[\left(\frac{1}{n+1} \sum_{j=0}^n S_{t_{j}} - K\right)^+\right].\label{eq:AsianOptionValue}
\end{eqnarray}
For simplicity, we assume that the observation points are equally distant,
so there is a constant time interval $\Delta t = t_i - t_{i-1}$ for all $i=1, 2, \ldots, n$.
In principle it is not possible to derive a closed-form pricing formula for Asian options,
not even under the most analytically tractable Black-Scholes model.
Some numerical scheme is needed for its pricing problem.



Benhamou's algorithm first numerically approximate the probability density of the average price $A$,
and then evaluate the option price (\ref{eq:AsianOptionValue}) with numerical integration.
The starting point is the following decomposition of $A$.
\begin{eqnarray}
   A &=& \frac{1}{n+1}\sum\limits^n_{i=0} S_{t_j} = \frac{1}{n+1} \left(S_{t_0} + S_{t_1} + S_{t_2} + \cdots + S_{t_n}\right)\label{eq:A_sum}\\
     &=& \frac{S_{t_0}}{n+1}\left(1+X_1+X_1X_2+\cdots+X_1X_2\cdots X_n\right)\nonumber\\
     &=& \frac{S_{t_0}}{n+1}\left(1+X_1\left(1+X_2\left(\cdots X_{n-1}\left(1+X_n\right)\right)\right)\right), \nonumber
\end{eqnarray}
where $X_i = S_{t_i}/S_{{t_{i-1}}}$ is the return of the underlying asset.
Denote $R_i = \log X_i$ as the log return.
Under the Black-Scholes model, $X_i$'s are iid log-normal and $R_i$'s are iid normal.
Now recursively define the sequence $\{\mu_n\}$ as follows
\begin{eqnarray}
     \mu_{n-1}   &=& \tilde{\mathsf E}[R_n]\label{eq:mu_init}\\
     \mu_{j-1}   &=& \tilde{\mathsf E}[R_{j}]+\log (1+\exp (\mu_{j}))\qquad j=n-1, n-2, \ldots , 1.\label{eq:mu_rec}
\end{eqnarray}
Also introduce two random sequences
\begin{eqnarray*}
D_j &=& \log(X_{j+1}(1+X_{j+2}(\ldots X_{n-2}(1+X_{n-1}(1+ X_n))))) - \mu_j, \\
Z_j &=& \log(1 + X_{j+1}(1+X_{j+2}(\ldots X_{n-2}(1+X_{n-1}(1+ X_n))))) - \mu_{j-1}.
\end{eqnarray*}
With the above notations it is not hard to verify the following identities.
The random sequence $\{D_j\}$ satisfies the recurrence relation
\begin{eqnarray}
\begin{cases}
D_n = R_n - \tilde{\mathsf E}[R_n],\\
D_{j-1} = R_{j} + Z_j \qquad j=n-1, n-2, \ldots , 1
\end{cases},\label{eq:benh_rec}
\end{eqnarray}
the density function of $Z_j$ can be written in terms of the density function of $D_j$ as
\begin{equation}
f_{Z_{j}}(x) =
\begin{cases}
f_{D_{j}}\left(\log (e^{x+\mu_{j-1}}-1) - \mu_j\right)\frac{e^{x+\mu_{j-1}}}{e^{x+\mu_{j-1}}-1}&\mbox{ if }x>0\\
0 &\mbox{ otherwise }
\end{cases},                \label{eq:D2Zdensity}
\end{equation}
and finally the average price $A$ can be written as
\begin{eqnarray}
A = \frac{S_{t_0}}{n+1}\left(1+e^{D_0 + \mu_0}\right).\label{eq:AB0}
\end{eqnarray}
Now a strategy to obtain the density function of $A$ is clear.
One can start from $f_{D_n}$, the density function of $D_n$, which is normal distributed with mean 0 by (\ref{eq:benh_rec}).
With $f_{D_n}$ one can evaluate $f_{Z_n}$ by (\ref{eq:D2Zdensity}).
By (\ref{eq:benh_rec}) we known $D_{n-1} = R_{n} + Z_n$, and, since $R_{n}$ and $Z_n$ are independent,
$f_{D_{n-1}}$ is the convolution of $f_{R_{n}}$ and $f_{Z_n}$.
This process can be used repeatedly to find $f_{D_{n-2}}, f_{D_{n-3}}, \ldots, f_{D_{0}}$.
Finally by (\ref{eq:AB0}), to obtain the density function of $A$ from $f_{D_{0}}$ one only needs a change of variable.




Closed-form density functions are not possible to obtain.
All the density functions of $D_j$'s need to be approximated numerically by grid functions.
First set up a wide interval $[-a, a]$ and a uniform grid $\{x_i\}$, where $-a = x_0 < x_1 < x_2\cdots < x_m = a$.
For any function $f$, the algorithm only keeps its values on the grid, that is $\{f(x_0), f(x_1), f(x_2), \ldots, f(x_m)\}$.
This is similar to the previously discussed convolution-based pricing algorithms.
Benhamou's algorithm is shown in Figure \ref{algo:conv_asian_benh}.





\begin{figure}[!t]
\begin{framed}
\begin{algorithmic}[1]
    \STATE Evaluate $\{\mu_j\}$ by (\ref{eq:mu_init}) and (\ref{eq:mu_rec})
    \STATE $f_R \leftarrow \text{grid function of } e^{-(x-(r-\sigma^2/2)\Delta t)^2/(2\sigma^2\Delta t)}/(\sigma\sqrt{2\pi\Delta t})$
    \STATE $f_{D_{n-1}} \leftarrow \text{grid function of } e^{-x^2/(2\sigma^2\Delta t)}/(\sigma\sqrt{2\pi\Delta t})$
    \FOR{$j=n-1$ \textbf{downto} $1$ }
        \STATE Given $f_{D_{j}}$, evaluate $f_{Z_j}$ by (\ref{eq:D2Zdensity}) and interpolation
        \STATE $f_{D_{j-1}} \leftarrow f_R \otimes f_{Z_j}$
    \ENDFOR
    \STATE Given $f_{D_0}$, evaluate $f_A$ by (\ref{eq:AB0}) and interpolation
    \STATE \textbf{return} $e^{-rT}\tilde{\mathsf E}[(A-K)^+]$
\end{algorithmic}
\end{framed}
\caption{Benhamou's Algorithm for Asian Options Pricing
} \label{algo:conv_asian_benh}
\end{figure}



\subsection{Control Variate-Enhanced Algorithm}



Benhamou's algorithm in Figure \ref{algo:conv_asian_benh} has a similar structure as previously discussed convolution-based pricing algorithms.
Thus it can be enhanced by the CV technique in the same methodology as before.
Take the Bermudan options pricing algorithm in Figure \ref{algo:conv_berm} as an example.
In both algorithms in Figures \ref{algo:conv_asian_benh} and \ref{algo:conv_berm},
line 6 is the convolution step which is linear and can be enhanced by the CV technique.
Specifically, line 6 in Benhamou's algorithm in Figure \ref{algo:conv_asian_benh} can be viewed as the ``convolve with $f_R$'' operator acting on $f_{Z_j}$.
Thus we can enhance the convolution step like we did in the CV-enhanced Bermudan options pricing algorithm in Figure \ref{algo:conv_berm_CV}.
First we pick a CV, a function $f_{Z_{j}}^{\text{CV}}$ that satisfies the two conditions:
\begin{enumerate}
  \item $f_{Z_{j}}$ is well-approximated by $f_{Z_{j}}^{\text{CV}}$, and
  \item both $f_{Z_{j}}^{\text{CV}}$ and $f_{Z_{j}}^{\text{CV}}\otimes f_{R_{j}}$ are known in closed form.
\end{enumerate}
We then evaluate the grid function of the difference $f_{Z_{j}}^{\text{diff}} = f_{Z_{j}} - f_{Z_{j}}^{\text{CV}}$
and replace line 6 in Benhamou's algorithm by the decomposition
\begin{eqnarray*}
f_{D_{j-1}} = f_{R_{j}} \otimes f_{Z_{j}}^{\text{CV}} + f_{R_{j}} \otimes f_{Z_{j}}^{\text{diff}}.
\end{eqnarray*}
The resulting algorithm is shown in Figure \ref{algo:conv_asian_benh_CV_1}.
This is very similar to the enhanced Bermudan options pricing algorithm in Figure \ref{algo:conv_berm_CV}.
Now we have two convolutions $f_{R_{j}} \otimes f_{Z_{j}}^{\text{CV}}$ and $f_{R_{j}} \otimes f_{Z_{j}}^{\text{diff}}$.
The first one is known in closed form; it does not require any numerical approximation and hence gives no numerical error.
The second convolution needs to be approximated by discrete convolution algorithm with grid functions,
but the numerical error will be tiny since $f_{Z_{j}}^{\text{diff}}$ only takes small values.
All these benefits are from the two conditions $f_{Z_{j}}^{\text{CV}}$ should satisfy.



\begin{figure}[!t]
\begin{framed}
\begin{algorithmic}[1]
    \STATE Evaluate $\{\mu_j\}$ by (\ref{eq:mu_init}) and (\ref{eq:mu_rec})
    \STATE $f_R \leftarrow \text{grid function of } e^{-(x-(r-\sigma^2/2)\Delta t)^2/(2\sigma^2\Delta t)}/(\sigma\sqrt{2\pi\Delta t})$
    \STATE $f_{D_{n-1}} \leftarrow \text{grid function of } e^{-x^2/(2\sigma^2\Delta t)}/(\sigma\sqrt{2\pi\Delta t})$
    \FOR{$j=n-1$ \textbf{downto} $1$ }
        \STATE Given $f_{D_{j}}$, evaluate $f_{Z_j}$ by (\ref{eq:D2Zdensity}) and interpolation
        \STATE $f_{Z_{j}}^{\text{diff}} \leftarrow f_{Z_{j}} - f_{Z_{j}}^{\text{CV}}$
        \STATE $f_{D_{j-1}} \leftarrow f_{R_{j}} \otimes f_{Z_{j}}^{\text{CV}} + f_{R_{j}} \otimes f_{Z_{j}}^{\text{diff}}$
    \ENDFOR
    \STATE Given $f_{D_0}$, evaluate $f_A$ by (\ref{eq:AB0}) and interpolation
    \STATE \textbf{return} $e^{-rT}\tilde{\mathsf E}[(A-K)^+]$
\end{algorithmic}
\end{framed}
\caption{Benhamou's Algorithm with Control Variate
} \label{algo:conv_asian_benh_CV_1}
\end{figure}



Unlike previously discussed convolution pricing algorithms, however,
in Benhamou's algorithm the discrete convolution is not the only source of error.
The interpolation in line 5 and 8 in Figure \ref{algo:conv_asian_benh} is also a numerical approximation.
This part can be enhanced by the CV technique as well.
Given the grid function of $f_{D_j}$, line 5 evaluates the grid function of $f_{Z_j}$ by interpolation and (\ref{eq:D2Zdensity}). %, which we repeat here.
%\begin{equation}
%f_{Z_{j}}(x) =
%\begin{cases}
%f_{D_{j}}\left(\log (e^{x+\mu_{j-1}}-1) - \mu_j\right)\frac{e^{x+\mu_{j-1}}}{e^{x+\mu_{j-1}}-1}&\mbox{ if }x>0\\
%0 &\mbox{ otherwise }
%\end{cases}.                \label{eq:D2ZdensityAgain}
%\end{equation}
To modify this step, we first pick a CV of $f_{D_j}$, a closed-form function $f_{D_j}^{\text{CV}}$ that approximates $f_{D_j}$.
We then find the grid function of the difference $f_{D_j}^{\text{diff}} = f_{D_j} - f_{D_j}^{\text{CV}}$ and evaluate $f_{Z_j}$ by
\begin{eqnarray}
f_{Z_{j}}(x) &=& \left( f_{D_{j}}^{\text{CV}}\left(\log (e^{x+\mu_{j-1}}-1) - \mu_j\right)
+ f_{D_{j}}^{\text{diff}}\left(\log (e^{x+\mu_{j-1}}-1) - \mu_j\right) \right) \frac{e^{x+\mu_{j-1}}}{e^{x+\mu_{j-1}}-1}\qquad\qquad \label{eq:IntpDecomp}
\end{eqnarray}
instead of (\ref{eq:D2Zdensity}).
This is only for positive $x$.
For negative $x$ or zero, we still set $f_{Z_j}(x)$ to zero.
On the right-hand side of (\ref{eq:IntpDecomp}), Since the CV $f_{D_{j}}^{\text{CV}}$ is known in closed form,
the first term in the parentheses at any grid point $x_i$ can be found exactly.
Only the second term requires an interpolation and will introduce numerical error.
But the error will be tiny, as long as $f_{D_{j}}^{\text{CV}}$ is a good approximation of $f_{D_{j}}$
so that the difference $f_{D_{j}}^{\text{diff}}$ only takes small values.
In contrast, in the original Benhamou algorithm which uses interpolation directly with (\ref{eq:D2Zdensity}),
since the variance of $D_j$ can be small, the curve $f_{D_j}$ can have a very tall and narrow shape.
Applying interpolation on this kind of function can yield significant numerical error.
That is why when the interpolation step in Benhamou's algorithm is also enhanced by the CV technique, the pricing error would be greatly improved.
The CV-enhanced algorithm is shown in Figure \ref{algo:conv_asian_benh_CV_2}.




\begin{figure}[!t]
\begin{framed}
\begin{algorithmic}[1]
    \STATE Evaluate $\{\mu_j\}$ by (\ref{eq:mu_init}) and (\ref{eq:mu_rec})
    \STATE $f_R \leftarrow \text{grid function of } e^{-(x-(r-\sigma^2/2)\Delta t)^2/(2\sigma^2\Delta t)}/(\sigma\sqrt{2\pi\Delta t})$
    \STATE $f_{D_{n-1}} \leftarrow \text{grid function of } e^{-x^2/(2\sigma^2\Delta t)}/(\sigma\sqrt{2\pi\Delta t})$
    \FOR{$j=n-1$ \textbf{downto} $1$ }
        \STATE $f_{D_{j}}^{\text{diff}} \leftarrow f_{D_{j}} - f_{D_{j}}^{\text{CV}}$
        \STATE Given $f_{D_{j}}^{\text{diff}}$, evaluate $f_{Z_j}$ by (\ref{eq:IntpDecomp}) and interpolation
        \STATE $f_{Z_{j}}^{\text{diff}} \leftarrow f_{Z_{j}} - f_{Z_{j}}^{\text{CV}}$
        \STATE $f_{D_{j-1}} \leftarrow f_{R_{j}} \otimes f_{Z_{j}}^{\text{CV}} + f_{R_{j}} \otimes f_{Z_{j}}^{\text{diff}}$
    \ENDFOR
    \STATE Given $f_{D_0}$, evaluate $f_A$ by (\ref{eq:AB0}) and interpolation
    \STATE \textbf{return} $e^{-rT}\tilde{\mathsf E}[(A-K)^+]$
\end{algorithmic}
\end{framed}
\caption{Another CV Version of Benhamou's Algorithm
} \label{algo:conv_asian_benh_CV_2}
\end{figure}



\subsection{Picking a Control Variate for Asian Options Pricing}
%In the CV-enhanced Benhamou algorithm in Figure \ref{algo:conv_asian_benh_CV_2},
%we need to find the CVs of $f_{D_j}$ and $f_{Z_j}$.
%As $f_{D_j}$ and $f_{Z_j}$ are both probability density functions,
%one possible choice is to use normal density functions with parameters estimated numerically,
As $f_{D_j}$ and $f_{Z_j}$ are both probability density functions,
for their CVs we use normal density functions with parameters estimated numerically,
like we did in discrete barrier options pricing.
Of course, just because $f_{D_j}$ and $f_{Z_j}$ are probability density functions does not mean they can be well-approximated by normal density functions,
but this choice can actually be justified.
Many literatures, including \citet{hamdan1971logarithm}, \citet{naus1969distribution} and \citet{schwartz1982distribution}
suggest that a sum of correlated log-normal random variables like (\ref{eq:A_sum}) can be well-approximated by a log-normal distribution.
Furthermore, with the average price $A$ approximated by a log-normal distribution, \citet{levy1992pricing} derives a closed-form approximation formula for Asian option prices.
Now by definition $D_j$ and $Z_j$ are just logarithms of the partial sums of $S_{t_j}$ with a shift $\mu_j$.
The fact that sum of correlated log-normal random variables can be well-approximated by a log-normal distribution implies that
both $D_j$ and $Z_j$ can be well-approximated by normal distributions.






To estimate the parameters of the density function, we also use numerical integration like we did in discrete barrier options pricing.
Given the grid function of $f_{Z_j}$, first numerically integrate $x f_{Z_j}$ to get an estimate of the mean value.
Say the estimate is $m_j$.
Then numerically integrate $(x-m_j)^2 f_{Z_j}$ to find an estimate of the variance and therefore the standard deviation.
Denote the estimated standard deviation by $s_j$.
We use
\begin{eqnarray}
f_{Z_j}^{\text{CV}}(x) = \frac{1}{s_j\sqrt{2\pi}}e^{-\frac{1}{2}\left(\frac{x-m_j}{s_j}\right)^2}\label{eq:fZCV}
\end{eqnarray}
as the CV of $f_{Z_j}$.
The same idea works to pick a CV of $f_{D_j}$.




For $f_{D_j}$, we only need the CV to be a closed-form function
so that we can evaluate the first term of the right-hand side of (\ref{eq:IntpDecomp}) without interpolation.
For $f_{Z_j}$, however, we need the closed-form formula of $f_{R_{j}} \otimes f_{Z_{j}}^{\text{CV}}$ to avoid numerical approximation
to find the first convolution in line 8 of the enhanced Benhamou algorithm in Figure \ref{algo:conv_asian_benh_CV_2}.
Since $R_j$ is normally distributed, with a CV of the form (\ref{eq:fZCV}),
$f_{R_{j}} \otimes f_{Z_{j}}^{\text{CV}}$ is simply another normal density function and is obviously known in closed form.




\subsection{Numerical Results}


As explained in Section \ref{sec:CONVBackgroundReview},
in any convolution-based pricing algorithm we first set up an interval $[-a, a]$ and a grid $-a = x_0 < x_1 < x_2 \cdots < x_m = a$.
For all numerical results in this section we set $a=2$ and change the number of grid points.


First we compare the Bermudan options pricing algorithm to its CV versions.
Consider a Bermudan put option with strike price $K=100$ and maturity $T=1$,
whose underlying asset has initial price $S_0 = 100$ and volatility $\sigma = 0.3$.
Let $r=0.1$ be the risk-free interest rate.
Assume that throughout the life of this one-year-long option,
there are 12 observation points at time $t = 1/12, 2/12, \ldots, 1$ where the option is allowed to be exercised.
For benchmark, we use the pricing result of the na\"ive convolution-based pricing algorithm with $m=102400$, which is $8.24327435926$.
Table \ref{tab:berm} shows an efficiency and accuracy comparison of three algorithms.
``Bermudan'' is the na\"ive convolution-based pricing algorithm,
``BS-CV'' its CV version with the Black-Scholes put price as the control,
and ``expCV'' is its CV version with (\ref{eq:CVem}) as the control.
The first column $m$ is the number of partitions of the interval $[-a, a]$.
``CPU (s)'' is the running time of the program in seconds, and ``log error'' columns give the pricing error taking common logarithm.
Because of the extra cost to evaluate the CV,
given the same number of partitions, both CV versions take more time than the na\"ive algorithm to accomplish the task.
But the extra cost gets less significant as $m$ goes larger and larger,
because in each loop iteration in the CV-enhanced algorithm in Figure \ref{algo:conv_berm_CV}, the CV in line 6 only takes $O(m)$ computations,
while the convolution step in line 7 using the FFT algorithm takes $O(m\log m)$ computations.
When $m$ is large, the majority of the computational time is spent on the convolution step.
Besides, given the same number of grid points the two CV versions both give more accurate pricing results than the na\"ive algorithm.
In fact, with the same computational time, the CV versions also give more accuracy than the na\"ive algorithm.
This is obvious in Figure \ref{fig:berm}.





\begin{table}[!t]
%\begin{figure}[!t]
\begin{center}
\begin{tabular}{|r|rr|rr|rr|}
\hline
{} &      \multicolumn{2}{|c|}{BS-CV}  &  \multicolumn{2}{|c|}{Bermudan}    &     \multicolumn{2}{|c|}{expCV}  \\
\hline
{$m$} &   CPU (s) & log error &   CPU (s) & log error &   CPU (s) & log error \\
\hline
100  &  0.001888 & $-$2.656520 &  0.001125 & $-$1.928496 &  0.002173 & $-$1.972770 \\
200  &  0.003415 & $-$3.380402 &  0.001569 & $-$2.548871 &  0.002956 & $-$2.846451 \\
400  &  0.005393 & $-$3.981319 &  0.002921 & $-$3.151388 &  0.005494 & $-$3.792331 \\
800  &  0.009425 & $-$4.651855 &  0.006419 & $-$3.889831 &  0.012747 & $-$4.842674 \\
1600 &  0.020544 & $-$5.353463 &  0.015812 & $-$4.478109 &  0.019162 & $-$5.547746 \\
3200 &  0.059600 & $-$6.315952 &  0.051682 & $-$5.048506 &  0.056450 & $-$6.448490 \\
6400 &  0.243433 & $-$6.890989 &  0.221945 & $-$5.604359 &  0.237604 & $-$7.224185 \\
\hline
\end{tabular}
\end{center}
\caption{
Efficiency and Accuracy Comparison of the Convolution Based Bermudan Options Pricing Algorithm and Its CV Versions.
``Bermudan'' is the na\"ive algorithm,
``BS-CV'' its CV version with the Black-Scholes-implied control,
and ``expCV'' is its CV version with (\ref{eq:CVem}) as the control.
``CPU (s)'' is the computational time in seconds,
and ``log error'' is the pricing error taking common logarithm.
}\label{tab:berm}
%\end{figure}
\end{table}



\begin{figure}[!t]
\begin{center}
    \parbox[t]{0.5\textwidth}{
        \centerline{\epsfig{figure=eps/conv.eps, width=0.5\textwidth}}
    }\hfill
    \parbox[t]{0.5\textwidth}{
        \centerline{\epsfig{figure=eps/convn.eps, width=0.5\textwidth}}
    }\hfill
\end{center}
\caption{
Efficiency and Accuracy Comparison of the Convolution Based Bermudan Options Pricing Algorithm and Its CV Versions.
``num'' is the number of grid points.
}\label{fig:berm}
\end{figure}





We get similar numerical results for Asian and discrete barrier options.
To examine the barrier options pricing results, consider a discrete up-and-out call option with strike price $K = 120$, barrier $H=130$ and maturity $T=1$,
whose underly asset has initial price $S_0 = 100$ and volatility $\sigma = 0.3$.
The risk-free interest rate is assumed to be a constant $r = 0.1$.
Assumed that there are 12 uniform observation points where the option can be knocked out.
The benchmark is 0.176355450733, obtained with number of partitions $m=102400$.
For Asian options pricing algorithms examination, we use the same numerical settings $S_0 = 100$, $K = 120$, $T=1$, $\sigma = 0.3$, $r = 0.1$.
There 13 uniform time points where the underlying asset price is observed for average evaluation, including $t=0$ and $t=T$.
The numerical results are shown in Figures \ref{tab:barrier}, \ref{fig:conv_barrier}, \ref{tab:conv_asian} and \ref{fig:conv_asian}.
The ``UOcall'' in Figures \ref{tab:barrier} and \ref{fig:conv_barrier}
refers to the na\"ive convolution-based pricing algorithm for discrete up-and-out call options,
and ``UOcallCV'' its CV version.
In Figures \ref{tab:conv_asian} and \ref{fig:conv_asian},
``Benh'' is Benhamou's algorithm, and ``BenhCV'' its CV version.
For discrete barrier option, we use the scaled normal density functions (\ref{eq:scaledBellShape}) as the CV, as described in Section \ref{sec:findingCVforBarrier}.
For Asian option, we use normal density functions (\ref{eq:fZCV}).




Same as in the Bermudan pricing, in both Asian and barrier options pricing, given the same number of partitions,
the CV versions are more time consuming algorithms because of the extra time spent on CV evaluations,
but the enhanced algorithms also get much more accuracy out of the CVs.
Overall, the CV evaluation is a worth-taking cost,
which can be seen in the graphs of pricing error against computational time in Figures \ref{fig:conv_barrier} and \ref{fig:conv_asian}.


\begin{table}[!t]
%\begin{figure}[!t]
\begin{center}
\begin{tabular}{|r|rr|rr|}
\hline
{} &     \multicolumn{2}{|c|}{UOcall}  &  \multicolumn{2}{|c|}{UOcallCV}    \\
\hline
{$m$} &   CPU (s) & log error &   CPU (s) & log error \\
\hline
200   &  0.001761 & $-$1.282222 &  0.007278 & $-$1.471718 \\
400   &  0.002679 & $-$1.588148 &  0.009405 & $-$1.762492 \\
800   &  0.005815 & $-$1.891260 &  0.013437 & $-$2.057356 \\
1600  &  0.014535 & $-$2.185156 &  0.025336 & $-$2.345303 \\
3200  &  0.049247 & $-$2.489426 &  0.063436 & $-$2.645831 \\
6400  &  0.203671 & $-$2.781629 &  0.217144 & $-$2.932702 \\
12800 &  0.642218 & $-$3.064788 &  0.696188 & $-$3.207904 \\
\hline
\end{tabular}
\end{center}
\caption{
Comparison of Discrete Barrier Options Pricing Algorithms
}\label{tab:barrier}
%\end{figure}
\end{table}



\begin{figure}[!t]
\begin{center}
    \parbox[t]{0.5\textwidth}{
        \centerline{\epsfig{figure=eps/conv_barrier.eps, width=0.5\textwidth}}
    }\hfill
    \parbox[t]{0.5\textwidth}{
        \centerline{\epsfig{figure=eps/convn_barrier.eps, width=0.5\textwidth}}
    }\hfill
\end{center}
\caption{
Efficiency and Accuracy Comparison of Discrete Barrier Options Pricing Algorithms
}\label{fig:conv_barrier}
\end{figure}






\begin{table}[!t]
%\begin{figure}[!t]
\begin{center}
\begin{tabular}{|r|rr|rr|}
\hline
{} &    \multicolumn{2}{|c|}{Benh}   &    \multicolumn{2}{|c|}{BenhCV}   \\
\hline
{$m$} &   CPU (s) & log error &   CPU (s) & log error \\
\hline
200   &  0.002641 & $-$1.605930 &  0.010837 & $-$2.365654 \\
400   &  0.003277 & $-$2.125959 &  0.014881 & $-$2.953226 \\
800   &  0.005955 & $-$2.776188 &  0.017535 & $-$3.555072 \\
1600  &  0.013135 & $-$3.394242 &  0.027068 & $-$4.157442 \\
3200  &  0.048506 & $-$3.963035 &  0.064932 & $-$4.761088 \\
6400  &  0.203419 & $-$4.582184 &  0.236476 & $-$5.370957 \\
12800 &  0.676367 & $-$5.175409 &  0.789486 & $-$6.006516 \\
\hline
\end{tabular}
\end{center}
\caption{
Comparison of Asian Options Pricing Algorithms
}\label{tab:conv_asian}
%\end{figure}
\end{table}




\begin{figure}[!t]
\begin{center}
    \parbox[t]{0.5\textwidth}{
        \centerline{\epsfig{figure=eps/conv_asian.eps, width=0.5\textwidth}}
    }\hfill
    \parbox[t]{0.5\textwidth}{
        \centerline{\epsfig{figure=eps/convn_aisan.eps, width=0.5\textwidth}}
    }\hfill
\end{center}
\caption{
Efficiency and Accuracy Comparison of Asian Options Pricing Algorithms
}\label{fig:conv_asian}
\end{figure}





\section{Pricing American Puts with Binomial Tree}
\label{Sec:AmePut}


Binomial trees for derivatives pricing are easy to understand and reasonably efficient as a numerical pricing algorithm, that is why it is popular in practice.
In this section we demonstrate how to enhance a binomial tree for American puts pricing with the CV technique.
For this purpose, first we introduce the binomial tree algorithm.



The argument starts with a discrete time world,
where at any time there can only be finitely many possible states for the price of the underlying asset.
Specifically, imagine a two period model where the underlying asset has price $S_0$ at time $t_0$.
At time $t_1$, it is assumed that the underlying asset price can only be $S_0 u$ or $S_0 d$, where $u>1$ and $d<1$ are some constants.
Let $R$ be the time $t_1$ value of one unit of the money market account and assume $d<R<u$.
Further assume that at time $t_1$ the derivative to be priced pays off $X_u$ if the underlying asset price goes up to $S_0u$,
and $X_d$ if it goes down to $S_0 d$.
Under these assumptions, it can be shown that to guarantee no arbitrage, the value of the derivative at $t_0$ must be
\begin{eqnarray}
V_0 = \frac1R\left[ \left(\frac{R-d}{u-d}\right) X_u + \left(\frac{u-R}{u-d}\right) X_d \right].\label{eq:universal_pricing_formula_discrete}
\end{eqnarray}
This is a discounted expectation under the (discrete) risk-neutral probability
\begin{eqnarray}
p = \left(\frac{R-d}{u-d}\right), \qquad q = \left(\frac{u-R}{u-d}\right)\label{eq:risk_neutral prob_discrete}
\end{eqnarray}
with discount factor $1/R$.






In reality, time is continuous, the value of an asset can change anytime,
and there are infinitely many possible states of an asset price at any specific time in the future.
In a binomial tree this is approximated by adding more time steps and more possible states of the underlying asset price.
One well-known specification by \citet{cox1979option} is as follows.
Let $r$ be the constant risk free interest rate, and $e^{rt}$ the time $t$ value of one unit of money market account.
Say we want to price a derivative with maturity $T$, and we have $n$ time steps from $t=0$ to $t=T$.
Denote $\Delta t = T/n$.
Let
\begin{eqnarray}
u = \exp(\sigma \sqrt{\Delta t}), \qquad d = 1/u\label{eq:udCRR}
\end{eqnarray}
for some constant $\sigma >0$.
At time $t=\Delta t$, one unit of money market account has value $e^{r\Delta t}$,
and the underlying asset price can either go up to $S_0 u = S_0 \exp(\sigma \sqrt{\Delta t})$ or go down to $S_0 d = S_0 \exp(-\sigma \sqrt{\Delta t})$.
These two possible states are the two tree nodes at $t=\Delta t$.
At each node, going one time step further, the underlying asset price can again either go up or down, giving us tree possible states, or nodes, at $t = 2\Delta t$.
They are $S_0 u^2$, $S_0$, and $S_0 d^2$.
In general, at $t = j\Delta t$, there are $j+1$ nodes $S_0 u^j, S_0 u^{j-2}, S_0 u^{j-4}, \ldots, S_0 d^{j}$.
At each node at each time step, by (\ref{eq:risk_neutral prob_discrete}), the underlying asset price goes up and down with risk neutral probabilities
$$
p = \left(\frac{R-d}{u-d}\right) = \frac{e^{r\Delta t} - e^{-\sqrt{\sigma \Delta t}}}{e^{\sqrt{\sigma \Delta t}} - e^{-\sqrt{\sigma \Delta t}}}, \qquad
q = \left(\frac{u-R}{u-d}\right) = \frac{e^{\sqrt{\sigma \Delta t}} - e^{r\Delta t}}{e^{\sqrt{\sigma \Delta t}} - e^{-\sqrt{\sigma \Delta t}}},
$$
respectively.
By (\ref{eq:universal_pricing_formula_discrete}),
the derivative price at a node at time $j\Delta t$ is the discounted weighted sum of the two following nodes at time $(j+1)\Delta t$ with weights $p$, $q$.
Thus, to price a derivative, one first set up the payoff at time $T = n\Delta t$.
Then (\ref{eq:universal_pricing_formula_discrete}) can be used to find the fair price of the derivative at time $(n-1)\Delta t$.
Having the price of the derivative at time $(n-1)\Delta t$,
(\ref{eq:universal_pricing_formula_discrete}) can be applied again to find the derivative price at time $(n-2)\Delta t$.
This procedure is called backward induction.
It can be used repeatedly and one obtains the derivative price backward in time until the time-0 value of the derivative is found.
This option pricing algorithm is known as the Cox-Ross-Rubinstein tree, or the CRR tree for short, in honor of \citet{cox1979option}.


The CRR tree pricing algorithm for a vanilla call option is shown in Figure \ref{algo:CRR}.
The risk-neutral probability is set up in line 4 and 5, and the payoff is set up in line 6.
Line 8 in the for loop is the backward induction step that evaluates the option values at the $(i-1)$th time step as a vector,
given the option value vector at the $i$th time step.
At time step $i$, there are $i+1$ possible states of the underlying asset value.
Thus the option value vector is $i+1$ dimensional, each element used to keep the option value for a different state.
At time step 0, there is only one state and therefore one option value, that is the initial fair price of the option returned in line 10.




It can be shown that as $n\rightarrow \infty$, the derivative price obtained by the CRR tree algorithm converges to the fair price evaluated under the Black-Scholes model
with the volatility of the underlying asset equal to $\sigma$.
Roughly speaking, at each tree node the underlying asset price goes up or down according to a fixed (risk-neutral) probability.
The log return at time $T = n\Delta t$ is therefore binomial distributed with $n+1$ possible states.
As $n\rightarrow \infty$, the central limit theorem guarantees that this binomial distribution converges to the same normal distribution as in the Black-Scholes model.



\begin{figure}[!t]
\begin{framed}
%\line(1,0){240}
\begin{algorithmic}[1]
    \STATE \textbf{input}: $S_0, K, r, s, T, n$
    \STATE $u \leftarrow e^{s\sqrt{T/n}}$
    \STATE $d \leftarrow 1/u$
    \STATE $p \leftarrow \frac{e^{rT/n} - d}{u-d}$
    \STATE $q \leftarrow 1-p $
    \STATE $\text{call} \leftarrow (\{S_0 u^n, S_0 u^{n-2}, S_0 u^{n-4}, \ldots, S_0 d^{n}\} - K)^+$
    \FOR{$i=n$ \textbf{downto} $1$ }
        \STATE $\text{call} \leftarrow e^{-rT/n}(p*\text{call}[0:i] + q*\text{call}[1 : i+1])$
    \ENDFOR
    \STATE \textbf{return} $\text{call}[0]$
\end{algorithmic}
\end{framed}
\caption{
CRR Tree Pricing Algorithm for European Call Options
}
\label{algo:CRR}
\end{figure}




\subsection{American Put and Its Pricing}




Unlike European options which can only be exercised at the maturity,
American options are options that can be exercised anytime before its maturity.
Consider a put option that gives its holder the right to sell the underlying asset at the strike price $K$.
Denote by $S_t$ the price of the underlying asset at time $t$.
If this put option is exercised at time $\tau$, the holder gets payoff $(K-S_{\tau})^+$.
Thus the fair price of this option would be the expectation under the risk-neutral probability, that is $\tilde E[e^{-r\tau}(K-S_{\tau})^+]$.
This is the fair value under the assumption that this option would be exercised at $\tau$.
A rational option holder would choose the best time $\tau$ to exercise the option,
the time that maximizes the option value based on all currently available information.
Thus the fair price of an American put is
\begin{eqnarray}
\max_{\tau \in \mathcal T_{[0, T]}} \tilde E[e^{-r\tau}(K-S_{\tau})^+],\label{eq:AmericanPut}
\end{eqnarray}
where $\mathcal T_{[0, T]}$ is the set of all stopping times in $[0, T]$.
%Stopping time is a kind of random time whose definition involves advanced concepts in probability theory, which is beyond the scope of this thesis,
%but intuitively, it just means whether or not the random time has already happened only depends on all currently available information.



The fair value of an American put can also be determined by a CRR tree pricing algorithm,
and the same as the European options pricing, when the number of time steps goes to infinity,
the pricing result given by the CRR tree converges to the theoretical fair price (\ref{eq:AmericanPut}) under the Black-Scholes model.
The only change we need to make in the CRR tree algorithm in Figure \ref{algo:CRR} is that for American options pricing,
after each backward induction step one should check the option value for all current possible states and update it if necessary.
If the option value is lower than if it is exercised, a rational investor would decide to exercise the option right away.
Specifically, let $P$ be the put value given by the backward induction for some possible state at some time step,
and $S$ the underlying asset value at the same time at the same state.
If the option is exercised and its holder sells the underlying asset at the strike price $K$ at this time, the payoff is the price difference $(K - S)$.
The CRR tree algorithm considers two scenarios.
Either keep the option with value $P$ or exercise it to get the payoff $(K - S)$.
The algorithm takes whichever is greater to be the updated put value for the state at that time.




The CRR tree pricing algorithm for American puts is shown in Figure \ref{algo:CRR_American}.
The risk-neutral probability, the terminal option value vector, and the payoff vector are set up in line 4, 5, 6 and 7,
same as the European options pricing algorithm in Figure \ref{algo:CRR}.
The backward induction step in line 9 in the loop is also the same,
but in line 10 the put value is compared to the payoff and updated if too low.
The comparison is done element by element.
The syntax of line 10 is similar to the vector operation in many programming languages or scientific libraries, such as the Python package Numpy.




\begin{figure}[!t]
\begin{framed}
%\line(1,0){240}
\begin{algorithmic}[1]
    \STATE \textbf{input}: $S_0, K, r, s, T, n$
    \STATE $u \leftarrow e^{s\sqrt{T/n}}$
    \STATE $d \leftarrow 1/u$
    \STATE $p \leftarrow \frac{e^{rT/n} - d}{u-d}$
    \STATE $q \leftarrow 1-p $
    \STATE $\text{payoff} \leftarrow (K - \{S_0 u^n, S_0 u^{n-2}, S_0 u^{n-4}, \ldots, S_0 d^{n}\})^+$
    \STATE $\text{put} \leftarrow \text{payoff}$
    \FOR{$i=n$ \textbf{downto} $1$ }
        \STATE $\text{put} \leftarrow e^{-rT/n}(p*\text{put}[0:i] + q*\text{put}[1 : i+1])$
        \STATE $\text{put} \leftarrow \max(\text{put}, \text{payoff})$
    \ENDFOR
    \STATE \textbf{return} $\text{put}[0]$
\end{algorithmic}
\end{framed}
\caption{
CRR Tree Pricing Algorithm for American Put Options
}
\label{algo:CRR_American}
\end{figure}




\subsection{Control Variate-Enhanced Algorithm}



The American put pricing algorithm in Figure \ref{algo:CRR_American} can be enhanced by the CV technique.
The backward induction step in any binomial tree algorithm is a coarse approximation of the convolution step in the convolution pricing algorithms,
and it is obviously a linear operation on the $\text{put}$ vector.
Thus it can be enhanced by the CV technique to greatly reduce the numerical error.
In the American put pricing algorithm in Figure \ref{algo:CRR_American},
in the $i$th iteration of the loop, before the backward induction step in line 9 the vector $\text{put}$ keeps the American put values at time step $i$.
We need to pick a CV that is reasonably close to the American put value, and that has a closed-form result when the convolution step is applied.
We use the European put value at time step $i$ given by the Black-Scholes formula.
If there were infinitely many time steps between time $t_i$ and $t_{i-1}$,
repeated backward induction will turn the European put value at time $t_i$ to its value at time $t_{i-1}$.
This is because the result of a CRR tree pricing algorithm for vanilla options converges to the fair price under the Black-Scholes model.
Thus the result given by the convolution step comes in closed-form and is given by the Black-Scholes put value at time $t_{i-1}$.



To apply the CV technique to the standard CRR tree algorithm in Figure \ref{algo:CRR_American},
we only modify the backward induction step in line 9, because it is the only linear operator on the $\text{put}$ vector.
The early exercise checking step in line 10 is not always linear.
Depending on the values kept in it, the $\text{put}$ vector can be partially replaced by the $\text{payoff}$ vector in this step.
In the CV-enhanced algorithm,
before sending the $\text{put}$ vector to the backward induction at time step $i$,
it is first subtracted by a vector that keeps the Black-Scholes put prices at time step $i$.
The backward induction is then applied to the difference.
The resulting vector is added by a vector of the Black-Scholes put prices at time step $i-1$ as the new $\text{put}$ vector to be checked for exercise.
The algorithm is shown in Figure \ref{algo:CRR_CV}.







\begin{figure}[!t]
\begin{framed}
%\line(1,0){240}
\begin{algorithmic}[1]
    \STATE \textbf{input}: $S_0, K, r, s, T, n$
    \STATE $u \leftarrow e^{s\sqrt{T/n}}$
    \STATE $d \leftarrow 1/u$
    \STATE $p \leftarrow \frac{e^{rT/n} - d}{u-d}$
    \STATE $q \leftarrow 1-p $
    \STATE $\text{payoff} \leftarrow (K - \{S_0 u^n, S_0 u^{n-2}, S_0 u^{n-4}, \ldots, S_0 d^{n}\})^+$
    \STATE $\text{put} \leftarrow payoff$
%    \FOR{$i=n$ \textbf{downto} $1$ }
%        \STATE $CV_i \leftarrow \text{vector of the Black-Scholes put values at time step $i$}$
%    \ENDFOR
    \FOR{$i=n$ \textbf{downto} $1$ }
        \STATE $\text{put}_{\text{diff}} \leftarrow put - \text{CV}^i$
        \STATE $\text{put}_{\text{diff}} \leftarrow e^{-rT/n}(p*\text{put}_{\text{diff}}[0:i] + q*\text{put}_{\text{diff}}[1 : i+1])$
        \STATE $\text{put} \leftarrow \text{put}_{\text{diff}} + \text{CV}^{i-1}$
        \STATE $\text{put} \leftarrow \max(\text{put}, \text{payoff})$
    \ENDFOR
    \STATE \textbf{return} $\text{put}[0]$
\end{algorithmic}
\end{framed}
\caption{
CV-Version American Put Pricing Algorithm.
$\text{CV}^i$ is a vector that keeps the Black-Scholes put values at time step $i$.
}
\label{algo:CRR_CV}
\end{figure}





In comparison with the original CRR tree algorithm in Figure \ref{algo:CRR_American}, its CV version in Figure \ref{algo:CRR_CV} yields more accurate pricing results.
The backward induction step in line 9 in Figure \ref{algo:CRR_American} is replaced by line 9, 10, and 11 in Figure \ref{algo:CRR_CV}.
When we choose the Black-Scholes put prices as the CV and subtract it from the $\text{put}$ vector,
we obtain the early exercise premium stored in the $\text{put}_{\text{diff}}$ vector.
It is premium the option holder pays for the right to exercise the option prior to maturity.
Also, the financial interpretation of the identity $\text{put} = \text{put}_{\text{diff}} + \text{CV}$ is
$$
\text{American option price} = \text{early exercise premium} + \text{European option price},
$$
except, of course, due to the accumulated error,
the early exercise premium kept in the $\text{put}_{\text{diff}}$ vector is not the theoretical true value but simply a numerical approximation.
In line 10 in Figure \ref{algo:CRR_American}, only the early exercise premium is sent to the backward induction step but not the European option price.
The backward induction step is a coarse approximation of the convolution step in the convolution pricing algorithm,
and it yields significant numerical errors.
Sending the entire American option price to such an operator would yield greater numerical error than sending the early exercise premium only.
Since the European option price in line 11 is given by the Black-Scholes formula, its value is exact.




The algorithm in Figure \ref{algo:CRR_CV} is however very inefficient, mainly because the Black-Scholes formula is evaluated at every node in the tree.
In the original CRR tree pricing algorithm in Figure \ref{algo:CRR_American},
at each node we take the discounted expected value of the put values in the previous time step (see line 9),
which only consists of very cheap arithmetic operations.
In contrast, evaluating the Black-Scholes formula requires taking logarithm, square root, exponential functions,
and two special function evaluations for the cumulated distribution function of the standard normal distribution.
Those are computationally much more expensive than backward induction.




It is possible to obtain an efficient CV-enhanced algorithm without loosing the accuracy we get from the CV technique.
First let us take another look at the original CRR tree pricing algorithm in Figure \ref{algo:CRR_American}.
In each loop iteration, the $\text{put}$ vector is sent to the backward induction step in line 9,
and the result is then compared with the payoff function in line 10
to determine if some of the option values kept in the $\text{put}$ vector are too low and need to be replaced by the payoff.
At $t = j\Delta t$, $\text{put}$ is a $j+1$ dimensional vector that keeps put values corresponding to the underlying asset prices
$S_0 u^j, S_0 u^{j-2}, S_0 u^{j-4}, \ldots, S_0 d^{j}$.
In general, the underlying asset price is in the form $S_0 u^{j-2i}$.
The greater $i$ is, the smaller the underlying asset price is, and the more likely the corresponding put value is too low and will be replaced by the payoff.
The smallest underlying asset price $S_0 u^{j-2i}$ whose corresponding put value is replaced by the payoff is known as the early exercise boundary.
It is a function of time.
Having the concept of early exercise boundary, one can achieve the same thing line 10 in Figure \ref{algo:CRR_American} does another way.
One just needs to locate the early exercise boundary and replaces all the put values with corresponding underlying asset price lower than the boundary by the payoff.
Line 12 in the CV-enhanced algorithm in Figure \ref{algo:CRR_CV} can be done in the same manner.
The CV-enhanced algorithm evaluates the Black-Scholes formula at all nodes just to locate the early exercise boundary, that is why it is so inefficient.
To speed up the algorithm, we only evaluate the Black-Scholes formula at the nodes that are close to the early exercise boundary,
only for the purpose of locating the boundary.



If we unwind the loop in Figure \ref{algo:CRR_CV}, the first few iterations would be as follows.
\begin{eqnarray*}
/* &&\text{(iteration $i=n$)} \qquad */\\
\text{put}_{\text{diff}} &\leftarrow& \text{put} - \text{CV}^n\\
\text{put}_{\text{diff}} &\leftarrow& e^{-rT/n}(p*\text{put}_{\text{diff}}[0:n] + q*\text{put}_{\text{diff}}[1 : n+1])\\
\text{put} &\leftarrow& \text{put}_{\text{diff}} + \text{CV}^{n-1}\\
\text{put} &\leftarrow& \max(\text{put}, \text{payoff})\\
/* &&\text{(iteration $i=n-1$)} \qquad */\\
\text{put}_{\text{diff}} &\leftarrow& \text{put} - \text{CV}^{n-1}\\
\text{put}_{\text{diff}} &\leftarrow& e^{-rT/n}(p*\text{put}_{\text{diff}}[0:{n-1}] + q*\text{put}_{\text{diff}}[1 : n])\\
\text{put} &\leftarrow& \text{put}_{\text{diff}} + \text{CV}^{n-2}\\
\text{put} &\leftarrow& \max(\text{put}, \text{payoff})\\
/* &&\text{(iteration $i=n-2$)} \qquad */\\
\text{put}_{\text{diff}} &\leftarrow& \text{put} - \text{CV}^{n-2}\\
\text{put}_{\text{diff}} &\leftarrow& e^{-rT/n}(p*\text{put}_{\text{diff}}[0:{n-1}] + q*\text{put}_{\text{diff}}[1 : {n-2}])\\
\text{put} &\leftarrow& \text{put}_{\text{diff}} + \text{CV}^{n-3}\\
\text{put} &\leftarrow& \max(\text{put}, \text{payoff})\\
&&\qquad\vdots
\end{eqnarray*}
We can see that before and after the early exercise checking step in iteration $i$, there are two opposite statements
\begin{eqnarray*}
\text{put} &\leftarrow& \text{put}_{\text{diff}} + \text{CV}^{i-1}\\
\text{put}_{\text{diff}} &\leftarrow& \text{put} - \text{CV}^{i-1}
\end{eqnarray*}
each evaluating $i-1$ Black-Scholes formulas for the vector $\text{CV}^{i-1}$.
This is unnecessary waste of computational time.
To speed up the algorithm, we only deal with the $\text{put}_{\text{diff}}$ vector after it is evaluated in the first iteration.
The statement
$$
\text{put} \leftarrow \max(\text{put}, \text{payoff})
$$
can be replaced by
$$
\text{put} - \text{CV}^{i-1} \leftarrow \max(\text{put} - \text{CV}^{i-1}, \text{payoff} - \text{CV}^{i-1})
$$
or equivalently
$$
\text{put}_{\text{diff}} \leftarrow \max(\text{put}_{\text{diff}}, \text{payoff} - \text{CV}^{i-1}).
$$
The CV, the Black-Scholes formula, is only evaluated near the early exercise boundary for the comparison of $\text{put}_{\text{diff}}$ and $\text{payoff} - \text{CV}^{i-1}$.
As a result, the procedure becomes
\begin{eqnarray*}
\text{put}_{\text{diff}} &\leftarrow& \vec 0\\
/* &&\text{(iteration $i=n$)} \qquad */\\
\text{put}_{\text{diff}} &\leftarrow& e^{-rT/n}(p*\text{put}_{\text{diff}}[0:n] + q*\text{put}_{\text{diff}}[1 : n+1])\\
\text{put}_{\text{diff}} &\leftarrow& \max(\text{put}_{\text{diff}}, \text{payoff} - \text{CV}^{n-1})\\
/* &&\text{(iteration $i=n-1$)} \qquad */\\
\text{put}_{\text{diff}} &\leftarrow& e^{-rT/n}(p*\text{put}_{\text{diff}}[0:{n-1}] + q*\text{put}_{\text{diff}}[1 : n])\\
\text{put}_{\text{diff}} &\leftarrow& \max(\text{put}_{\text{diff}}, \text{payoff} - \text{CV}^{n-2})\\
/* &&\text{(iteration $i=n-2$)} \qquad */\\
\text{put}_{\text{diff}} &\leftarrow& e^{-rT/n}(p*\text{put}_{\text{diff}}[0:{n-1}] + q*\text{put}_{\text{diff}}[1 : {n-2}])\\
\text{put}_{\text{diff}} &\leftarrow& \max(\text{put}_{\text{diff}}, \text{payoff} - \text{CV}^{n-3})\\
&&\qquad\vdots
\end{eqnarray*}
In the first line, instead of $\text{put} - \text{CV}^n$, we assign a zero vector to $\text{put}_{\text{diff}}$ because of two reasons.
First, this can be justified since the $\text{put}_{\text{diff}}$ vector is used to keep the early exercise premium, which is zero at maturity.
Secondly, $\text{put} - \text{CV}^n$ still requires $n$ Black-Scholes formula evaluations.
Setting $\text{put}_{\text{diff}}$ to zero vector is more efficient.
Such CV-enhanced algorithm is shown in Figure \ref{algo:CRR_CV_fast},
where in line 10 the CV is only evaluated near the early exercise boundary.
This can be viewed as a tree of the early exercise premium, instead of the put value.




\begin{figure}[!t]
\begin{framed}
%\line(1,0){240}
\begin{algorithmic}[1]
    \STATE \textbf{input}: $S_0, K, r, s, T, n$
    \STATE $u \leftarrow e^{s\sqrt{T/n}}$
    \STATE $d \leftarrow 1/u$
    \STATE $p \leftarrow \frac{e^{rT/n} - d}{u-d}$
    \STATE $q \leftarrow 1-p $
    \STATE $\text{payoff} \leftarrow (K - \{S_0 u^n, S_0 u^{n-2}, S_0 u^{n-4}, \ldots, S_0 d^{n}\})^+$
    \STATE $\text{put}_{\text{diff}} \leftarrow \text{$n$ dimensional zero vector}$
    \FOR{$i=n$ \textbf{downto} $1$ }
        \STATE $\text{put}_{\text{diff}} \leftarrow e^{-rT/n}(p*\text{put}_{\text{diff}}[0:{n-1}] + q*\text{put}_{\text{diff}}[1 : {n-2}])$
        \STATE $\text{put}_{\text{diff}} \leftarrow \max(\text{put}_{\text{diff}}, \text{payoff} - \text{CV}^{i-1})$
    \ENDFOR
    \STATE \textbf{return} $(\text{put}_{\text{diff}}[0] + \text{Black-Scholes price at 0})$
\end{algorithmic}
\end{framed}
\caption{
Efficient CV Version of the American Put Pricing Algorithm.
$\text{CV}_i$ is a vector that keeps the Black-Scholes put values at time step $i$.
}
\label{algo:CRR_CV_fast}
\end{figure}


Only the put values corresponding to underlying asset prices that are greater than the early exercise boundary need to be evaluated.
Below the early exercise boundary, the corresponding put values at all nodes are replaced by the payoff.
Since the early exercise boundary is continuous,
every time the algorithm goes one step backward, to locate the boundary
we just need to evaluate $\text{payoff} - \text{CV}^{i-1}$, and hence the Black-Scholes formula, at the nodes that are close to the boundary of the previous time step.
In a way, we find the early exercise boundary recursively.
The initial value of this recursion is the early exercise boundary at maturity, which is $S=K$.




\subsection{Hull and White's Algorithm}



There is another American put pricing algorithm proposed by \citet{hull1988use} which is based on binomial tree and the CV technique,
but the framework is different from ours.
While both Hull and White and our algorithms use the Black-Scholes formula as the CV,
in our algorithm a different CV is applied in every time step.
Overall $n$ CVs are used, $n$ the number of time steps.
In Hull and White's algorithm, the CV technique runs alongside the CRR tree.
In their algorithm, a CRR tree similar to Figure \ref{algo:CRR} is first used to determine the value of the European put with the same set of parameters.
Say the result is $P_E$.
Another CRR tree is than used to determine the American put value, just like in Figure \ref{algo:CRR_American}.
Suppose the result is $P_A$.
Theoretically, when the number of time steps goes to infinity, $P_E$ converges to the put value evaluated by the Black-Scholes formula,
and $P_A$ also converges to the theoretical no-arbitrage price (\ref{eq:AmericanPut}).
Thus $P_A - P_E$ converges to the early exercise premium.
Denote by $P_{BS}$ the time-0 put value given by the Black-Scholes formula, then
$$
P_A - P_E + P_{BS}
$$
converges to the American put value.
%The algorithm is shown in Figure \ref{algo:CRR_Hull}.
%Line 8, 9 and 10 is the loop evaluating $P_E$, and the loop in line 12, 13, 14 and 15 evaluates $P_A$.


With the same finite number of time steps $n$, if the numerical errors of $P_A$ and $P_E$ are in the same range,
we find the difference $P_A - P_E$ in the hope that the errors will cancel out.
This algorithm, and the explanation why it could give more accurate pricing result, can also be found in \citet{hull2006options}.



%
%\begin{figure}[!t]
%\begin{framed}
%%\line(1,0){240}
%\begin{algorithmic}[1]
%    \STATE \textbf{input}: $S_0, K, r, s, T, n$
%    \STATE $u \leftarrow e^{s\sqrt{T/n}}$
%    \STATE $d \leftarrow 1/u$
%    \STATE $p \leftarrow \frac{e^{rT/n} - d}{u-d}$
%    \STATE $q \leftarrow 1-p $
%    \STATE $payoff \leftarrow (K - \{S_0 u^n, S_0 u^{n-2}, S_0 u^{n-4}, \ldots, S_0 d^{n}\})^+$
%    \STATE $put^{(E)} \leftarrow payoff$
%    \FOR{$i=n$ \textbf{downto} $1$ }
%        \STATE $put^{(E)} \leftarrow e^{-rT/n}(p*put^{(E)}[0:i] + q*put^{(E)}[1 : i+1])$
%    \ENDFOR
%    \STATE $put^{(A)} \leftarrow payoff$
%    \FOR{$i=n$ \textbf{downto} $1$ }
%        \STATE $put^{(A)} \leftarrow e^{-rT/n}(p*put^{(A)}[0:i] + q*put^{(A)}[1 : i+1])$
%        \STATE $put^{(A)} \leftarrow \max(put^{(A)}, payoff)$
%    \ENDFOR
%    \STATE \textbf{return} $(put^{(A)}[0] - put^{(E)}[0] + \text{Black-Scholes price at 0})$
%\end{algorithmic}
%\end{framed}
%\caption{
%Hull and White's Pricing Algorithm for American Put Options
%}
%\label{algo:CRR_Hull}
%\end{figure}






\subsection{Numerical Results}



Consider an American put option with strike price $K=40$ and maturity $T=3$.
Suppose the underlying asset has initial price $S_0 = 40$ and volatility $\sigma = 0.3$.
Let the constant risk-free interest rate be $r = 0.1$.
The traditional CRR tree algorithm in Figure \ref{algo:CRR_American} with number of partitions $n=102400$ suggests that
the put option has value 1.237687, which we will use as a benchmark.



We compare the pricing results of Hull and While's algorithm, the traditional CRR tree, and its CV version in Figure \ref{algo:CRR_CV_fast}.
The computational times and the absolute errors are shown in Table \ref{tab:Tree} for different numbers of time steps.
The pricing results by Hull and White's algorithm are less accurate for small $n$, but for $n>400$, the accuracy is always better than the CRR tree.
Given the same number of time steps, the CV version always gives better pricing results than both the CRR tree and Hull and White's algorithm.
The only exception is when $n = 3200$ Hull and White's algorithm gives a bit more accuracy than the CV version.
This is in fact an example of the instability of Hull and White's algorithm, which will be discussed later.




The running times listed in Table \ref{tab:Tree} shows that
for small $n$ the CV version is slower than both the CRR tree and Hull and White's algorithm.
This is because the CRR tree and Hull and White's algorithm only perform the backward induction step at each node,
which is a much computationally cheaper operation than the Black-Scholes formula evaluation needed in our CV version.
This difference in speed is less significant when $n$ goes large.
The reason is that in the CV version the Black-Scholes formula is only evaluated around the early exercise boundary,
so it is only evaluated $O(n)$ times,
while in all three algorithms the backward induction at all nodes requires $O(n^2)$ computations.
When $n>1600$, the CV-enhanced algorithm starts to go faster than Hull and White's.
When $n>3200$, it gives the best numerical results, in terms of both efficiency and accuracy.




\begin{table}[!t]
%\begin{figure}[!t]
\begin{center}
\begin{tabular}{|r|rr|rr|rr|}
\hline
{} &       \multicolumn{2}{|c|}{CRR}  &  \multicolumn{2}{|c|}{CV}   &   \multicolumn{2}{|c|}{HW}   \\
\hline
{$n$} & Abs error &    CPU (s) & Abs error &   CPU (s) & Abs error &    CPU (s) \\
\hline
100   &  0.003601 &   0.007769 &  0.002413 &  0.034853 &  0.004081 &   0.011390 \\
200   &  0.001796 &   0.016581 &  0.001217 &  0.056165 &  0.002048 &   0.024034 \\
400   &  0.000939 &   0.036914 &  0.000657 &  0.108347 &  0.000984 &   0.052816 \\
800   &  0.000503 &   0.093100 &  0.000363 &  0.222466 &  0.000458 &   0.125053 \\
1600  &  0.000247 &   0.251381 &  0.000178 &  0.465793 &  0.000233 &   0.319739 \\
3200  &  0.000151 &   0.788970 &  0.000116 &  0.888058 &  0.000089 &   0.915061 \\
6400  &  0.000068 &   2.589824 &  0.000050 &  1.831340 &  0.000053 &   2.849856 \\
12800 &  0.000029 &   9.503834 &  0.000021 &  3.861430 &  0.000031 &   9.931539 \\
25600 &  0.000013 &  34.190770 &  0.000009 &  7.750417 &  0.000017 &  36.476493 \\
\hline
\end{tabular}
\end{center}
\caption{
Efficiency and Accuracy Comparison of Various Binominal Tree Algorithms.
CRR refers to the traditional CRR tree algorithm in Figure \ref{algo:CRR_American},
HW is Hull and White's algorithm, % in Figure \ref{algo:CRR_Hull},
and CV is our CV version of the CRR tree, the algorithm in Figure \ref{algo:CRR_CV_fast}.
}\label{tab:Tree}
%\end{figure}
\end{table}



Figure \ref{fig:Tree} is a picture of the pricing result against the number of time steps.
As shown in the figure, while the results by all three algorithms converge to the benchmark 1.237687,
our CV version is the most stable.
This is because our CV-enhanced algorithm is a tree of the early exercise premiums instead of the put values.
While the put value has a discontinuity in the first derivative at $S_T = K$,
the early exercise boundary is just a zero function at maturity.
In the literature, the discontinuity in the first derivative has been identified as the source of the oscillatory convergence.
See for example \citet{leisen1998pricing}, \citet{klassen2001simple}.
\citet{klassen2001simple} suggests using
$$
u = e^{\sigma\sqrt{T/n} + \frac{\log(K/S_0)}{n}}, \qquad d = e^{-\sigma\sqrt{T/n} + \frac{\log(K/S_0)}{n}}
$$
instead of the CRR specification (\ref{eq:udCRR}),
and shows that this choice of $u$, $d$ values can eliminate the oscillation of the numerical results.
With a tree of the early exercise premiums, there is no discontinuity of the first derivative
and therefore no need to change the CRR specification.
That is why the convergence of our CV version shown in Figure \ref{fig:Tree} seems more stable than other algorithms.






\begin{figure}[!t]
\begin{center}
    \parbox[t]{0.6\textwidth}{
        \centerline{\epsfig{figure=eps/Tree.eps, width=0.6\textwidth}}
    }\hfill
\end{center}
\caption{
Pricing Results Against Number of Time Steps
}\label{fig:Tree}
\end{figure}





\section{Conclusion}
In a general framework we extend the CV technique to enhance various numerical option pricing algorithms.
Unlike in the Monte Carlo simulation where the CV is applied to the algorithm as a whole,
in our framework CVs can be applied to some steps in the algorithm, which makes the framework more general.
An algorithm as a whole might not satisfy the conditions under which it can benefit from the use of CVs, but some of its individual parts may.
As examples we demonstrate how to apply the CV technique to enhance Carr and Madan's algorithm,
the convolution-based pricing algorithms, and binomial trees.
Numerical results show that the CV-enhanced algorithms are more efficient and accurate than the original algorithms in all cases considered.




One advantage the general framework inherits from the traditional CV technique is the flexibility of choosing CV.
Using Carr and Madan's algorithm and the convolution-based Bermudan puts pricing algorithm as examples,
we show that different choices of CVs can lead to different pricing accuracy.
By choosing better CVs, it is possible to design more accurate pricing algorithms in this new framework.
Moreover, in some cases, like the convolution-based pricing algorithms, the target to be approximated by CVs is easy to visualize,
which makes it easier to pick good CVs than the Monte Carlo simulation.







\appendix






\section{Heston Model and Its Characteristic Function}
\label{appendix:heston}
Under the Heston stochastic-volatility model, the price process $S_t$ of the underlying asset of an option is modeled by
\begin{eqnarray*}
\begin{cases}
dS_t &= r S_tdt + \sqrt{V_t} S_td\tilde W^S_t\\
dV_t &= \kappa(\theta - V_t) dt + \sigma\sqrt{V_t} d\tilde W^V_t\\
\end{cases},
\end{eqnarray*}
where $r>0$ is the risk-free interest rate,
$\tilde W^S_t$ and $\tilde W^V_t$ are standard Brownian motions with correlation $d\tilde W^S_td\tilde W^V_t = \rho dt$, $\rho\in(-1, 1)$,
and the parameters $\kappa>0$, $\theta>0$, $\sigma>0$.
The characteristic function of the log-return $X_T = \log(S_T/S_0)$ is known in closed form as
\begin{eqnarray*}
\phi_T(u) = \tilde E\left[e^{iuX_T}\right] = e^{C(u, T)+D(u, T)V_0},%\label{eq:chfHeston}
\end{eqnarray*}
where
\begin{eqnarray*}
C(u, \tau) &=& iru\tau + \frac{\kappa \theta  }{\sigma^2} \left( \left(\kappa -i\rho\sigma u - d\right)\tau - 2\log\left[\frac{1-g e^{-d\tau}}{1-g}\right] \right),\\
D(u, \tau) &=& \frac{ \kappa -i\rho\sigma u - d}{\sigma^2} \left[\frac{1-e^{-d\tau}}{1 - g e^{-d\tau}}\right],\\
g &=& \frac{ \kappa -i\rho\sigma u - d}{\kappa -i\rho\sigma u + d}~,\\
d &=& \sqrt{\left(\kappa -i\rho\sigma u \right)^2 + \sigma^2\left(iu+ u^2\right)}.
\end{eqnarray*}
This is a different formulation from the characteristic function formula first derived in the appendix of \citet{heston1993closed}.
\citet{lord2010complex} prove that with this formulation
taking the principal branch of the the complex logarithm in $C(\tau)$ will always give us the correct characteristic function value.




%\bibliographystyle{plain}
\bibliographystyle{chicago}
\bibliography{GCV_db}





\end{document}
